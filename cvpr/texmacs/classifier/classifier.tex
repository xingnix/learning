\documentclass{beamer}
\usepackage[english]{babel}
\usepackage{amsmath,graphicx}

%%%%%%%%%% Start TeXmacs macros
\catcode`\<=\active \def<{
\fontencoding{T1}\selectfont\symbol{60}\fontencoding{\encodingdefault}}
\catcode`\>=\active \def>{
\fontencoding{T1}\selectfont\symbol{62}\fontencoding{\encodingdefault}}
\newcommand{\nospace}{}
\newcommand{\tmfoldedstd}[2]{\trivlist{\item[$\bullet$]\mbox{}#1}}
\newcommand{\tmmathbf}[1]{\ensuremath{\boldsymbol{#1}}}
\newcommand{\tmop}[1]{\ensuremath{\operatorname{#1}}}
\newenvironment{itemizedot}{\begin{itemize} \renewcommand{\labelitemi}{$\bullet$}\renewcommand{\labelitemii}{$\bullet$}\renewcommand{\labelitemiii}{$\bullet$}\renewcommand{\labelitemiv}{$\bullet$}}{\end{itemize}}
%%%%%%%%%% End TeXmacs macros

\begin{document}

{\screens{\begin{frame}
  \
  
  \
  
  \
  
  \
  
  \
  
  \title{计算视觉与模式识别}
  
  \maketitle
  
  \ 
\end{frame}}{\begin{frame}
  \frametitle{分类器}
  
  数据集
  \begin{eqnarray*}
    (x_1, y_1) & \cdots & (x_n, y_n)
  \end{eqnarray*}
  分类
  \begin{eqnarray*}
    x_{\tmop{new}} & \rightarrow & ?
  \end{eqnarray*}
\end{frame}}{\begin{frame}
  \frametitle{风险函数}
  
  \
  
  损失函数
  \begin{eqnarray*}
    L (i \rightarrow j) & = & \left\{\begin{array}{l}
      l > 0\\
      0
    \end{array}\right. \qquad \begin{array}{l}
      i \neq j\\
      i = j
    \end{array}
  \end{eqnarray*}
  
  
  风险函数
  \begin{eqnarray*}
    R (s) & = & P r\{1 \rightarrow 2| \tmop{using} s\}L (1 \rightarrow 2) + P
    r\{2 \rightarrow 1| \tmop{using} s\}L (2 \rightarrow 1)
  \end{eqnarray*}
  
  
  \ 
\end{frame}}{\begin{frame}
  \frametitle{分类边界}
  \begin{eqnarray*}
    &  & P\{\tmop{class} \tmop{is} 2| x\}L (2 \rightarrow 1) +
    P\{\tmop{class} \tmop{is} 1| x\}L (1 \rightarrow 1)\\
    & = & P\{\tmop{class} \tmop{is} 2| x\}L (2 \rightarrow 1) + 0\\
    & = & p (2| x) L (2 \rightarrow 1)\\
    &  & P\{\tmop{class} \tmop{is} 1| x\}L (1 \rightarrow 2)\\
    & = & p (1| x) L (1 \rightarrow 2)
  \end{eqnarray*}
  边界上
  \begin{eqnarray*}
    p (2| x) L (2 \rightarrow 1) & = & p (1| x) L (1 \rightarrow 2)
  \end{eqnarray*}
  得
  \begin{eqnarray*}
    p (x| 2) p (2) L (2 \rightarrow 1) & = & p (x| 1) p (1) L (1 \rightarrow
    2)
  \end{eqnarray*}
\end{frame}}{\begin{frame}
  \frametitle{分类}
  
  \
  
  \
  
  
  \begin{eqnarray*}
    x \rightarrow 1 \hspace{5em} p (x| 2) p (2) L (2 \rightarrow 1) & < & p
    (x| 1) p (1) L (1 \rightarrow 2)\\
    x \rightarrow 2 \hspace{5em} p (x| 2) p (2) L (2 \rightarrow 1) & > & p
    (x| 1) p (1) L (1 \rightarrow 2)
  \end{eqnarray*}
\end{frame}}{\begin{frame}
  \frametitle{多类别贝叶斯分类器}
  
  损失函数
  \begin{itemizedot}
    \item $\exists k, \Pr (k|x) > 1 - d$
    \begin{eqnarray*}
      L (i \rightarrow j) & = & \left\{\begin{array}{ll}
        1 & i \neq j\\
        0 & i = j
      \end{array}\right.
    \end{eqnarray*}
    \item $\forall k, \Pr (k|x) < 1 - d$,d<1
    \begin{eqnarray*}
      L (i \rightarrow j) & = & d
    \end{eqnarray*}
  \end{itemizedot}
  选择类别
  \begin{eqnarray*}
    c & = & \arg \max_k \Pr (k|x)
  \end{eqnarray*}
\end{frame}}{\begin{frame}
  \
  
  \resizebox{1\columnwidth}{!}{\includegraphics{img/decision_boundary.png}}
\end{frame}}{\begin{frame}
  \frametitle{Logistic regression}
  \begin{eqnarray*}
    \log \frac{p (1|\tmmathbf{x})}{p (- 1|\tmmathbf{x})} & = & \tmmathbf{a}^T
    \tmmathbf{x}\\
    p (1|\tmmathbf{x}) & = & \frac{e^{\tmmathbf{a}^T \tmmathbf{x}}}{1 +
    e^{\tmmathbf{a}^T \tmmathbf{x}}}\\
    p (- 1|\tmmathbf{x}) & = & \frac{1}{1 + e^{\tmmathbf{a}^T \tmmathbf{x}}}\\
    L & = & - \sum_i \frac{1 + y_i}{2} \tmmathbf{a}^T \tmmathbf{x}_i - \ln (1
    + e^{\tmmathbf{a}^T \tmmathbf{x}_i})\\
    \hat{a} & = & \arg \min_a L
  \end{eqnarray*}
  \tmfoldedstd{note}{\begin{eqnarray*}
    p (1|\tmmathbf{x}) & = & \frac{1}{1 + e^{-\tmmathbf{a}^T \tmmathbf{x}}}\\
    p (- 1|\tmmathbf{x}) & = & \frac{1}{1 + e^{\tmmathbf{a}^T \tmmathbf{x}}}\\
    p (y|\tmmathbf{x}) & = & \frac{1}{1 + e^{- y\tmmathbf{a}^T \tmmathbf{x}}}
    \qquad (y = \pm 1)\\
    L (y_i, \gamma_i) & = & - \ln \nospace [p (y_i |\tmmathbf{x}_i)]\\
    & = & \ln (1 + e^{- y_i \gamma_i}) \qquad (\gamma_i =\tmmathbf{a}^T
    \tmmathbf{x}_i)
  \end{eqnarray*}}
\end{frame}}{\begin{frame}
  \frametitle{正则化}
  
  
  \begin{eqnarray*}
    L & = & - \sum_i \frac{1 + y_i}{2} \tmmathbf{a}^T \tmmathbf{x}_i - \ln (1
    + e^{\tmmathbf{a}^T \tmmathbf{x}_i}) + \lambda \| \tmmathbf{a} \|_p
  \end{eqnarray*}
  \begin{itemizedot}
    \item $p = 2$
    \begin{eqnarray*}
      \| \tmmathbf{a} \|_2 & = & \lambda \tmmathbf{a}^T \tmmathbf{a}
    \end{eqnarray*}
    \item p=1
    \begin{eqnarray*}
      \| \tmmathbf{a} \|_1 & = & \sum_i | a_i |
    \end{eqnarray*}
  \end{itemizedot}
\end{frame}}{\begin{frame}
  \resizebox{1\columnwidth}{!}{\includegraphics{img/decision_boundary_only.png}}
  
  \ 
\end{frame}}{\begin{frame}
  \frametitle{正态类别条件分布}
  \begin{eqnarray*}
    \tmmathbf{\mu}_k & = & \frac{1}{N_k} \sum_{i = 1}^{N_k} \tmmathbf{x}_{k,
    i}\\
    \tmmathbf{\Sigma}_k & = & \frac{1}{N_k - 1} \sum_{i = 1}^{N_k}
    (\tmmathbf{x}_{k, i} -\tmmathbf{\mu}_k) (\tmmathbf{x}_{k, i}
    -\tmmathbf{\mu}_k)^T
  \end{eqnarray*}
  选择
  \begin{eqnarray*}
    c & = & \arg \min_k \delta (\tmmathbf{x}; \tmmathbf{\mu}_k,
    \tmmathbf{\Sigma}_k)^2 - \Pr \{ k \}
  \end{eqnarray*}
  其中
  \begin{eqnarray*}
    \delta (\tmmathbf{x}; \tmmathbf{\mu}_k, \tmmathbf{\Sigma}_k) & = &
    \frac{1}{2} ((\tmmathbf{x}-\tmmathbf{\mu}_k)^T \tmmathbf{\Sigma}_k^{- 1}
    (\tmmathbf{x}-\tmmathbf{\mu}_k))^{1 / 2}
  \end{eqnarray*}
\end{frame}}{\begin{frame}
  \frametitle{k邻分类器}
  
  对于特征向量$\tmmathbf{x}$
  \begin{itemizedot}
    \item 确定距离$\tmmathbf{x}$最近的k个训练样例
    
    \item 确定k个样例中属于各类别的样例数
    
    \item 有最多样例的类别记作c,个数记作n
    
    \item $n > l$时$x \in c$,否则拒绝分类
  \end{itemizedot}
\end{frame}}{\begin{frame}
  \frametitle{估计、提高性能}
  \begin{itemizedot}
    \item Cross Validation
    
    划分数据集,交替选择训练集与测试集
    
    \item Bootstrapping
    
    部分数据集用于训练,余下的测试,出错样例再次训练
  \end{itemizedot}
\end{frame}}{\begin{frame}
  \frametitle{皮肤像素分类}
  
  \
  
  比较:
  \begin{eqnarray*}
    \frac{p (\tmmathbf{x}| \tmop{skin}) p (\tmop{skin})}{p (\tmmathbf{x})} L
    (\tmop{skin} \rightarrow \tmop{not} \tmop{skin}) &  & \\
    \frac{p (\tmmathbf{x}| \tmop{not} \tmop{skin}) p (\tmop{not}
    \tmop{skin})}{p (\tmmathbf{x})} L (\tmop{not} \tmop{skin} \rightarrow
    \tmop{skin}) &  & 
  \end{eqnarray*}
\end{frame}}{\begin{frame}
  \
  
  {\hspace{6em}}\resizebox{0.6\columnwidth}{!}{\includegraphics{img/skin_sample.png}}
\end{frame}}{\begin{frame}
  \
  
  \resizebox{1\columnwidth}{!}{\includegraphics{img/skin_roc.png}}
\end{frame}}{\begin{frame}
  \frametitle{朴素贝叶斯}
  
  
  \begin{eqnarray*}
    P (\tmop{image} | \tmop{face}) & = & P (\tmop{label} 1 \tmop{at} (x 1, y
    1), \ldots, \tmop{label} k \tmop{at} (\tmop{xk}, \tmop{yk}) |
    \tmop{face})\\
    & = & P (\tmop{label} 1 \tmop{at} (x 1, y 1) | \tmop{face}) \cdots P
    (\tmop{label} k \tmop{at} (\tmop{xk}, \tmop{yk}) | \tmop{face})
  \end{eqnarray*}
\end{frame}}{\begin{frame}
  \frametitle{PCA}
  \begin{eqnarray*}
    v (\tmmathbf{x}_i) & = & \tmmathbf{v}^T (\tmmathbf{x}_i -\tmmathbf{\mu})\\
    \tmop{var} (\tmmathbf{v}) & = & \frac{1}{n - 1} v (\tmmathbf{x}_i) v
    (\tmmathbf{x}_i)^T\\
    & = & \frac{1}{n - 1} \sum_{i = 1}^n \tmmathbf{v}^T (\tmmathbf{x}_i
    -\tmmathbf{\mu}) (\tmmathbf{x}_i -\tmmathbf{\mu})^T \tmmathbf{v} \\
    & = & \tmmathbf{v}^T \Sigma \tmmathbf{v}
  \end{eqnarray*}
\end{frame}}{\begin{frame}
  \
  
  {\hspace{5em}}\resizebox{0.7\columnwidth}{!}{\includegraphics{img/pca.png}}
  
\end{frame}}{\begin{frame}
  \frametitle{Canonical Variates}
  \begin{eqnarray*}
    \overline{\tmmathbf{\mu}} & = & \frac{1}{g} \sum_{j = 1}^g
    \tmmathbf{\mu}_j\\
    \mathcal{B} & = & \frac{1}{g - 1} \sum_{j = 1}^g (\tmmathbf{\mu}_j -
    \overline{\tmmathbf{\mu}}) (\tmmathbf{\mu}_j -
    \overline{\tmmathbf{\mu}})^T
  \end{eqnarray*}
  最大化
  \[ \frac{\tmmathbf{v}_1^T \mathcal{B}\tmmathbf{v}_1}{\tmmathbf{v}_1^T \Sigma
     \tmmathbf{v}_1} \]
  得
  \begin{eqnarray*}
    \mathcal{B}\tmmathbf{v}_1 + \lambda \Sigma \tmmathbf{v}_1 & = & 0
  \end{eqnarray*}
\end{frame}}{\qquad\resizebox{0.8\columnwidth}{!}{\includegraphics{img/canonical_variate.png}}

\tmfoldedstd{Figure 25.10}{Principal component analysis doesn't take into
account the fact that there may be more than one class of item in a dataset.
This can lead to significant problems. For a classifier, we would like to
obtain a set of features that firstly reduces the number of features and
secondly makes the difference between classes most obvious. For the data set
on the top, one class is indicated by circles and the other by stars. PCA
would suggest projection onto a vertical axis, which captures the variance in
the dataset, but cannot be used to discriminate it, as we can see from the
axes obtained by PCA, which are overlaid on the data set. The bottom row shows
the projections onto those axes. On the bottom left, we show the projection
onto the first principal component --- which has higher variance, but
separates the classes poorly --- and on the bottom right, we show the
projection onto the second principal component --- which has significantly
lower variance (look at the axes) and gives better separation.}}}

\end{document}
