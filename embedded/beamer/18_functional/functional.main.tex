\begin{document}

\def\lecturename{嵌入式技术}

\title{\insertlecture}

\author{邢超}

\institute
{
  西北工业大学航天学院
}

%\mode<presentation>{\subject{嵌入式系统}}

%  start a lecture  --------------------------
\lecture[EC]{嵌入式技术}{}
\subtitle{函数式程序设计}
\date{2012}


%\setbeamertemplate{background}{\pgfimage[width=\paperwidth,height=\paperheight]{image/flower}}
%\setbeamercovered{transparent}
%\mode<presentation>{\beamerdefaultoverlayspecification{<+->}}

\begin{frame}
  \maketitle
\end{frame}


\section{函数式语言介绍}
\begin{frame}[containsverbatim]{命令式语言imperative language}
\begin{itemize}
\item 结果与运算次序有关
\item 副作用side effect
\end{itemize}
\begin{lstlisting}
int a;
int f(int x){
a=x;
return x;
}
int main(int argc, char ** argv){
f(0);
f(1);
return 0;
}
\end{lstlisting}
\end{frame}

\begin{frame}{函数式语言}
\begin{itemize}
\item immutable data
\item Pure function
\item First class function
\item Recursion
\end{itemize}
\end{frame}

\begin{frame}[containsverbatim]{Tail Recursion}
\begin{lstlisting}
#include <sys/time.h>
#include <stdio.h>
#include <stdlib.h>

long long fib0(long long n, long long fib_n){
  if(n>2) return fib0(n-1)+fib0(n-2);
  else return 1;
}

long long fib(long long n, long long N,
          long long fib_ns1, long long fib_n){
  if(n<N) return fib(n+1,N,fib_n,fib_ns1+fib_n);
  else return fib_n;
}
int main(int argc, char **argv){
   fib0(atoi(argv[1]));
   fib(2,atoi(argv[1]),1,1);
return 0;
}
\end{lstlisting}
\end{frame}

\begin{frame}{函数式语言列表}
\begin{center}\pgfimage[width=0.9\columnwidth]{image/ComputerLanguagesChart}\end{center}
\end{frame}

\section{Lisp}
\section{Scheme}
\section{Ocaml}
\begin{frame}[containsverbatim]{curry}
\lstset{language=[objective]caml}
\begin{lstlisting}
# let rec fact n = if n<2 then 1 else n*fact(n-1) ;;
val fact : int -> int = <fun>
# fact 8 ;;
- : int = 40320
# let next x = x+1;;
val next : int -> int = <fun>
# let compose f g x = f(g(x));;
val compose : ('a -> 'b) -> ('c -> 'a) -> 'c -> 'b = <fun>
# let weird = compose fact next;;
val weird : int -> int = <fun>
# weird 7;;
- : int = 40320
# compose fact next 7;;
- : int = 40320
\end{lstlisting}
\end{frame}

\begin{frame}[containsverbatim]{pattern matching}
\lstset{language=[objective]caml}
\begin{lstlisting}
$ ledit ocaml
        Objective Caml version 3.06
# let rec sum lst =
    match lst with
      [] -> 0
    | head :: tail -> head + sum tail ;;
val sum : int list -> int = <fun>
# sum [ 1; 2; 3 ] ;;
- : int = 6
\end{lstlisting}
\end{frame}


\begin{frame}[containsverbatim]{List}
\lstset{language=[objective]caml}
\begin{lstlisting}
# List.map ((+) 2)  [1 ; 2 ; 3];;
- : int list = [3; 4; 5]
# List.fold_left (+) 0 [1 ;2; 3] / 3;;
- : int = 2
# List.filter ((>) 5) [1;2;9;10];;
- : int list = [1; 2]

let rec fold_right f a lst = match lst with
  | [] -> a
  | x :: xs -> f x (fold_right f a xs);;
(* fold_right f 0 [1;2;3] = f 1 (f 2 (f 3 0)) *)
let rec fold_left f a lst = match lst with
  | [] -> a
  | x :: xs -> fold_left f (f a x) xs
(* fold_left f 0 [1;2;3] = f (f (f 0 1) 2) 3 *)
\end{lstlisting}
\end{frame}

\section{Haskell}


\section{思考}
\begin{frame}{思考}
\begin{itemize}
\item 常见的函数式程序设计语言有哪些?特点是什么?
\item 如何利用函数式语言的优点更好地设计C/C++程序?
\end{itemize}
\end{frame}

\end{document}
