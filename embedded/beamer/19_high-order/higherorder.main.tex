\begin{document}

\def\lecturename{嵌入式技术}

\title{\insertlecture}

\author{邢超}

\institute
{
  西北工业大学航天学院
}

%\mode<presentation>{\subject{嵌入式系统}}

%  start a lecture  --------------------------
\lecture[EC]{嵌入式技术}{}
\subtitle{高阶编程}
\date{2013}


%\setbeamertemplate{background}{\pgfimage[width=\paperwidth,height=\paperheight]{image/flower}}
%\setbeamercovered{transparent}
%\mode<presentation>{\beamerdefaultoverlayspecification{<+->}}

\begin{frame}
  \maketitle
\end{frame}


\section{Standard Template Library (STL)}
\begin{frame}[containsverbatim]{Expression Template}
\lstset{language=c++}
\begin{lstlisting}
template <int dim, class T>
struct inner_product
{
    T operator()(T* a, T* b)
    {
        return inner_product<dim - 1, T>()(a + 1, b + 1) +
               inner_product<1, T>()(a, b);
    }
};

template <class T>
struct inner_product<1, T>
{
    T operator()(T* a, T* b)
    { return (*a) * (*b); }
};

inner_product<4, int>()(a, b);

\end{lstlisting}
\end{frame}

\begin{frame}[containsverbatim]{Fibonacci Sequence at compile time}
\lstset{language=c++}
\begin{lstlisting}
#include <iostream>
#include <cassert>
using namespace std;
 
template<int stage> struct Fib {
  static const uint64_t value =
              Fib<stage-1>::value + Fib<stage-2>::value;
   static inline uint64_t getValue(int i)
  {
    if (i == stage) 
    {
      return value; 
    } else {
      return Fib<stage-1>::getValue(i); 
    }
  }
};
\end{lstlisting}
\end{frame}
 
\begin{frame}[containsverbatim]{Fibonacci Sequence at compile time}
\lstset{language=c++}
\begin{lstlisting}
template<> // Template specialization for the 0's case.
struct Fib<0>
{
  static const uint64_t value = 1;
 
  static inline uint64_t getValue(int i)
  {
    assert(i == 0);
    return 1;
  }
};
\end{lstlisting}
\end{frame}
 
\begin{frame}[containsverbatim]{Fibonacci Sequence at compile time}
\lstset{language=c++}
\begin{lstlisting}
template<> // Template specialization for the 1's case
struct Fib<1>
{
  static const uint64_t value = 1;
 
  static inline uint64_t getValue(int i)
  {
    if (i == 1)
    {
      return value;
    } else {
      return Fib<0>::getValue(i);
    }
  }
};
\end{lstlisting}
\end{frame}
 
\begin{frame}[containsverbatim]{Fibonacci Sequence at compile time}
\lstset{language=c++}
\begin{lstlisting}
int main(int, char *[])
{
  //Generate (at compile time) 100 places of the Fib sequence.
  //Then, (at runtime) output the 100 calculated places.
  //Note: a 64 bit int overflows at place 92
  for (int i = 0; i < 100; ++i)
  {
    cout << "n:=" << i << " => " << Fib<100>::getValue(i) << endl;
  }
}

\end{lstlisting}
\end{frame}

\section{Lisp Macro}
\begin{frame}{Lisp Macro}
\begin{itemize}
\item comp.lang.lisp and any other comp.lang.* group with macro in the subject
\begin{itemize}
\item Lispnik: “Lisp is the best because of its macros!”
\item Othernik: “You think Lisp is good because of macros?! But macros are horrible and evil; Lisp must be horrible and evil.”
\end{itemize}
\item Usage
\begin{itemize}
\item function
\item lazy evaluation
\item syntax
\item Domain Specific Language (DSL)
\end{itemize}
\end{itemize}
\end{frame}

\begin{frame}[containsverbatim]{defmacro}
\lstset{language=lisp}
\begin{lstlisting}
(defmacro backwards (expr) (reverse expr))
(macroexpand `(backwards ("hello world!" t format)))
\end{lstlisting}
\end{frame}

\section{camlp4}

Caml Preprocessor and Pretty-Printer
one of its most important applications is the definition of domain-specific extensions of the syntax of OCaml
author: Daniel de Rauglaudre
\section{思考}
\begin{frame}{思考}
\begin{itemize}
\item 当前常见程序设计语言的新特性是什么?
\item 高阶编程的优缺点是什么?
\end{itemize}
\end{frame}

\end{document}
