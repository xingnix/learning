\begin{document}

\def\lecturename{嵌入式技术}

\title{\insertlecture}

\author{邢超}

\institute
{
  西北工业大学航天学院
}

%\mode<presentation>{\subject{嵌入式系统}}

%  start a lecture  --------------------------
\lecture[EC]{嵌入式技术}{}
\subtitle{GNU/Linux文件系统}
\date{2015}


%\setbeamertemplate{background}{\pgfimage[width=\paperwidth,height=\paperheight]{image/flower}}
%\setbeamercovered{transparent}
%\mode<presentation>{\beamerdefaultoverlayspecification{<+->}}

\begin{frame}
  \maketitle
\end{frame}

\section{文件系统}
\begin{frame}{Linux文件系统种类}
\begin{itemize}
\item  romfs
\item  ramdisk
\item  tmpfs
\item  JFFS/JFFS2 ( Journaling Flash File System )
\item  YAFFS ( Yet Another Flash File System )
\item  cramfs
\item  SquashFS
\item  NFS
\end{itemize}
\end{frame}

\subsection{VFS}
\begin{frame}{VFS}
\begin{center}\pgfimage[width=1.0\columnwidth]{image/vfs}\end{center}
\end{frame}

\subsection{EXT2文件系统}
\begin{frame}{EXT2文件系统}
\begin{itemize}
\item 引导块:在文件系统的开头,通常为一个扇区,存放引导程序,用于读入并启动操作系统
\item 超级块(superblock):用于记录文件系统的管理信息。特定的文件系统定义了特定超级块
\item inode区(索引节点):一个文件(或目录)占据一个索引节点。利用根节点(首个索引节点),可以把一个文件系统挂在另一个文件系统的非叶节点上
\item 数据区:用于存放文件数据或者管理数据(如一级间址块、二级间址块等)
\end{itemize}
\end{frame}

\begin{frame}{文件存储方式}
\begin{itemize}
\item EXT2中文件由逻辑块的序列组成。数据块的长度相同。不同的EXT2系统长度可以不同 。 文件总是整块存储,不足一块的部分也占用一个数据块
\item EXT2中的每个文件都用一个单独的inode(即stuct ext2\_inode结构)来描述,而每个inode都有一个唯一的标志号。 通过使用inode来定义文件系统的结构以及描述系统中每个文件的管理信息
\end{itemize}
\end{frame}


\subsection{romfs File System}
\begin{frame}{romfs文件系统}
\begin{itemize}
\item romfs(rom file system)是一种只读文件系统
\item 管理代码占用的空间比较小
\item 使用genromfs工具创建
\item 未完全实现文件访问权限
\item 如果要进行写操作:
\begin{itemize}
\item 在编译的时候加上写访问功能
\item 或者在运行时另外生成一个RAMdisk送暂存数据。
\end{itemize}
\end{itemize}
\end{frame}

\begin{frame}{romfs文件系统结构}
\begin{itemize}
\item 文件系统结构
\item 文件结构
\end{itemize}
\end{frame}

\begin{frame}[containsverbatim]{文件系统结构}
include/linux/binfmts.h:
\begin{lstlisting}
struct romfs_super_block {
       __u32 word0;     /*   -rom    */
       __u32 word1;     /*   1fs-   */
       __u32 size;      /* size of the file system */
       __u32 checksum;  /* checksum of first 512k */
       char name[0];    /* volume name */
};
\end{lstlisting}
\end{frame}

\begin{frame}[containsverbatim]{文件头结构}
include/linux/binfmts.h:
\begin{lstlisting}
struct romfs_inode {
       __u32 next; /* low 4 bits for file type*/
       __u32 spec; /* spec.info */
       __u32 size; /* file size */
       __u32 checksum; /* checksum of file */
       char name[0];  /* file name */
};

\end{lstlisting}
\end{frame}

\begin{frame}{spec.info}
\begin{itemize}
\item 0:硬链接,spec.info域的内容用于链接的目标文件
\item 1:目录,spec.info域的内容为第一个文件的文件头
\item 2:普通文件,spec.info域的内容无效,应设置为0
\item 3:符号链接,spec.info域的内容无效,应设置为0
\item 4:块设备,spec.info域内容为各16bit的主、从设备号 
\item 5:字符设备,spec.info域的内容无效,设置为0
\item 6:网络socket套接字spec.info域的内容无效,设置为0
\item 7:fifo管道文件,spec.info域的内容无效,设置为0
\end{itemize}
\end{frame}

\begin{frame}{romfs示例}
\begin{center}\pgfimage[width=1.0\columnwidth]{image/romfshex}\end{center}
\end{frame}

\begin{frame}{romfs示例}
\begin{center}\pgfimage[width=1.0\columnwidth]{image/romfsmap}\end{center}
\end{frame}


\section{文件系统使用}
\subsection{mount}
\begin{frame}{文件系统使用}
\begin{itemize}
\item etc/fstab
\item mount/umount
\end{itemize}
\end{frame}

\subsection{FHS}
\begin{frame}{FHS: Filesystem Hierarchy Standard}
\begin{center}\pgfimage[width=0.9\columnwidth]{image/verzeichnisse_baum}\end{center}
\end{frame}

 \section{思考}
 \begin{frame}{思考}
 \begin{itemize}
 \item GNU/Linux常用哪些文件系统?
 \item romfs的优缺点?
 \end{itemize}
 \end{frame}


\end{document}
