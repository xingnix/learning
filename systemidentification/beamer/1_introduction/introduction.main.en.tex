\begin{document}
%\begin{CJK*}{GBK}{song}

\def\lecturename{System Identification}

\title{\insertlecture}

\author{Xing Chao}

\institute
{
  School of Astronautics , Northwestern Polytechnical University
}

%\mode<presentation>{\subject{嵌入式系统}}

%  start a lecture  --------------------------
\lecture[SI]{Introduction to System Identification}{}
\subtitle{Basic concepts in System Identification}
\date{2012}


%\setbeamertemplate{background}{\pgfimage[width=\paperwidth,height=\paperheight]{image/flower}}
%\setbeamercovered{transparent}
%\mode<presentation>{\beamerdefaultoverlayspecification{<+->}}

\begin{frame}
  \maketitle
\end{frame}

%\begin{frame}{课程内容}
%\begin{center}\pgfimage[width=0.9\columnwidth]{image/content}\end{center}
%\end{frame}

%\begin{frame}{与其它课程关系}
%\begin{center}\pgfimage[width=0.9\columnwidth]{image/relation}\end{center}
%\end{frame}



\section{Main content}

\begin{frame}{Main content of the course}
\begin{itemize}
\item Purpose of System identification
\item Identification methods
\item Specific steps for identification
\end{itemize}
How to study:
\begin{itemize}
\item What does the course do?
\item What problem is mainly solved?
\item What are the methods?
\item What are the advantages and disadvantages of each method?
\item What is the scope of application of each method?
\end{itemize}
\end{frame}

\section{Basic concepts in System Identification}
\begin{frame}{Status and purpose of system identification}
\begin{description}
\item[Control thesry]classical control theory、modern control theory、intelligent control theory
\item[Classical control]applying the time domain method, the root locus method and the frequency domain method to design controller for a plant
\item[Modern control] linear system theory、optimal control theory and optimal estimation theory, etc.
\item[intelligent control]neural network, expert system and artificial intelligence
\end{description}
\end{frame}

\begin{frame}{Linear system theory}
The basis of modern control, mainly to solve the model description and basic knowledge of the system. That is, a linear system can generally be described as:
\begin{eqnarray}
\dot x = Ax+Bu \\
y=Cx+Du
\end{eqnarray}
\begin{description}
\item[Optimal Control] Solve how to obtain the optimal input u(t) under the constraint of a certain performance criteria;
\item[Optimal estimation] Mainly solve the estimation and prediction of the state variable X
\end{description}
\end{frame}


\begin{frame}{System identification purpose}
\begin{itemize}
\item Prerequisites for solving the above problem:
\begin{itemize}
\item A, B, C, and D in the model are known.
\item That is, the structure and parameters of the system are known.
\item That is to know the transfer function of the system, or the impulse transfer function, or the difference equation, or the frequency characteristics of the system。
\end{itemize}
\item So how do you get the structure and parameters of the system?
\item System identification purpose: How to get the model of the system and its parameters?
\end{itemize}
\end{frame}

\section{Model description of the system}
\begin{frame}{System model definition and characteristics}
\begin{description}
\item[Model definition]
    Part of the essence of the system is reduced to a useful form of description.
\item[Model characteristics]
\begin{itemize}
\item Multiple model descriptions can be used for the same system; 
\item The same model can reflect different actual systems;
\item Model accuracy and complexity。
\end{itemize}
\end{description}
\end{frame}

\begin{frame}{model representation}
\begin{itemize}
	\item intuition model
	\item physical model
	\item chart model
	\item mathematical model.
\end{itemize}
Among them, the chart model is a non-parametric model, and the mathematical model is a parametric model.
\end{frame}



\begin{frame}{mathematical model classification}
\begin{description}
	\item[time domain]
	\begin{itemize}
		\item differential equation
		\item difference equation
		\item equation of state
	\end{itemize}
	\item[Complex domain]
	\begin{itemize}
		\item transfer function
		\item impulse transfer function
	\end{itemize}
	\item[frequency domain]
	\begin{itemize}
		\item frequency characteristics
		\item description function
	\end{itemize}
\end{description}
\end{frame}


\begin{frame}{Model in system identification}
system identification acquires the non-parametric model and parametric model of a system.
\begin{description}
	\item[non-parametric model]
	\begin{itemize}
		\item frequency characteristic curve
		\item impulse response curve
	\end{itemize}
	\item[parametric model]
	\begin{itemize}
		\item differential/difference equation
		\item transfer function
		\item impulse transfer function
	\end{itemize}
	\item[model conversion]
	\begin{itemize}
	\item The parametric models can be transformed from each other;
	\item The non-parametric model can be transformed into a parametric model.
	\end{itemize}
\end{description}
\end{frame}

\section{Methods and principles for establishing mathematical models}
\begin{frame}{model creation method}
\begin{itemize}
	\item theoretical analysis method
	\item experimental test method: use the system input and output data to establish a mathematical model of the system.
\end{itemize}
\end{frame}

\begin{frame}{modeling principles}
\begin{itemize}
	\item The purpose of the model is definite;
	\item clear physical concept;
	\item identification is unbiased and consistent;
	\item conforms to the law of parsimony(Occam's razor). The number of parameters to be identified is small.
\end{itemize}
\end{frame}

\section{System identification and classification}


\begin{frame}{System identification definition}
\begin{itemize}
\item    Definition: Based on the system input and output data, determine a model equivalent to the system being tested from a given set of model classes.
\item    Three elements of system identification: data, model classes and criteria.
\begin{itemize}
\item    Data: recorded input/output data, often containing noise;
\item    Model class: selected models
\item    Criterion: This is the cost function, usually the error criterion.
\end{itemize}
\end{itemize}
\end{frame}

\begin{frame}{System Identification General Process}
 System identification is divided into model structure identification and model parameter identification.
The general process is:
\begin{itemize}
\item clarifies the purpose of the identified system model;
\item pre-select the type of mathematical model of the system to be identified;
\item design the experiment for identification, recording I/O data;
\item data preprocessing, wild point culling;
\item model structure identification, identification system order n;
\item select the parameter estimation method to identify other parameters of the system;
\item model validation.
\end{itemize}
The focus of this course: parameter estimation method
\end{frame}

\begin{frame}{System Identification Category}
\begin{itemize}
\item linear system identification and nonlinear system identification;
\item centralized parameter identification and distributed parameter identification;
\item system structure identification and system parameter identification;
\item classic identification and modern identification;
\item open loop system identification and closed loop system identification;
\item Offline identification and online identification.
\end{itemize}
\end{frame}

\begin{frame}{offline identification}
\begin{itemize}
\item Process: After the system model and order n are selected, record all the I/O data of the system, and then use the parameter estimation method to identify the model parameters of the system.
\item Features: The amount of data to be stored is large, the amount of calculation is large, and the recognition accuracy is high. Post-mortem data processing methods cannot be used in real-time control systems.
\end{itemize}
\end{frame}

\begin{frame}{online identification}
\begin{itemize}
\item process: After the system model and order n are selected, first obtain a small amount of data, estimate the system model parameters, and then obtain new I/O data, and use the recursive correction algorithm to obtain new parameter estimates, and repeat the above process. Until the system stops running.
\item Features: Small amount of data, small amount of calculation, and slightly lower recognition accuracy. It is an online data processing method for real-time control systems.
\end{itemize}
\end{frame}

\section{System Identification Error Criteria}
\begin{frame}{System Identification Error Criteria}
Error criteria are usually expressed as functionalities of errors
\begin{eqnarray}
J(\theta)=\Sigma_{k=1}^Nf(\epsilon(k))
\end{eqnarray}
$\epsilon(k)$ is the error between the model and the actual system. It can be output error or input error, or it can be generalized error. The general function $f$ is taken as the square of the error:
\begin{eqnarray}
f(\epsilon(k))=\epsilon^2(k)
\end{eqnarray}
\begin{itemize}
\item Input error $\epsilon(k)=u(k)-u_m(k)=u(k)-S^{-1}[y_m(k)]$
\item output error $\epsilon(k)=y(k)-y_m(k)$
\end{itemize}
This course uses output errors.
\end{frame}

\section{Thinking}
\begin{frame}{thinking}
\begin{itemize}
\item What is the relationship between system identification and other courses?
\item How to learn system identification?
\end{itemize}
\end{frame}

%\end{CJK*}
\end{document}
