\begin{document}
%\begin{CJK*}{GBK}{song}
\newcommand{\vect}[1]{\boldsymbol{#1}}

\def\lecturename{系统辨识}

\title{\insertlecture}

\author{邢超}

\institute
{
  西北工业大学航天学院
}

%\mode<presentation>{\subject{嵌入式系统}}

%  start a lecture  --------------------------
\lecture[]{系统结构辨识}{}
\subtitle{}
\date{2012}


%\setbeamertemplate{background}{\pgfimage[width=\paperwidth,height=\paperheight]{image/flower}}
%\setbeamercovered{transparent}
%\mode<presentation>{\beamerdefaultoverlayspecification{<+->}}

\begin{frame}
  \maketitle
\end{frame}


\section{简介}

\begin{frame}{模型阶的确定}
\begin{enumerate}
\item 按残差方差定阶
\item AIC准则
\item 按残差白色定阶
\item 零点—极点消去检验定阶
\item 利用行列式比定阶
\item 利用Hankel矩阵定阶
\end{enumerate}
\end{frame}

\section{按残差方差定阶}
\begin{frame}{按残差方差定阶}
计算不同阶次n辨识结果的估计误差方差,按估计误差方差最小或最显著变化原则来确定模型阶次n。

\begin{enumerate}
\item 按估计误差方差最小定阶
\item F检验法
\end{enumerate}
\end{frame}


\begin{frame}{按估计误差方差最小定阶}
\begin{eqnarray*}
a(z^{-1})y_k &=& b(z^{-1})u_k+\varepsilon_k \\
Y &=& \Phi \theta + \varepsilon \\
e &=& Y-\Phi\hat\theta \\
\hat\theta &=& (\Phi^T\Phi)^{-1}\Phi^T Y \\
J &=& e^T e =\sum_{k=n+1}^N e_k^2
\end{eqnarray*}
\end{frame}

\begin{frame}{F检验法}
\begin{eqnarray*}
t(n_i,n_{i+1}) &=& \frac{J_i-J_{i+1}}{J_{i+1}}\frac{N-2n_{i+1}}{2(n_{i+1}-n_i)}\\
&\sim& F(2n_{i+1}-2n_i,N-2n_{i+1})\\
t(n,n+1) &=& \frac{J_n-J_{n+1}}{J_{n+1}}\frac{N-2n-2}{2}\\
&=& \sim F(2,N-2n-2)
\end{eqnarray*}
\end{frame}

\section{AIC信息准则}
\begin{frame}{AIC信息准则(akaike)}
\begin{eqnarray*}
AIC &=& -2\ln L + 2p
\end{eqnarray*}
其中:
\begin{description}
\item[$L$]模型的似然函数;
\item[$p$]模型中的参数个数。
\end{description}
\end{frame}

\begin{frame}{计算公式:白噪声情况}
\begin{eqnarray*}
Y &=& \Phi\theta+\varepsilon  \\
L(Y|\theta) &=& (2\pi\sigma_\varepsilon^2)^{-N/2}exp\left[-\frac{(Y-\Phi\theta)^T(Y-\Phi\theta)}{2\sigma_\varepsilon^2}\right] \\
\ln L(Y|\theta) &=& -\frac{N}{2}\ln 2\pi -\frac{N}{2}\ln\sigma_\varepsilon^2-\frac{(Y-\Phi\theta)^T(Y-\Phi\theta)}{2\sigma_\varepsilon^2}
\end{eqnarray*}
由
\begin{eqnarray*}
\frac{\partial \ln L}{\partial\theta} &=& 0 \\
\frac{\partial \ln L}{\partial\sigma_\varepsilon^2} &=& 0 
\end{eqnarray*}
得:
\begin{eqnarray*}
\hat\theta &=& (\Phi^T\Phi)^{-1}\Phi^T Y \\
\hat\sigma_\varepsilon^2 &=& \frac{(Y-\Phi\hat\theta)^T(Y-\Phi\hat\theta)}{N} \\
\end{eqnarray*}
所以
\begin{eqnarray*}
\ln L &=& -\frac{N}{2}\ln\hat\sigma_\varepsilon^2+const \\
AIC &=& N\ln\hat\sigma_\varepsilon^2+2(n_1+n_2)
\end{eqnarray*}
\end{frame}

\begin{frame}{有色噪声情况}
\begin{eqnarray*}
Y&=&\Phi_a\theta_a+\Phi_b\theta_b+\Phi_c\theta_c+\varepsilon \\
AIC &=& N\ln\hat\sigma_\varepsilon^2+2(n_a+n_b+n_c) \\
\hat\sigma_\varepsilon^2 &=& \frac{1}{N}\sum_{k=n+1}^{n+N}\hat\varepsilon^2(k) \\
\hat\varepsilon(k) &=& y_k+\sum_{i=1}^{n_a}\hat a_i y_{k-i}-\sum_{i=0}^{n_b}\hat b_i u_{k-i}-\sum_{i=1}^{n_c}\hat c_i \hat\varepsilon_{k-i}
\end{eqnarray*}
\end{frame}


\section{按残差白色定阶}
\begin{frame}{按残差白色定阶}
若阶次n设计合适,则残差近似为白噪声。因此可利用计算残差e(k)的自相关函数来检查白色性。 
\begin{eqnarray*}
\hat R(i) &=& \frac{1}{N}\sum_{k=n+1}^{n+N}e(k)e(k+1) \\
\hat r(i) &=& \frac{\hat R(i) }{\hat R(0)}
\end{eqnarray*}
\end{frame}

\section{零点—极点消去检验}
\begin{frame}{零点—极点消去检验}
\begin{eqnarray*}
a(z^{-1})y(k) &=& b(z^{-1})u(k)+\varepsilon(k) \\
G(z) &=& \frac{b(z^{-1})}{a(z^{-1})}
\end{eqnarray*}
根据$G(z)$中是否存在零极点对消确定系统阶数。
\end{frame}

\section{行列式比定阶}
\begin{frame}{行列式比定阶}
\begin{eqnarray*}
Y &=& \Phi\theta \\
Q &=& \frac{\Phi^T\Phi}{N} \\
DR(n) &=& \frac{\det Q(n)}{\det Q(n+1)}
\end{eqnarray*}
\end{frame}

\section{Hankel矩阵定阶}
\begin{frame}{Hankel矩阵定阶}
\begin{eqnarray*}
D &=& \frac{\sum \det H(n,k)}{\sum \det H(n+1),k)} \\
H(n,k) &=& \begin{bmatrix}
g_k & g_{k+1} & \cdots & g_{k+n-1} \\
g_{k+1} & g_{k+2} & \cdots & g_{k+n} \\
\vdots &  \vdots  &       & \vdots \\
g_{k+n-1} & g_{k+n} & \cdots & g_{k+2n-2} 
\end{bmatrix}
\end{eqnarray*}
还可以将$g_k$替换为$\rho_k$:
\begin{eqnarray*}
\rho_k &=& \frac{\hat R(k)}{\hat R(0)} \\
\hat R(k) &=& \frac{1}{N}\sum_{i=n+1}^N g_i g_{k-i}
\end{eqnarray*}
\end{frame}

% \section{思考}
% \begin{frame}{思考}
% \begin{itemize}
% \item 极大似然法辨识思想
% \end{itemize}
% \end{frame}


%\end{CJK*}
\end{document}
