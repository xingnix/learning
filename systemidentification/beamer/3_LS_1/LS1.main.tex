\DeclareMathOperator*{\argmin}{arg\,min}
\begin{document}
%\begin{CJK*}{GBK}{song}
\newcommand{\vect}[1]{\boldsymbol{#1}}

\def\lecturename{系统辨识}

\title{\insertlecture}

\author{邢超}

\institute
{
  西北工业大学航天学院
}

%\mode<presentation>{\subject{嵌入式系统}}

%  start a lecture  --------------------------
\lecture[LS]{最小二乘法辨识}{}
\subtitle{白噪声情况}
\date{2012}


%\setbeamertemplate{background}{\pgfimage[width=\paperwidth,height=\paperheight]{image/flower}}
%\setbeamercovered{transparent}
%\mode<presentation>{\beamerdefaultoverlayspecification{<+->}}

\begin{frame}
  \maketitle
\end{frame}

\section{最小二乘估计}

\begin{frame}{基于输入/输出数据的系统模型描述}
SISO系统的差分方程为
$$
y_k+a_1 y_{k-1}+\cdots + a_n y_{k-n}=b_0 u_k+\cdots + b_n u_{k-n}+\xi_k
$$
在时刻$k=n+1,n+2,\cdots,n+N$,有
\begin{gather*}
y_{n+1}+a_1 y_n+\cdots + a_n y_1 = b_0 u_{n+1}+\cdots + b_n u_1+\xi_{n+1}  \\
y_{n+2}+a_1 y_{n+1}+\cdots + a_n y_2 = b_0 u_{n+2}+\cdots + b_n u_{2}+\xi_{n+2}  \\
\cdots    \\
y_{n+N}+a_1 y_{n+N-1}+\cdots + a_n y_{N} = b_0 u_{n+N}+\cdots + b_n u_{N}+\xi_{n+N}
\end{gather*}
\end{frame}

\begin{frame}{向量形式}
\begin{eqnarray*}
Y &=& \Phi \theta + \xi \\
Y &=& \begin{bmatrix}  y_{n+1} & y_{n+2} & \cdots & y_{n+N}  \end{bmatrix}^T  \\
\Phi &=& \begin{bmatrix}
-y_{n}    & \cdots & -y_{1}   & u_{n+1} & \cdots & u_{1}  \\
-y_{n+1}  & \cdots & -y_{2}   & u_{n+2} & \cdots & u_{2}  \\
\vdots   &        & \vdots  & \vdots &        & \vdots \\
-y_{n+N-1} & \cdots & -y_{N}   & u_{n+N} & \cdots & u_{N}
\end{bmatrix} \\
\theta &=& \begin{bmatrix}a_1 & \cdots & a_n & b_0 & \cdots & b_n \end{bmatrix}^T\\
\xi &=& \begin{bmatrix}  \xi_{n+1} &  \xi_{n+2} & \cdots & \xi_{n+N} \end{bmatrix}^T
\end{eqnarray*}
\end{frame}

\begin{frame}{基本的最小二乘法:辨识准则}
辨识准则:残差平方和最小。

%\newcommand{\argmin}{\operatornamewithlimits{argmin}}
\begin{eqnarray*}
J &=& \sum_{k=n+1}^{n+N}e^2(k)  \\
&=& (Y-\Phi\hat\theta)^T(Y-\Phi\hat\theta) \\
\hat\theta_{LS} &=& \argmin_{\hat\theta} J
\end{eqnarray*}
\end{frame}

\begin{frame}{基本的最小二乘法:求导}
\begin{eqnarray*}
\frac{\partial J}{\partial \hat\theta_k} &=& \frac{\partial \sum_i(Y_i-\sum_m\Phi_{i,m}\hat\theta_m)^2}{\partial\hat\theta_k}  \\
&=& 2\sum_i(Y_i-\sum_m\Phi_{i,m}\hat\theta_m)\frac{\partial (Y_i-\sum_m\Phi_{i,m}\hat\theta_m)}{\partial\hat\theta_k} \\
&=& 2\sum_i(Y_i-\sum_m\Phi_{i,m}\hat\theta_m)\frac{\partial(-\sum_m\Phi_{i,m}\hat\theta_m)}{\partial\hat\theta_k} \\
&=& -2\sum_i(Y_i-\sum_m\Phi_{i,m}\hat\theta_m)\Phi_{i,k} \\
\frac{\partial J}{\partial \hat\theta}&=& (-2(Y-\Phi\hat\theta)^T\Phi)^T \\
&=& -2\Phi^T(Y-\Phi\hat\theta) 
\end{eqnarray*}
\end{frame}

\begin{frame}{基本的最小二乘法:求解}
\begin{eqnarray*}
-2\Phi^T(Y-\Phi\hat\theta_{LS}) &=&0\\
\Phi^T Y-\Phi^T\Phi\hat\theta_{LS} &=&0\\
\Phi^T Y &=& \Phi^T\Phi\hat\theta_{LS} \\
\hat\theta_{LS}&=& (\Phi^T\Phi)^{-1}\Phi^T Y 
\end{eqnarray*}
\end{frame}

\begin{frame}{基本的最小二乘法:二阶导数}
\begin{eqnarray*}
\frac{\partial^2 J}{\partial\hat\theta^2} &=& \frac{\partial (-2\Phi^T(Y-\Phi\hat\theta))}{\partial\hat\theta} \\
\frac{\partial \frac{\partial J}{\partial\hat\theta}}{\partial \hat\theta_s}&=& \frac{\partial (-2\sum_i(Y_i-\sum_m\Phi_{i,m}\hat\theta_m)\Phi_{i,k})}{\partial \hat\theta_s} \\
&=& 2\sum_i\frac{\partial \sum_m\Phi_{i,m}\hat\theta_m}{\partial \hat\theta_s}\Phi_{i,k} \\
&=& 2\sum_i\Phi_{i,s}\Phi_{i,k} \\
\frac{\partial^2 J}{\partial\hat\theta^2} &=& 2\Phi^T\Phi
\end{eqnarray*}
\end{frame}


\begin{frame}{最小二乘法对输入信号的要求:$\begin{bmatrix} Y_{N \times n} &  U_{N\times(n+1)}\end{bmatrix}$}
\begin{eqnarray*}
\Phi^T\Phi &=&\begin{bmatrix} Y_{N \times n} &  U_{N\times(n+1)}\end{bmatrix}^T\begin{bmatrix} Y_{N\times n} &  U_{N\times(n+1)}\end{bmatrix} \\
&=& \begin{bmatrix}
Y_{N \times n}^T Y_{N\times n}  & Y_{N \times n}^T U_{N\times(n+1)}  \\
U_{N\times(n+1)}^T Y_{N\times n}  &  U_{N\times(n+1)}^T U_{N\times(n+1)}
\end{bmatrix}
\end{eqnarray*}
其中:
\begin{eqnarray*}
Y_{N\times n} &=& 
\begin{bmatrix}
-y_{n}    & \cdots & -y_{1}    \\
-y_{n+1}  & \cdots & -y_{2}    \\
\vdots    &        & \vdots   \\
y_{n+N-1} & \cdots & -y_{N}  
\end{bmatrix} \\
U_{N\times(n+1)} &=&
\begin{bmatrix}
u_{n+1} & \cdots & u_{1} \\
u_{n+2} & \cdots & u_{2} \\
\vdots &         & \vdots \\
u_{n+N} & \cdots & u_{N} 
\end{bmatrix} 
\end{eqnarray*}
\end{frame}

\begin{frame}{最小二乘法对输入信号的要求:$\begin{bmatrix} Y_{N \times n} &  U_{N\times(n+1)}\end{bmatrix}$}
\begin{eqnarray*}
(Y_{N\times n}^T Y_{N\times n})_{i,j} &=& \sum_{k=1}^{N-1+\min\{i,j\}}y_{n-i+k}y_{n-j+k} \\
(Y_{N\times n}^T U_{N\times (n+1)})_{i,j} &=& -\sum_{k=1}^{N-1+\min\{i,j-1\}}y_{n-i+k}u_{n+1-j+k} \\
(U_{N\times (n+1)}^T Y_{N\times n}^T )_{i,j} &=& -\sum_{k=1}^{N-1+\min\{j,i-1\}}y_{n-j+k}u_{n+1-i+k} \\
(U_{N\times (n+1)}^T U_{N\times (n+1)})_{i,j} &=& \sum_{k=1}^{N-2+\min\{i,j\}}u_{n+1-i+k}u_{n+1-j+k}
\end{eqnarray*}
\end{frame}

\begin{frame}{最小二乘法对输入信号的要求:$\begin{bmatrix}R_y & R_{yu} \\ R_{uy} & R_{u}\end{bmatrix}$}
\begin{eqnarray*}
\lim_{N\rightarrow\infty}\frac{\Phi^T\Phi}{N} &=& 
\frac{1}{N}
\begin{bmatrix}
Y_{N \times n}^T Y_{N\times n}  & Y_{N \times n}^T U_{N\times(n+1)}  \\
U_{N\times(n+1)}^T Y_{N\times n}  &  U_{N\times(n+1)}^T U_{N\times(n+1)} 
\end{bmatrix}\\
&=& \begin{bmatrix} R_y & R_{yu} \\ R_{uy} & R_u\end{bmatrix}
\end{eqnarray*}
其中:
\begin{eqnarray*}
R_y &=& 
\begin{bmatrix}
R_y(0) & R_y(1) & \cdots & R_y(n-1) \\
R_y(1) & R_y(0) & \cdots & R_y(n-2) \\
\vdots & \vdots &        & \vdots   \\
R_y(n-1) & R_y(n-2) & \cdots & R_y(0) \\
\end{bmatrix}\\
R_{yu} &=&
\begin{bmatrix}
-R_{yu}(1) & -R_{yu}(0) & \cdots & -R_{yu}(1-n) \\
-R_{yu}(2) & -R_{yu}(1) & \cdots & -R_{yu}(2-n) \\
\vdots     & \vdots     &        &\vdots \\
-R_{yu}(n) & -R_{yu}(n-1) & \cdots & -R_{yu}(0) 
\end{bmatrix} \\
\end{eqnarray*}
\end{frame}

\begin{frame}{最小二乘法对输入信号的要求:$\begin{bmatrix}R_y & R_{yu} \\ R_{uy} & R_{u}\end{bmatrix}$}
\begin{eqnarray*}
R_{uy} &=& R_{yu}^T \\
R_{uu} &=&
\begin{bmatrix}
R_u(0) & R_u(1) & \cdots & R_u(n) \\
R_u(1) & R_u(0) & \cdots & R_u(n-1) \\
\vdots     & \vdots     &        &\vdots \\
R_u(n) & R_u(n-1) & \cdots & R_u(0)
\end{bmatrix}
\end{eqnarray*}

%于是有:J取得极小值→φTφ正定→R正定→Ru正定。

%因此:J取得极小值的必要条件为Ru为正定阵。
\end{frame}

\begin{frame}{(n+1)阶持续激励信号}
\begin{itemize}
\item 定义:如果序列$\{u(k)\}$的(n+1)阶方阵$R_u$是正定的,则称序列
$\{u(k)\}$为$(n+1)$阶持续激励信号。
\item 最小二乘法对输入信号的要求为:$\{u(k)\}$为$(n+1)$阶持续激励信号
\item 若$R_u$为强对角线占优矩阵,则$R_u$正定。以下输入信号均能满足$R_u$正定的要求:
\begin{itemize}
\item 白噪声序列;
\item 伪随机二位式噪声序列;
\item 有色噪声随机信号序列。
\end{itemize}
\item 工程上常用“伪随机二位式噪声序列”、“有色噪声随机信号序列”作为输入信号。
\end{itemize}
\end{frame}

\mode<beamer>{
\begin{frame}{最小二乘估计的概率性质}
%最小二乘估计的概率性质主要有以下四方面:
\begin{itemize}
\item 估计的无偏性;
\item 估计的一致性;
\item 估计的有效性;
\item 估计的渐进正态性。
\end{itemize}
%我们主要讨论前两项:无偏性和一致性。
\end{frame}
}

\begin{frame}{估计的无偏性}
若$E\{\hat\theta\}=\theta$则称$\hat\theta$是参数$\theta$的无偏估计。
\begin{eqnarray*}
Y &=& \Phi\theta +\xi \\
\hat\theta &=& (\Phi^T\Phi)^{-1}\Phi^T Y \\
E[\hat\theta] &=& E[(\Phi^T\Phi)^{-1}\Phi^T Y] \\
&=& E[(\Phi^T\Phi)^{-1}\Phi^T (\Phi\theta+\xi)] \\
&=& E[(\Phi^T\Phi)^{-1}\Phi^T \Phi\theta+(\Phi^T\Phi)^{-1}\Phi^T \xi] \\
&=& E[\theta+(\Phi^T\Phi)^{-1}\Phi^T \xi] 
\end{eqnarray*}
LS无偏估计的充要条件为:
\begin{eqnarray*}
 E[(\Phi^T\Phi)^{-1}\Phi^T \xi] &=& 0
\end{eqnarray*}
%LS无偏估计的充分条件为:$\{\xi(k)\}$为零均值不相关随机序列,且与$\{u(k)\}$无关。
\end{frame}

\begin{frame}{一致性估计}
\article{若参数估计值以概率1收敛于真值$\theta$,则称估计值具有一致性。定义:}
$$
\lim_{N\rightarrow\infty}P\{|\hat\theta-\theta\}=1
$$
设$\xi\{(k)\}$为与$\{u(k)\}$无关的零均值独立同分布随机序列:
\begin{eqnarray*}
 E(\hat\theta-\theta)^2 &=& E[(\Phi^T\Phi)^{-1}\Phi^T\xi\xi^T\Phi(\Phi^T\Phi)^{-1}] \\
 &=& E[\frac{1}{N^2}(\frac{1}{N}\Phi^T\Phi)^{-1}\Phi^T\xi\xi^T \Phi(\frac{1}{N}\Phi^T\Phi)^{-1}] \\
   \lim_{N\rightarrow\infty}E(\hat\theta-\theta)^2  &=& \frac{1}{N^2}R^{-1}E[\Phi^T\xi\xi^T\Phi] R^{-1} \\
  &=& \frac{1}{N^2}R^{-1}\sigma^2 E[\Phi^T\Phi] R^{-1} \\
  &=& \frac{1}{N^2}R^{-1}\sigma^2 N R R^{-1} \\
  &=& \frac{\sigma^2}{N}R^{-1} \\
  &=& 0
\end{eqnarray*}
\end{frame}

\mode<beamer>{
\begin{frame}{估计值的有效性}
\article{若参数估计误差的方差达到最小值,则称该估计值是有效估计值。}
%若$\{\xi(k)\}$是零均值且服从正态分布的白噪声序列。则最小二乘参数估计值$\hat\theta$为有效估计值,参数估计偏差的方差达到Cramer-Rao不等式的下界
Cramer-Rao不等式:
$$
D\hat\theta = M^{-1}
$$
其中:
$$
M=E\left[\left(\frac{\partial\ln p(y|\theta)}{\partial\theta}\right)^T\left(\frac{\partial\ln p(y|\theta)}{\partial\theta}\right)\right]
$$
\end{frame}
}
\mode<beamer>{
\begin{frame}{估计值的有效性}
\begin{eqnarray*}
y &=& \Phi\theta+\xi \\
y &\sim& N(\Phi\theta,\sigma^2 I) \\
p(y|\theta) &=& (2\pi\sigma^2)^{-\frac{N}{2}} exp\left[-\frac{1}{2\sigma^2}(y-\Phi\theta)^T(y-\Phi\theta)\right] \\
\frac{\partial \ln p(y|\theta)}{\partial \theta} &=& -\frac{1}{\sigma^2}(y-\Phi\theta)^T\Phi \\
M &=& E\left[\frac{1}{\sigma^4}\Phi^T(y-\Phi\theta)(y-\Phi\theta)^T\Phi \right]\\
 &=& \frac{1}{\sigma^4}E[\Phi^T \xi \xi^T \Phi ]\\
\lim_{N\to\infty}M^{-1} &=& \sigma^4(\sigma^2 E[\Phi^T \Phi ])^{-1}\\
&=& \frac{\sigma^2}{N}R^{-1} 
\end{eqnarray*}
\end{frame}
}
\mode<beamer>{
\begin{frame}{估计值的渐近正态性}
设$\{\xi(k)\}$是零均值且服从正态分布的白噪声序列。则:
\begin{eqnarray*}
y &=& \Phi\theta+\xi \\
y &\sim& N(\Phi\theta,\sigma^2 I) \\
\hat\theta &=& (\Phi^T\Phi)^{-1}\Phi^T y 
\end{eqnarray*}
\end{frame}
}

\section{模型阶次递增算法}
\begin{frame}{模型阶次递增算法:算法特点}
\begin{itemize}
\item 按模型阶次$n$递推的算法;
\item 适合模型阶次$n$未知的情况下应用
\item 辨识精度与基本最小二乘相同
\item 辨识速度比基本最小二乘有较大提高
\item 不需计算高阶矩阵的逆
\end{itemize}
\end{frame}

\begin{frame}{系统模型}
\begin{eqnarray*}
Y &=& \Phi_n\theta_n +\xi \\
\Phi_n &=& \begin{bmatrix}
u_{n+1}  &   -y_{n}    & u_{n}      & \cdots & -y_{1}  & u_{1}  \\
u_{n+2}  &   -y_{n+1}  & u_{n+1}    & \cdots & -y_{2}  & u_{2}  \\
\vdots   &   \vdots    & \vdots     &        & \vdots  & \vdots \\
u_{n+N}  &   -y_{n+N-1} & u_{n+N-1} & \cdots & -y_{N}  & u_{N}
\end{bmatrix} \\
&=& \begin{bmatrix}X_1 & \cdots & X_{2n+1} \end{bmatrix} \\
\theta_n &=& \begin{bmatrix}b_0 & a_1 & b_1 & \cdots & a_n & b_n \end{bmatrix}^T \\
\xi &=& \begin{bmatrix}\xi_{n+1} &\cdots & \xi_{n+N}\end{bmatrix}^T  \\
Y &=& \begin{bmatrix}y_{n+1} &\cdots & y_{n+N}\end{bmatrix}^T 
\end{eqnarray*}
\end{frame}

\begin{frame}{从$n=0$开始辨识}
\begin{eqnarray*}
\Phi_0 &=& X_1 \\
\hat\theta_0 &=& (\Phi_0^T\Phi_0)^{-1}\Phi_0^T Y \\
&=& \frac{\displaystyle \sum_{i=n+1}^{n+N}u_iy_i}{\displaystyle \sum_{i=n+1}^{n+N}u_i^2}
\end{eqnarray*}
\end{frame}

\begin{frame}{从$n$到$n+1$}
根据模型阶次为$n$时的参数辨识结果,求模型阶次为n+1时的辨识结果。
求解时分两步进行,首先求解$\tilde P_n$,其次求解$ P_{n+1}$。
\begin{eqnarray*}
\Phi_{n+1} &=& \begin{bmatrix} \Phi_{n} & X_{2n+2} & X_{2n+3} \end{bmatrix} \\
 &=& \begin{bmatrix} \tilde\Phi_{n} & X_{2n+3} \end{bmatrix} \\
\tilde\Phi_n &\triangleq& \begin{bmatrix} \Phi_{n} & X_{2n+2} \end{bmatrix}  \\
P_n &\triangleq& (\Phi_{n}^T\Phi_{n})^{-1} \\
\tilde{P}_n &\triangleq& (\tilde\Phi_{n}^T\tilde\Phi_{n})^{-1}  \\
P_{n+1} &=& (\Phi_{n+1}^T\Phi_{n+1})^{-1} 
\end{eqnarray*}
\end{frame}

\begin{frame}{从$n$到$n+1$:$P_{n+1}$}
\begin{eqnarray*}
P_{n+1} &=& 
\begin{bmatrix}
\tilde\Phi_n^T\tilde\Phi_n & \tilde\Phi_n^T X_{2n+3} \\
X_{2n+3}^T\tilde\Phi_n    & X_{2n+3}^T X_{2n+3}
\end{bmatrix}^{-1} \\
&=& \begin{bmatrix} A_{11} & A_{12} \\ A_{21} & A_{22} \end{bmatrix} \\
A_{22} &=& (X_{2n+3}^T X_{2n+3}-X_{2n+3}^T\tilde\Phi_n\tilde{P}_n\tilde\Phi_n^T X_{2n+3})^{-1} \\
A_{12} &=& A_{21}^T \\
&=& -\tilde{P}_n\tilde\Phi_n^T X_{2n+3} A_{22} \\
A_{11} &=& \tilde{P}_n- A_{12} X_{2n+3}^T\tilde\Phi_n \tilde{P}_n^T
\end{eqnarray*}
\end{frame}

\begin{frame}{从$n$到$n+1$:$\tilde P_n$}
\begin{eqnarray*}
\tilde{P}_n &=& 
\begin{bmatrix}
\Phi_n^T\Phi_n & \Phi_n^T X_{2n+2} \\
X_{2n+2}^T\Phi_n    & X_{2n+2}^T X_{2n+3}
\end{bmatrix}^{-1} \\
&=& \begin{bmatrix} B_{11} & B_{12} \\ B_{21} & B_{22} \end{bmatrix} \\
B_{22} &=& (X_{2n+2}^T X_{2n+2}-X_{2n+2}^T\Phi_n P_n\Phi_n^T X_{2n+2})^{-1} \\
B_{12} &=& B_{21}^T \\
&=& -P_n \Phi_n^T X_{2n+2} B_{22} \\
B_{11} &=& P_n- B_{12} X_{2n+2}^T\Phi_n P_n^T
\end{eqnarray*}
\end{frame}


\begin{frame}{计算步聚}
\begin{itemize}
\item  初始化,计算$P_0=(\Phi_0^T\Phi_0)^{-1}$
\item  计算$\hat\theta_0=P_0\Phi_0^T Y$
\item 迭代
\begin{itemize}
\item  根据$P_n$计算$\tilde{P}_n$
\item  根据$\tilde{P}_n$计算$P_{n+1}$
\item  计算$\hat\theta_{n+1}=P_{n+1}\Phi_{n+1}^T Y$
\end{itemize}
\end{itemize}
\end{frame}

\section{递推最小二乘法}
%\article{解决问题:(n+N)组观测数据时的参数估计值已知,现在又得到了一组新的观测值(u(n+N+1),y(n+N+1)),如何采用最小二乘法进行在线估计新的估计值问题}

\begin{frame}{递推算法推导:模型}
假设已获取了数据长度为$N$的输入输出数据,则由最小二乘估计有:
\begin{eqnarray*}
Y_N &=& \Phi_N\theta+\xi_N \\
\hat\theta_N &=& (\Phi_N^T\Phi_N)^{-1}\Phi_N^T Y_N \\
\tilde\theta_N &=& \theta-\tilde\theta_N \\
&=& -(\Phi_N^T\Phi_N)^{-1}\Phi_N^T\xi_N
\end{eqnarray*}
获得新的数据$u_{n+N+1},y_{n+N+1}$后,有:
\begin{eqnarray*}
y_{(n+N+1)} &=& \Psi^T\theta+\xi_{(n+N+1)} \\
y_{N+1} &=& \Psi^T\theta+\xi_{N+1} \\
\Psi_i &=& \begin{bmatrix} -y_{(n+i-1)} & \cdots & -y_{(i)} & u_{(n+i)} & \cdots & u_{(i)} \end{bmatrix}^T  \\
\begin{bmatrix} Y_N \\ y_{N+1} \end{bmatrix} &=& 
\begin{bmatrix} \Phi_N \\ \Psi_{N+1}^T \end{bmatrix} \theta +
\begin{bmatrix} \xi_N \\ \xi_{N+1} \end{bmatrix} 
\end{eqnarray*}
\end{frame}

\bgroup
\setbeamertemplate{sidebar right}{}
\begin{frame}{递推算法推导:$P_{N+1}$}
\begin{eqnarray*}
\hat\theta_{N+1} &=&
\left(
\begin{bmatrix} \Phi_N \\ \Psi_{N+1}^T\end{bmatrix}^T
\begin{bmatrix} \Phi_N \\ \Psi_{N+1}^T\end{bmatrix}
\right)^{-1}
\begin{bmatrix} \Phi_N \\ \Psi_{N+1}^T\end{bmatrix}^T
\begin{bmatrix} Y_N \\ y_{N+1}\end{bmatrix} \\
&=& (\Phi_N^T\underbrace{\Phi_N}_{N,2n+1}+\underbrace{\Psi_{N+1}}_{2n+1,1}\Psi_{N+1}^T)^{-1}(\Phi_N^T \underbrace{Y_N}_{N,1} +\Psi_{N+1}\underbrace{y_{N+1}}_{1,1})  \\
\hat\theta_{N+1} &=& P_{N+1}(\Phi_N^T Y_N + \Psi_{N+1} y_{N+1}) 
\end{eqnarray*}
其中:
\begin{eqnarray*}
P_{N+1} &=&(P_N^{-1}+\Psi_{N+1}\Psi_{N+1}^T)^{-1} \\
P_{N} &=& (\Phi_N^T\Phi_N)^{-1}
\end{eqnarray*}
\end{frame}
\egroup

\begin{frame}{递推算法推导:矩阵求逆引理}
若相应矩阵的逆均存在,则有:
$$
(A+BC^T)^{-1}=A^{-1}-A^{-1}B(I+C^T A^{-1} B)^{-1} C^T A^{-1}
$$
所以:
\begin{eqnarray*}
P_{N+1} &=&(P_N^{-1}+\Psi_{N+1}\Psi_{N+1}^T)^{-1} \\
&=& P_N -P_N\Psi_{N+1}(1+\Psi_{N+1}^T P_N \Psi_{N+1})^{-1}\Psi_{N+1}^T P_N \\
\hat\theta_{N+1} &=& A+B \\
A&=&P_{N+1}\Phi_N^T Y_N \\
B&=&P_{N+1}\Psi_{N+1} y_{N+1} \\
i&=& 1+\Psi_{N+1}^T P_N \Psi_{N+1}
\end{eqnarray*}
\end{frame}

\bgroup
\setbeamertemplate{sidebar right}{}
\begin{frame}{递推算法推导:化简}
\begin{eqnarray*}
A &=& (P_N -P_N\Psi_{N+1}i^{-1}\Psi_{N+1}^T P_N )\Phi_N^T Y_N\\
&=& P_N\Phi_N^T Y_N -P_N\Psi_{N+1}i^{-1}\Psi_{N+1}^T P_N \Phi_N^T Y_N\\
&=& \hat\theta_N-P_N\Psi_{N+1}i^{-1}\Psi_{N+1}^T \hat\theta_N\\
B &=& (P_N -P_N\Psi_{N+1}i^{-1}\Psi_{N+1}^T P_N )\Psi_{N+1}y_{N+1}\\
&=& i^{-1}(P_N(1+\Psi_{N+1}^T P_N \Psi_{N+1}) -P_N\Psi_{N+1}\Psi_{N+1}^T P_N )\Psi_{N+1}y_{N+1}\\
&=& i^{-1}(P_N+P_N\Psi_{N+1}^T P_N \Psi_{N+1} -P_N\Psi_{N+1}\Psi_{N+1}^T P_N )\Psi_{N+1}y_{N+1}\\
&=& i^{-1}(P_N\Psi_{N+1}+P_N\Psi_{N+1}^T P_N \Psi_{N+1}\Psi_{N+1} \\
&&  -P_N\Psi_{N+1}\Psi_{N+1}^T P_N \Psi_{N+1})y_{N+1}\\
&=& i^{-1}(P_N\Psi_{N+1}+P_N\Psi_{N+1}\Psi_{N+1}^T P_N \Psi_{N+1} \\
&& -P_N\Psi_{N+1}\Psi_{N+1}^T P_N \Psi_{N+1})y_{N+1}\\
&=& i^{-1}P_N\Psi_{N+1}y_{N+1}
% A &=& (P_N -P_N\Psi_{N+1}(1+\Psi_{N+1}^T P_N \Psi_{N+1})^{-1}\Psi_{N+1}^T P_N )\Phi_N^T Y_N\\
% &=& P_N\Phi_N^T Y_N -P_N\Psi_{N+1}(1+\Psi_{N+1}^T P_N \Psi_{N+1})^{-1}\Psi_{N+1}^T P_N \Phi_N^T Y_N\\
% &=& \hat\theta_N-P_N\Psi_{N+1}(1+\Psi_{N+1}^T P_N \Psi_{N+1})^{-1}\Psi_{N+1}^T \hat\theta_N\\
% B &=& (P_N -P_N\Psi_{N+1}(1+\Psi_{N+1}^T P_N \Psi_{N+1})^{-1}\Psi_{N+1}^T P_N )\Psi_{N+1}y_{N+1}\\
%  &=& (1+\Psi_{N+1}^T P_N \Psi_{N+1})^{-1}(P_N(1+\Psi_{N+1}^T P_N \Psi_{N+1}) -P_N\Psi_{N+1}\Psi_{N+1}^T P_N )\Psi_{N+1}y_{N+1}\\
%  &=& (1+\Psi_{N+1}^T P_N \Psi_{N+1})^{-1}(P_N+P_N\Psi_{N+1}^T P_N \Psi_{N+1} -P_N\Psi_{N+1}\Psi_{N+1}^T P_N )\Psi_{N+1}y_{N+1}\\
%  &=& (1+\Psi_{N+1}^T P_N \Psi_{N+1})^{-1}(P_N\Psi_{N+1}y_{N+1}+P_N\Psi_{N+1}^T P_N \Psi_{N+1}\Psi_{N+1}y_{N+1} -P_N\Psi_{N+1}\Psi_{N+1}^T P_N \Psi_{N+1}y_{N+1})\\
%  &=& (1+\Psi_{N+1}^T P_N \Psi_{N+1})^{-1}(P_N\Psi_{N+1}+P_N\Psi_{N+1}\Psi_{N+1}^T P_N \Psi_{N+1} -P_N\Psi_{N+1}\Psi_{N+1}^T P_N \Psi_{N+1})y_{N+1}\\
%  &=& (1+\Psi_{N+1}^T P_N \Psi_{N+1})^{-1}P_N\Psi_{N+1}y_{N+1}\\
\end{eqnarray*}
\article{提示:$\Psi_{N+1}^T P_N \Psi_{N+1}$为标量}
\end{frame}
\egroup

\begin{frame}{递推算法推导:结果}
\begin{eqnarray*}
\hat\theta_{N+1} &=& \hat\theta_N-P_N\Psi_{N+1}i^{-1}\Psi_{N+1}^T \hat\theta_N +i^{-1}P_N\Psi_{N+1}y_{N+1} \\
&=& \hat\theta_N+i^{-1}P_N\Psi_{N+1}(-\Psi_{N+1}^T \hat\theta_N +y_{N+1}) \\
&=& \hat\theta_N+ K_{N+1}(y_{N+1}-\Psi_{N+1}^T\hat\theta_N) \\
K_{N+1} &=& P_N\Psi_{N+1}(1+\Psi_{N+1}^T P_N \Psi_{N+1})^{-1} \\
P_{N+1} &=& P_N -K_{N+1}\Psi_{N+1}^T P_N 
\end{eqnarray*}

初值获取方法:
\begin{itemize}
\item 基本最小二乘估计
\item $\hat\theta_0=0,P_0=c^2I$,其中$c$为充分大的常数。
\end{itemize}
\end{frame}

\begin{frame}{收敛性:$P_N$}
\begin{eqnarray*}
P_N &=& (P_{N-1}^{-1}+\Psi_N\Psi_N^T)^{-1} \\
P_N^{-1} &=& P_{N-1}^{-1}+\Psi_N\Psi_N^T  \\
P_{N-1}^{-1} &=& P_{N-2}^{-1}+\Psi_{N-1}\Psi_{N-1}^T  \\
P_{N-2}^{-1} &=& P_{N-3}^{-1}+\Psi_{N-2}\Psi_{N-2}^T  \\
P_{N-3}^{-1} &=& P_{N-4}^{-1}+\Psi_{N-3}\Psi_{N-3}^T  \\
& \vdots & \\
P_1^{-1} &=& P_0^{-1}+\Psi_1\Psi_1^T  \\
P_N^{-1} &=& P_0^{-1}+\sum_{i=1}^{N}\Psi_i\Psi_i^T \\
\end{eqnarray*}
\end{frame}

\begin{frame}{收敛性}
\article{$\Psi_{i}$对应的是$\Phi_{N}$的第$i$行}
\begin{eqnarray*}
\Phi_N &=& \begin{bmatrix} \Psi_{1}^T \\ \Psi_{2}^T \\ \vdots \\ \Psi_{N}^T \end{bmatrix}  \\
P_N^{-1} &=& \frac{1}{c^2}I+
\begin{bmatrix} \Psi_{1} & \Psi_{2} & \cdots & \Psi_{N} \end{bmatrix}
\begin{bmatrix} \Psi_{1}^T \\ \Psi_{2}^T \\ \vdots \\ \Psi_{N}^T \end{bmatrix}  \\
&=& \frac{1}{c^2}I+\Phi^T\Phi \\
\lim_{c\to\infty}P_N^{-1} &=& \Phi_N^T\Phi_N  \\
\hat\theta_N &=& P_N\Phi_N^T Y_N \\
 &=& (\Phi_N^T\Phi_N)^{-1}\Phi_N^T Y_N 
\end{eqnarray*}
\end{frame}

\section{问题讨论}
\begin{frame}{残差与新息的关系}
新息(Innovation)$\tilde y_i = y_i - \Psi_i^T \hat\theta_{i-1}$用来描述$i$时刻的预报误差。
残差$\varepsilon_i = y_i - \Psi_i^T\hat\theta_i$用来描述$i$时刻的输出偏差。
\begin{eqnarray*}
\varepsilon &=& y_i - \Psi_i^T\hat\theta_i  \\
&=& y_i - \Psi_i^T(\hat\theta_{i-1}+K_i \tilde y_i) \\
&=& \tilde y_i -\Psi_i^T K_i \tilde y_i \\
&=& (1-\Psi_i^T K_i) \tilde y_i \\
&=& (1-\Psi_i^T P_{i-1} \Psi_i(\Psi_i^T P_{i-1} \Psi_i+1)^{-1}) \tilde y_i \\
&=& \frac{\Psi_i^T P_{i-1} \Psi_i+1-\Psi_i^T P_{i-1} \Psi_i}{\Psi_i^T P_{i-1} \Psi_i+1} \tilde y_i \\
&=& \frac{\tilde y_i}{\Psi_i^T P_{i-1} \Psi_i+1}  
\end{eqnarray*}
\end{frame}

\begin{frame}{准则函数的递推计算}
\begin{eqnarray*}
J_{i} &=& (Y_i-\Phi_i\theta_i)^T (Y_i-\Phi_i\theta_i)   \\
J_{i-1} &=& (Y_{i-1}-\Phi_{i-1}\theta_{i-1})^T (Y_{i-1}-\Phi_{i-1}\theta_{i-1}) \\
Y_i-\Phi_i\theta_i &=& Y_i - \Phi_i (\hat\theta_{i-1}+K_i \tilde y_i) \\
&=& \begin{bmatrix}  Y_{i-1}  \\  y_i \end{bmatrix} -\begin{bmatrix}\Phi_{i-1} \\ \Psi_i^T \end{bmatrix} (\hat\theta_{i-1}+ K_i \tilde y_k)  \\
&=& \begin{bmatrix} Y_{i-1}-\Phi_{i-1}\hat\theta_{i-1}  \\ \tilde y_i \end{bmatrix} -\begin{bmatrix}\Phi_{i-1} \\ \Psi_i^T \end{bmatrix}  K_i \tilde y_k  \\
\end{eqnarray*}
\end{frame}

\begin{frame}{准则函数的递推计算}
\begin{eqnarray*}
J_i &=& J_{i-1} - 2 K_i^T\Phi_{i-1}^T(Y_{i-1}-\Phi_{i-1}\hat\theta_{i-1})\tilde y_i + K_i^T \Phi_{i-1}^T\Phi_{i-1} K_i \tilde y_i^2  \\
&& +(1-2K_i^T\Psi_i+K_i^T\Psi_i\Psi_i^T K_i)\tilde y_i^2   \\
&=& J_{i-1} - 2 K_i^T(\Phi_{i-1}^TY_{i-1}-\Phi_{i-1}^T\Phi_{i-1}\hat\theta_{i-1})\tilde y_i  \\
&&  +(1-2K_i^T\Psi_i+K_i^T\Phi_i\Phi_i^T K_i)\tilde y_i^2   \\
&=& J_{i-1}  +  (1-2K_i^T\Psi_i+K_i^T\Phi_i\Phi_i^T K_i)\tilde y_i^2   \\
&=& J_{i-1}  +  (1-2K_i^T\Psi_i+K_i^T P_i^{-1} K_i)\tilde y_i^2   \\
&=& J_{i-1}  +  (1-2K_i^T\Psi_i+K_i^T \Psi_i)\tilde y_i^2   \\
&=& J_{i-1}  +  (1-K_i^T\Psi_i)\tilde y_i^2   \\
&=& J_{i-1}  +  (1-\Psi_i^T P_{i-1} \Psi_i(\Psi_i^T P_{i-1} \Psi_i+1)^{-1}) \tilde y_i^2 \\
&=& J_{i-1}  +  \frac{\Psi_i^T P_{i-1} \Psi_i+1-\Psi_i^T P_{i-1} \Psi_i}{\Psi_i^T P_{i-1} \Psi_i+1} \tilde y_i^2 \\
&=& J_{i-1}  +  \frac{\tilde y_i^2}{\Psi_i^T P_{i-1} \Psi_i+1}  
\end{eqnarray*}
\end{frame}
\begin{frame}{增益矩阵$K_i$的计算误差对$P_i$的影响}
当$K_i$存在误差$\delta K_i$时:
\begin{eqnarray*}
\delta P_i &=& \delta K_i \Psi_i^T P_{i-1}
\end{eqnarray*}
计算$P_i$的新形式:
\begin{eqnarray*}
P_i &=& (I-K_i\Psi_i^T)P_{i-1}  \\
&=& (I-K_i\Psi_i^T)P_{i-1} -P_{i-1}\Psi_i K_i^T+P_{i-1}\Psi_i K_i^T \\
&=& (I-K_i\Psi_i^T)P_{i-1} -P_{i-1}\Psi_i K_i^T+K_i(\Psi_i^T P_{i-1}\Psi_i+1)K_i^T \\
&=& (I-K_i\Psi_i^T)P_{i-1} -(I-K_i\Psi_i^T)P_{i-1}\Psi_i K_i^T+K_i K_i^T \\
&=& (I-K_i\Psi_i^T)(P_{i-1} -P_{i-1}\Psi_i K_i^T)+K_i K_i^T \\
&=& (I-K_i\Psi_i^T)P_{i-1} (I-\Psi_i K_i^T)+K_i K_i^T 
\end{eqnarray*}
\end{frame}

\begin{frame}{增益矩阵$K_i$的计算误差对$P_i$的影响}
当$K_i$存在误差$\delta K_i$时:
\begin{eqnarray*}
\delta P_i  &=& (I-(K_i+\delta K_i)\Psi_i^T)P_{i-1} (I-\Psi_i (K_i+\delta K_i)^T) \\
&& +(K_i+\delta K_i) (K_i+\delta K_i)^T -P_i  \\
 &=& -\delta K_i\Psi_i^T P_{i-1} (I-\Psi_i K_i^T)+K_i \delta K_i^T  \\
 && -(I-K_i\Psi_i^T)P_{i-1} \Psi_i \delta K_i^T+\delta K_i K_i^T   \\
 && + \delta K_i\Psi_i^T P_{i-1} \Psi_i \delta K_i^T+\delta K_i \delta K_i^T  \\
 && +(I-K_i\Psi_i^T)P_{i-1} (I-\Psi_i K_i^T)+K_i K_i^T -P_i  \\
 &=& -\delta K_i\Psi_i^T P_{i-1} (I-\Psi_i K_i^T)+K_i \delta K_i^T  \\
 && -(I-K_i\Psi_i^T)P_{i-1} \Psi_i \delta K_i^T+\delta K_i K_i^T  +O(\delta K_i) \\
 &=& -\delta K_i\Psi_i^T P_i^T+\delta K_i K_i^T -P_i \Psi_i \delta K_i^T+ K_i \delta K_i^T +O(\delta K_i) \\
 &=& -\delta K_i K_i^T +\delta K_i K_i^T -K_i \delta K_i^T+ K_i \delta K_i^T +O(\delta K_i) \\
 &=& O(\delta K_i) 
\end{eqnarray*}
\end{frame}

\begin{frame}{递推算法的稳定性:差分方程}
\begin{eqnarray*}
y_i &=& \Psi_i^T \theta+\xi_i \\
\tilde\theta_i &\stackrel{def}{=}&\theta-\hat\theta_i \\
&=& \theta-[\hat\theta_{i-1} + K_i (y_i-\Psi_i^T\hat\theta_{i-1})] \\
&=& \tilde\theta_{i-1}-K_i (y_i-\Psi_i^T\hat\theta_{i-1}) \\
&=& \tilde\theta_{i-1}-K_i (\Psi_i^T\theta + \xi_i -\Psi_i^T\hat\theta_{i-1}) \\
&=& \tilde\theta_{i-1}-K_i (\Psi_i^T\tilde\theta_{i-1} + \xi_i ) \\
&=& (I-K_i \Psi_i^T)\tilde\theta_{i-1} - K_i\xi_i \\
&=& P_i P_{i-1}^{-1}\tilde\theta_{i-1} - K_i\xi_i \\
&=& A_i \tilde\theta_{i-1} - K_i\xi_i \\
A_i &=& P_i P_{i-1}^{-1} 
\end{eqnarray*}
\end{frame}

\begin{frame}{递推算法的稳定性:特征值}
\begin{eqnarray*}
A_i x &=& \lambda x \\
(P_{i-1}^{-1}+\Psi_i\Psi_i^T)^{-1} P_{i-1}^{-1}x &=& \lambda x \\
P_{i-1}^{-1}x &=& [P_{i-1}^{-1}+\Psi_i\Psi_i^T]\lambda x \\
(1-\lambda)P_{i-1}^{-1}x &=& \lambda \Psi_i\Psi_i^T x \\
(1-\lambda) x^T P_{i-1}^{-1}x &=& \lambda x^T \Psi_i\Psi_i^T x 
\end{eqnarray*}
其中:$ P_{i-1}^{-1} $正定,与$\Psi_i\Psi_i^T$非负定,所以$0<\lambda \leq 1$。即:$\tilde\theta_i\leq\tilde\theta_0$。
\end{frame}

\begin{frame}{最小二乘估计与Kalman滤波的关系}
状态模型:
\begin{eqnarray*}
\theta_{i+1} &=& \theta_i \\
y_i &=& \Psi_i^T\theta_i +\xi_i
\end{eqnarray*}
Kalman滤波器:
\begin{eqnarray*}
\hat\theta_i &=& \hat\theta_{i-1} +K_i(y_i-\Psi_i^T\hat\theta_{i-1})  \\
K_i &=& S_i\Psi_i(\Psi_i^T S_i\Psi_i+\sigma^2)^{-1}  \\
S_i &=&  P_{i-1} \\
P_i &=&   (I-K_i\Psi_i^T)P_{i-1} \\
\hat\theta_0 &=& 0 \\
\end{eqnarray*}
\end{frame}

%\begin{frame}{作业}
%\begin{itemize}
%\item 复习自相关函数、互相关函数概念
%\item 推导基本最小二乘辨识算法
%\item 推导可辨识条件
%\item 推导课本中关于最小二乘概率性质的结论
%\end{itemize}
%\end{frame}
%\end{CJK*}
\end{document}
