% Created 2017-05-18 Thu 19:21
\documentclass{beamer}
\usepackage{fixltx2e}
\usepackage{graphicx}
\usepackage{longtable}
\usepackage{float}
\usepackage{wrapfig}
\usepackage{soul}
\usepackage{textcomp}
\usepackage{marvosym}
\usepackage{wasysym}
\usepackage{latexsym}
\usepackage{amssymb}
\usepackage{hyperref}
\tolerance=1000
\usepackage{etex}
\usepackage{amsmath}
\usepackage{pstricks}
\usepackage{pgfplots}
\usepackage{tikz}
\usepackage[europeanresistors,americaninductors]{circuitikz}
\usepackage{colortbl}
\usepackage{yfonts}
\usetikzlibrary{shapes,arrows}
\usetikzlibrary{positioning}
\usetikzlibrary{arrows,shapes}
\usetikzlibrary{intersections}
\usetikzlibrary{calc,patterns,decorations.pathmorphing,decorations.markings}
\usepackage[BoldFont,SlantFont,CJKchecksingle]{xeCJK}
\setCJKmainfont[BoldFont=Evermore Hei]{Evermore Kai}
\setCJKmonofont{Evermore Kai}
\usepackage{pst-node}
\usepackage{pst-plot}
\psset{unit=5mm}
\usepackage{beamerarticle}
\mode<beamer>{\usetheme{Frankfurt}}
\mode<beamer>{\usecolortheme{dove}}
\mode<article>{\hypersetup{colorlinks=true,pdfborder={0 0 0}}}
\AtBeginSection[]{\begin{frame}<beamer>\frametitle{Topic}\tableofcontents[currentsection]\end{frame}}
\setbeamercovered{transparent}
\providecommand{\alert}[1]{\textbf{#1}}

\title{频域滤波}
\author{}
\date{}
\hypersetup{
  pdfkeywords={},
  pdfsubject={},
  pdfcreator={Emacs Org-mode version 7.9.3f}}

\begin{document}

\maketitle

\begin{frame}
\frametitle{Outline}
\setcounter{tocdepth}{3}
\tableofcontents
\end{frame}













\section{背景}
\label{sec-1}
\begin{frame}
\frametitle{傅里叶与傅里叶变换}
\label{sec-1-1}

\begin{itemize}
\item 法国数学家傅里叶(生于1768年)在1822年出版的《热分析理论》一书中指出:任何周期函数都可以表达为不同频率的正弦和或余弦和的形式,即傅立叶级数。
\item 20世纪50年代后期,快速傅立叶变换算法出现,得到了广泛的应用。
\end{itemize}
\end{frame}
\begin{frame}
\frametitle{Joseph Fourier}
\label{sec-1-2}

\begin{center}
\includegraphics[width=0.3\textwidth]{./image/Fourier2.jpg}
\end{center}
\end{frame}
\begin{frame}
\frametitle{函数表示}
\label{sec-1-3}

\begin{itemize}
\item $f(x)$
\item $f(x)=f(0)+f'(0)x+f''(0)\frac{t^2}{2!}+\cdots$
\end{itemize}
\end{frame}
\section{一维傅里叶变换及其反变换}
\label{sec-2}
\begin{frame}
\frametitle{连续傅里叶变换}
\label{sec-2-1}

\begin{itemize}
\item 连续函数f(x)的傅立叶变换F(u):
\end{itemize}
\[ F(u) = \int_{-\infty}^{\infty}f(x) e^{-j2\pi ux}dx \]
\begin{itemize}
\item 傅立叶变换F(u)的反变换:
\end{itemize}
\[ f(u) = \int_{-\infty}^{\infty}F(u) e^{j2\pi ux}du \]
\end{frame}
\begin{frame}
\frametitle{离散傅里叶变换}
\label{sec-2-2}

\begin{itemize}
\item 离散函数 $f(x)$ (其中 $x,u=0,1,2,\cdots,M-1$ )的傅里叶变换:
  \[ F(u) = \frac{1}{M}\sum_{x=0}^{M-1}f(x)e^{\frac{-j2\pi u x}{M}}\]
\item 反变换
  \[ f(x) = \sum_{u=0}^{M-1}F(u)e^{\frac{-j2\pi u x}{M}}\]
\end{itemize}
\end{frame}
\begin{frame}
\frametitle{频谱}
\label{sec-2-3}

\[ F(u) = |F(u)|e^{-j\phi(u)} \]
\begin{itemize}
\item 频率谱
\end{itemize}
\[ |F(u)| =\sqrt{R^2(u)+I^2(u)} \]
\begin{itemize}
\item 相位谱
\end{itemize}
\[ \phi(u) = \arctan\frac{I(u)}{R(u)} \]
\begin{itemize}
\item 功率谱
\end{itemize}
\[ P(u) =R^2(u)+I^2(u) \]
\end{frame}
\section{二维离散傅里叶变换}
\label{sec-3}
\begin{frame}
\frametitle{二维离散傅里叶变换与反变换}
\label{sec-3-1}

\begin{itemize}
\item 一个图像尺寸为 $M\times N$ 的函数 $f(x,y)$ 的离散傅立叶变换 $F(u,v)$ :
\end{itemize}
\[ F(u,v) =\frac{1}{MN}\sum_{x=0}^{M-1}\sum_{y=0}^{N-1}f(x,y)e^{-j2\pi(\frac{ux}{M}+\frac{vy}{N})}\]
\begin{itemize}
\item 反变换:
\end{itemize}
\[ f(x,y) =\sum_{u=0}^{M-1}\sum_{v=0}^{N-1}F(u,v)e^{j2\pi(\frac{ux}{M}+\frac{vy}{N})}\]
\end{frame}
\begin{frame}
\frametitle{频谱}
\label{sec-3-2}

\[ F(u,v) = |F(u,v)|e^{-j\phi(u,v)} \]
\begin{itemize}
\item 频率谱
\end{itemize}
\[ |F(u,v)| =\sqrt{R^2(u,v)+I^2(u,v)} \]
\begin{itemize}
\item 相位谱
\end{itemize}
\[ \phi(u,v) = \arctan\frac{I(u,v)}{R(u,v)} \]
\begin{itemize}
\item 功率谱
\end{itemize}
\[ P(u,v) =R^2(u,v)+I^2(u,v) \]
\end{frame}
\section{二维傅里叶变换的性质}
\label{sec-4}
\begin{frame}
\frametitle{线性}
\label{sec-4-1}

\[af_1 (x,y)+bf_2(x,y) \Leftrightarrow aF_1(u,v)+bF_2(u,v) \]
\end{frame}
\begin{frame}
\frametitle{平移特性}
\label{sec-4-2}

\begin{align*}
f(x,y)e^{j2\pi(\frac{u_0 x}{M}+\frac{v_0 y}{N})} &\Leftrightarrow F(u-u_0,v-v_0) \\
f(x-x_0,y-y_0) &\Leftrightarrow F(u,v)e^{-j2\pi(\frac{u x_0}{M}+\frac{v y_0}{N})}
\end{align*}
当 $u_0=\frac{M}{2},v_0=\frac{N}{2}$ 时,
\begin{align*}
f(x,y)e^{j2\pi(\frac{u_0 x}{M}+\frac{v_0 y}{N})} &= f(x,y)e^{-j\pi(x+y)} \\
&=f(x,y)(-1)^{x+y} \\
{\cal F}[f(x,y)(-1)^{x+y}] &= F(u-\frac{M}{2},v-\frac{N}{2})
\end{align*}
\end{frame}
\begin{frame}
\frametitle{共轭对称性}
\label{sec-4-3}

\begin{align*}
F(u,v)&=F^{*}(-u,-v)\\
|F(u,v)|&=|F(-u,-v)|
\end{align*}
\end{frame}
\begin{frame}
\frametitle{比例}
\label{sec-4-4}

\[f(ax,by) \Leftrightarrow \frac{1}{|ab|}F(\frac{u}{a},\frac{v}{b})\]
\end{frame}
\begin{frame}
\frametitle{微分}
\label{sec-4-5}

\begin{align*}
\frac{\partial^n f(x,y)}{\partial x^n} &\Leftrightarrow (ju)^n F(u,v) \\
(-jx)^n f(x,y) &\Leftrightarrow \frac{\partial^n F(x,y)}{\partial x^n}
\end{align*}
拉普拉斯算子:
\[\nabla^2 f(x,y) =\frac{\partial^2 f}{\partial x^2}+\frac{\partial^2 f}{\partial y^2} \Leftrightarrow -(u^2+v^2)F(u,v)  \]
\end{frame}
\begin{frame}
\frametitle{卷积}
\label{sec-4-6}

\begin{align*}
f(x,y)*h(x,y) &= \frac{1}{MN}\sum_{m=0}^{M-1}\sum_{n=0}^{N-1}f(m,n)h(x-m,y-n) \\
\mathcal{F}[f(x,y)*h(x,y)] &= F(u,v)H(u,v)\\
f(x,y)h(x,y) &= \mathcal{F}^{-1}[F(u,v)*H(u,v)]
\end{align*}
\end{frame}
\section{频域滤波}
\label{sec-5}
\begin{frame}
\frametitle{频率域的基本性质}
\label{sec-5-1}

\begin{itemize}
\item 低频对应图像中变化平缓的区域
\item 高频对应图像中变化剧烈的区域(噪声、边缘等)
\end{itemize}
\end{frame}
\begin{frame}
\frametitle{频域滤波步骤}
\label{sec-5-2}

\begin{itemize}
\item 计算图像的DFT $F(u,v)=\mathcal{F}[f(u,v)]$
\item 滤波, 滤波器为 $H(u,v)$
  \[G(u,v) = H(u,v) F(u,v)\]
\item 反变换  $g(u,v)=\mathcal{F}^{-1}[G(u,v)]$
\end{itemize}
\end{frame}
\begin{frame}
\frametitle{基本滤波器性质}
\label{sec-5-3}

\begin{itemize}
\item 低通滤波器:通低频阻高频,用于图像平滑,消除高频噪声
\item 高通滤波器:通高频阻低频,用于图像锐化,检测边缘
\end{itemize}
\end{frame}
\begin{frame}
\frametitle{空间域滤波与频率域滤波关系}
\label{sec-5-4}

\begin{itemize}
\item 在频域指定滤波器
\item 反变换
\item 在空域执行卷积完成滤波
\end{itemize}
\end{frame}
\section{基于高斯函数的滤波}
\label{sec-6}
\begin{frame}
\frametitle{高斯函数低通滤波器}
\label{sec-6-1}

\begin{itemize}
\item 高斯滤波器函数(低通):
\end{itemize}
\[ H(u)=Ae^{\frac{-u^2}{2\delta^2}} \]
\begin{itemize}
\item 对应的空间域函数:
\end{itemize}
\[ h(x)=\sqrt{2\pi}\delta Ae^{-2\pi^2\delta^2 x^2}\]
\end{frame}
\begin{frame}
\frametitle{高斯滤波器特性}
\label{sec-6-2}

\begin{itemize}
\item 频域和空域高斯滤波器都是实函数
\item 高斯曲线直观,易于操作
\item 高斯滤波器参数: $\delta$ 增大,则 $H(u)$ 变宽 $h(x)$ 变窄
   \begin{align*}
   \lim_{\delta\to\infty} H(u) &=A \\
   \lim_{\delta\to\infty} h(x) &=\infty
   \end{align*}
\end{itemize}
\end{frame}
\section{低通滤波器}
\label{sec-7}
\begin{frame}[fragile]
\frametitle{理想低通滤波器}
\label{sec-7-1}

\begin{itemize}
\item 理想低通滤波器函数
\end{itemize}
\[ H(u,v) =\begin{cases} 1, \qquad D(u,v) \leq D_0 \\ 0, \qquad D(u,v) > D_0 \end{cases} \]
\begin{itemize}
\item 振铃现象示例:
\end{itemize}

\begin{verbatim}
f(x)=[1,1,0,0,0,0,0,0,0,0,0,0]
h(x)=[1,1,1,1,0,0,0,0,0,1,1,1]
g(x)=[0.89,0.89,0.23,0.17,0.0,0.11,
      0.06,0.06,0.11,0.0,0.17,0.23]
\end{verbatim}
\end{frame}
\section{频域锐化滤波器}
\label{sec-8}
\begin{frame}
\frametitle{理想高通滤波器}
\label{sec-8-1}

\begin{itemize}
\item 理想高通滤波器函数
\end{itemize}
\[ H(u,v) = \begin{cases} 0, \qquad D(u,v)\leq D_0 \\    1, \qquad D(u,v) > D_0 \end{cases} \]
\end{frame}
\begin{frame}
\frametitle{陷波滤波器}
\label{sec-8-2}

\[ H(u,v) = \begin{cases} 0, \qquad (u,v)=(0,0) \\    1, \qquad \text{其它} \end{cases} \]
\end{frame}

\end{document}
