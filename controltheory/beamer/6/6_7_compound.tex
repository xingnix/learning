% Created 2014-11-23 星期日 22:31
\documentclass[table]{beamer}
\usepackage{fixltx2e}
\usepackage{graphicx}
\usepackage{longtable}
\usepackage{float}
\usepackage{wrapfig}
\usepackage{soul}
\usepackage{textcomp}
\usepackage{marvosym}
\usepackage{wasysym}
\usepackage{latexsym}
\usepackage{amssymb}
\usepackage{hyperref}
\tolerance=1000
\usepackage{etex}
\usepackage{amsmath}
\usepackage{pstricks}
\usepackage{pgfplots}
\pgfplotsset{compat=1.5}
\usepackage{tikz}
\usepackage[europeanresistors,americaninductors]{circuitikz}
\usepackage{colortbl}
\usepackage{yfonts}
\usetikzlibrary{shapes,arrows}
\usetikzlibrary{positioning}
\usetikzlibrary{arrows,shapes}
\usetikzlibrary{intersections}
\usetikzlibrary{calc,patterns,decorations.pathmorphing,decorations.markings}
\usepackage[BoldFont,SlantFont,CJKchecksingle]{xeCJK}
\setCJKmainfont[BoldFont=Evermore Hei]{Evermore Kai}
\setCJKmonofont{Evermore Kai}
\usepackage{pst-node}
\usepackage{pst-plot}
\psset{unit=5mm}
\mode<beamer>{\usetheme{Frankfurt}}
\mode<beamer>{\usecolortheme{dove}}
\mode<article>{\hypersetup{colorlinks=true,pdfborder={0 0 0}}}
\mode<beamer>{\AtBeginSection[]{\begin{frame}<beamer>\frametitle{Topic}\tableofcontents[currentsection]\end{frame}}}
\setbeamercovered{transparent}
\subtitle{复合校正}
\providecommand{\alert}[1]{\textbf{#1}}

\title{线性系统校正方法}
\author{邢超}
\date{}
\hypersetup{
  pdfkeywords={},
  pdfsubject={},
  pdfcreator={Emacs Org-mode version 7.9.3f}}

\begin{document}

\maketitle

\begin{frame}
\frametitle{Outline}
\setcounter{tocdepth}{3}
\tableofcontents
\end{frame}













\section{复合校正特点}
\label{sec-1}
\begin{frame}
\frametitle{复合校正特点}
\label{sec-1-1}

\begin{tikzpicture}[node distance=2em,auto,>=latex', thick]
%\path[use as bounding box] (-1,0) rectangle (10,-2); 
\path[->] node[] (r) {$R(s)$}; 
\path[->] node[ circle,inner sep=2pt,minimum size=1pt,draw,label=below left:$   $ ,right =of r] (p1) {}; 
\path[->](r) edge node {} (p1) ; 
\path[->] node[draw, right =of p1] (gc) {$G_c(s)$}; 
\path[->](p1) edge node {} (gc) ; 
\path[->] node[ circle,inner sep=2pt,minimum size=1pt,draw,label=below left:$   $ ,right =of gc] (p2) {}; 
\path[->](gc) edge node {} (p2) ; 
\path[red,->] node[draw, right =of p2] (g1) {$G_1(s)$}; 
\path[->] (p2) edge node {} (g1) ; 
\path[->] node[ circle,inner sep=2pt,minimum size=1pt,draw,label=below left:$   $ ,right =of g1] (p3) {}; 
\path[->] (g1) edge node {} (p3) ; 
\path[red,->] node[draw, right =of p3] (g2) {$G_2(s)$}; 
\path[->] (p3) edge node {} (g2); 
\path[->] node[ right =of g2] (o) {$C(s)$}; 
\path[->] (g2) edge node {} (o); 

\path[blue,->] node[draw, above =of gc] (gr) {$G_r(s)$}; 
\path[->] node[ circle,inner sep=2pt,minimum size=1pt,draw,label=below left:$   $ ,right =of gr] (pr) {}; 
\path[ draw] (r.east)+(1em,0) |-   (gr.west); 
\path[->, draw] (gr.east) -- (pr); 
\path[blue,->] node[draw, above =of g1] (gn) {$G_n(s)$}; 
\path[ draw] (p3.north)+(0,5em) |-   (gn.east); 
\path[->,draw] (p3.north)+(0,5em) -- node[very near end] {$+$} node[very near start] {$N(s)$} (p3.north) ; 
\path[->, draw] (gn.west) -- node[very near end] {$+$} (pr); 
\path[->, draw] (pr) -- node[very near end] {$+$} (p2); 

\path[->, draw] (g2.east)+(1em,0) -- +(1em,-3em) -| node[very near end] {$-$} (p1); 
\end{tikzpicture} 

\begin{itemize}
\item 应用场合:系统中存在可测量的扰动,或者对系统的稳态精度和响应速度要求很高
\end{itemize}
\begin{itemize}
\item 校正方式:
\begin{itemize}
\item 按扰动补偿
\item 按输入补偿
\end{itemize}
\end{itemize}
\end{frame}
\section{按扰动补偿的复合校正}
\label{sec-2}
\begin{frame}
\frametitle{按扰动补偿的复合校正}
\label{sec-2-1}


\begin{tikzpicture}[node distance=2em,auto,>=latex', thick]
%\path[use as bounding box] (-1,0) rectangle (10,-2); 
\path[->] node[] (r) {$R(s)$}; 
\path[->] node[ circle,inner sep=2pt,minimum size=1pt,draw,label=below left:$   $ ,right =of r] (p1) {}; 
\path[->](r) edge node {} (p1) ; 
\path[->] node[ circle,inner sep=2pt,minimum size=1pt,draw,label=below left:$   $ ,right =of p1] (p2) {}; 
\path[->](p1) edge node {} (p2) ; 
\path[red,->] node[draw,  inner sep=5pt,right =of p2] (g1) {$G_1(s)$}; 
\path[->] (p2) edge node {} (g1) ; 
\path[->] node[ circle,inner sep=2pt,minimum size=1pt,draw,label=below left:$   $ ,right =of g1] (p3) {}; 
\path[->] (g1) edge node {} (p3) ; 
\path[red,->] node[draw, inner sep=5pt,right =of p3] (g2) {$G_2(s)$}; 
\path[->] (p3) edge node {} (g2); 
\path[->] node[ right =of g2] (o) {$C(s)$}; 
\path[->] (g2) edge node {} (o); 
\path[blue,->] node[draw, inner sep=5pt,above =of g1] (gn) {$G_n(s)$}; 
\path[ draw] (p3.north)+(0,5em) |-   (gn.east); 
\path[->,draw] (p3.north)+(0,5em) -- node[very near end] {$+$} node[very near start] {$N(s)$} (p3.north) ; 
\path[->, draw] (gn.west) -| node[very near end] {$+$} (p2); 
\path[->, draw] (g2.east)+(1em,0) -- +(1em,-3em) -| node[very near end] {$-$} (p1); 
\end{tikzpicture} 

\begin{itemize}
\item 目的:使扰动不对系统的输出产生任何影响: $\Phi_N(s)=0$
\item 条件:扰动可测
\end{itemize}
\end{frame}
\begin{frame}
\frametitle{按扰动补偿的复合校正:设计 $G_n(s)$}
\label{sec-2-2}

\begin{itemize}
\item 对扰动的误差全补偿条件:
    \begin{eqnarray*}
    \Phi_N(s)&=&0 \\
    \frac{G_2+G_nG_1G_2}{1+G_1G_2} & = & 0 \\
    G_2+G_nG_1G_2 &=& 0 \\
    G_n &=& \frac{-1}{G_1}
    \end{eqnarray*}
\item <2->  全补偿时, $G_n(s)$ 的分子阶次大于分母阶次, 物理上不可实现.
\item <3-> 部分补偿
\begin{itemize}
\item 在系统性能起主要影响的频段内近似补偿($n\geq m$),
\item 稳态补偿($\lim_{s\to 0}G_n(s)$)
\end{itemize}
\end{itemize}
\end{frame}
\begin{frame}
\frametitle{按扰动补偿的复合校正示例1}
\label{sec-2-3}

某伺服控制系统结构图如下:

\begin{tikzpicture}[node distance=2em,auto,>=latex', thick]
%    gn<------------N(s)
%    v +            kn/km
%   --o--k1/(T1s+1) -o---  km/s/(Tm s+1)--+ 
%     ^-------------------------------/
%\path[use as bounding box] (-1,0) rectangle (10,-2); 
\path[->] node[] (r) {$R(s)$}; 
\path[->] node[ circle,inner sep=2pt,minimum size=1pt,draw,label=below left:$   $ ,right =of r] (p1) {}; 
\path[->](r) edge node {} (p1) ; 
\path[->] node[ circle,inner sep=2pt,minimum size=1pt,draw,label=below left:$   $ ,right =of p1] (p2) {}; 
\path[->](p1) edge node {} (p2) ; 
\path[red,->] node[draw,  inner sep=5pt,right =of p2] (g1) {$\frac{K_1}{T_1 s+1}$}; 
\path[->] (p2) edge node {} (g1) ; 
\path[->] node[ circle,inner sep=2pt,minimum size=1pt,draw,label=below left:$   $ ,right =of g1] (p3) {}; 
\path[->] (g1) edge node {} (p3) ; 
\path[red,->] node[draw, inner sep=5pt,right =of p3] (g2) {$\frac{K_m}{s(T_m s+1)}$}; 
\path[->] (p3) edge node {} (g2); 
\path[->] node[ right =of g2] (o) {$C(s)$}; 
\path[->] (g2) edge node {} (o); 
\path[blue,->] node[draw, inner sep=5pt,above =of p2] (gn) {$G_n(s)$}; 
\path[->, draw] (gn)-- node[very near end] {$+$} (p2); 
\path[red,->] node[draw, inner sep=5pt,above =of p3] (g3) {$\frac{K_n}{K_m}$}; 
\path[->] (g3) edge node {} node[very near end] {$+$} (p3); 
\path[->, draw] (g3.north)+(0,1em) -|   (gn.north); 
\path[->,draw] (g3.north)+(0,2em) --  node[very near start] {$N(s)$} (g3.north) ; 
\path[->, draw] (g2.east)+(1em,0) -- +(1em,-3em) -| node[very near end] {$-$} (p1); 
\end{tikzpicture} 

\begin{itemize}
\item 设计对  $N(s)$ 的全补偿校正网絡  $G_n(s)$  ,
\item 近似全补偿校正网絡  $G_{n1}(s)$  ,
\item 稳态全补偿网絡  $G_{n2}(s)$
\end{itemize}
\end{frame}
\begin{frame}
\frametitle{按扰动补偿的复合校正示例1:解:}
\label{sec-2-4}

\begin{itemize}
\item 全补偿:
      \begin{eqnarray*}
      \Phi_N(s) &= & 0\\
      \frac{C(s)}{N(s)} &=& 0\\
      \frac{K_n}{K_m}+G_n(s)\cdot\frac{K_1}{T_1 s+1}  &=& 0\\
      G_n(s) &=&-\frac{K_n(T_1 s+1)}{K_1 K_m}
      \end{eqnarray*}
\item <2->近似全补偿:  $G_{n1}(s)=-\frac{K_n}{K_1 K_m }\cdot\frac{T_1 s+1}{T_2 s+1}$  .其中  $T_1\gg T_2$  .
\item <3->稳态全补偿:  $G_{n1}(s)=-\frac{K_n}{K_1 K_m }$
\end{itemize}
\end{frame}
\section{按输入补偿的复合校正}
\label{sec-3}
\begin{frame}
\frametitle{按输入补偿的复合校正}
\label{sec-3-1}

目的:使输出完全跟踪输入信号,即  $C(s)=R(s)$ 

\begin{tikzpicture}[node distance=2em,auto,>=latex', thick] 
%\path[use as bounding box] (-1,0) rectangle (10,-2); 
\path[->] node[] (r) {$R(s)$}; 
\path[->] node[ right =of r] (rr) {}; 
\path[->] node[ circle,inner sep=2pt,minimum size=1pt,draw,label=below left:$   $ ,right =of rr] (p1) {}; 
\path[->](r) edge node {} (p1) ; 
\path[->] node[ circle,inner sep=2pt,minimum size=1pt,draw,label=below left:$   $ ,right =of p1] (p2) {}; 
\path[->](p1) edge node[midway] {$E(s)$} (p2) ; 
\path[red,->] node[draw, inner sep=5pt,right =of p2] (g) {$G(s)$}; 
\path[->] (p2) edge node {} (g); 
\path[->] node[ right =of g] (o) {$C(s)$}; 
\path[->] (g) edge node {} (o); 
\path[blue,->] node[draw, inner sep=5pt,above =of p1] (gr) {$G_r(s)$}; 
\path[->, draw] (rr.west) |-   (gr.west); 
\path[->, draw] (gr.east) -| node[very near end] {$+$} (p2); 
\path[->, draw] (g.east)+(1em,0) -- +(1em,-3em) -| node[very near end] {$-$} (p1); 
\end{tikzpicture} 

\begin{eqnarray*}
C(s) &=&(E(s)+G_r(s)R(s))G(s)\\
E(s) &=& R(s)-C(s)\\
E(s) &=&\frac{1-G_r(s)G(s)}{1+G(s)}R(s)
\end{eqnarray*}
 $G_r(s)$ 为按输入补偿的复合校正装置(前馈装置)传递函数.
\end{frame}
\begin{frame}
\frametitle{按输入补偿的复合校正分析}
\label{sec-3-2}

  对输入信号的误差全补偿条件: 
     \begin{align*}
     E(s) &= R(s)-C(s) \\
          &= \frac{1-G_r(s)G(s)}{1+G(s)}R(s)\\
          &= 0 \\
   G_r(s) &= \frac{1}{G(s)}
     \end{align*}
\end{frame}
\begin{frame}
\frametitle{按输入补偿的复合校正:部分补偿:}
\label{sec-3-3}

\begin{itemize}
\item 采用满足跟踪精度要求的部分补偿条件,在对系统性能起主要影响的频段内实现补偿,使  $G_r(s)$ 可物理实现.
\item <2->设反馈系统开环传递函数:  $G(s)$  ,取  $G_r(s)=\lambda_1 s$  得:
      \begin{align*}
      G(s) & =  \frac{K_v}{s(a_ns^{n-1}+a_{n-1}s^{n-2}+\cdots+a_1)} \\
      \Phi_e(s) &= \frac{1-G(s)G_r(s)}{1+G(s)} \\
              &= \frac{s(a_ns^{n-1}+a_{n-1}s^{n-2}+\cdots+a_1)-K_v\lambda_1 s}{s(a_ns^{n-1}+a_{n-1}s^{n-2}+\cdots+a_1)+K_v}
      \end{align*}
\end{itemize}
\end{frame}
\begin{frame}
\frametitle{按输入补偿的复合校正:部分补偿(续):}
\label{sec-3-4}

\begin{itemize}
\item 若取  $\lambda_1=\frac{a_1}{K_v}$  则有:
      \begin{align*}
       \Phi_e(s)&= \frac{s(a_ns^{n-1}+a_{n-1}s^{n-2}+\cdots+a_2s)}{s(a_ns^{n-1}+a_{n-1}s^{n-2}+\cdots+a_1)+K_v}
      \end{align*}
      系统为II型系统.
\item <2->同理,取  $G_r(s)=\lambda_1 s+\lambda_2 s^2, \lambda_1=\frac{a_1}{K_v},\lambda_2=\frac{a_2}{K_v}$ ,则
      \begin{align*}
       \Phi_e(s)&= \frac{s(a_ns^{n-1}+a_{n-1}s^{n-2}+\cdots+a_3s^2)}{s(a_ns^{n-1}+a_{n-1}s^{n-2}+\cdots+a_1)+K_v}
      \end{align*}
      系统为III型系统.
\end{itemize}
\end{frame}
\begin{frame}
\frametitle{前馈系统分析}
\label{sec-3-5}

\begin{itemize}
\item 前馈系统:
      \begin{tikzpicture}[node distance=2em,auto,>=latex', thick] 
      %\path[use as bounding box] (-1,0) rectangle (10,-2); 
      \path[->] node[] (r) {$R(s)$}; 
      \path[->] node[ right =of r] (rr) {}; 
      \path[->] node[ circle,inner sep=2pt,minimum size=1pt,draw,label=below left:$   $ ,right =of rr] (p1) {}; 
      \path[->](r) edge node {} (p1) ; 
      \path[->] node[draw, inner sep=5pt,right =of p1] (gc) {$G_c(s)$}; 
      \path[->] (p1) edge node[midway] {$E(s)$} (gc); 
      \path[->] node[ circle,inner sep=2pt,minimum size=1pt,draw,label=below left:$   $ ,right =of gc] (p2) {}; 
      \path[->](gc) edge  (p2) ; 
      \path[red,->] node[draw, inner sep=5pt,right =of p2] (g) {$G(s)$}; 
      \path[->] (p2) edge node {} (g); 
      \path[->] node[ right =of g] (o) {$C(s)$}; 
      \path[->] (g) edge node {} (o); 
      \path[blue,->] node[draw, inner sep=5pt,above =of gc] (gr) {$G_r(s)$}; 
      \path[->, draw] (rr.west) |-   (gr.west); 
      \path[->, draw] (gr.east) -| node[very near end] {$+$} (p2); 
      \path[->, draw] (g.east)+(1em,0) -- +(1em,-3em) -| node[very near end] {$-$} (p1); 
      \end{tikzpicture}
\item <2->误差全补偿 \mode<article>{,分析方法类似,}
        \begin{align*}
        C(s) &= R(s)\\
        C(s) &= R(s)G_r(s)G(s)+E(s)G_c(s)G(s)  \\
        E(s) &= R(s)-C(s)=0 \\
        G_r(s)G(s) &= 1 
     %  G_r(s) &= \frac{1}{G(s)}
        \end{align*}
\end{itemize}
\end{frame}
\begin{frame}
\frametitle{前馈系统部分补偿:}
\label{sec-3-6}

\begin{align*}
\Phi_e^{(0)}(s) &= \frac{1}{1+G_c(s)G(s)}  & \Phi_e(s) &= \frac{1-G(s)G_r(s)}{1+G_c(s)G(s)}
\end{align*}
\begin{itemize}
\item <2->设: 
   \begin{align*}
   G(s)G_r(s) &=\frac{\lambda_0+\lambda_1 s+\lambda_2 s^2+\cdots+\lambda_n s^n}{a_0+a_1 s+ a_2 s^2+\cdots+a_n s^n}
   \end{align*}
\item <3->得:
    \begin{align*}
    \Phi_e(s) &=\frac{a_0+a_1 s+\cdots+a_n s^n-(\lambda_0+\lambda_1 s+\cdots+\lambda_n s^n)}{(a_0+a_1 s+ a_2 s^2+\cdots+a_n s^n)(1+G_c(s)G(s))}
    \end{align*}
\item <4-> 将  $\Phi_e(s)$  与  $\Phi_e^{(0)}(s)$  比较可知,当 
     \begin{equation*}
     \begin{cases}
     \lambda_i=a_i & i=1,2,\cdots,k \\
     \lambda_i=0 & i=k+1,\cdots ,n 
     \end{cases}
     \end{equation*}
    时,系统类型可提高  $k$
\end{itemize}
\end{frame}
\begin{frame}
\frametitle{前馈系统分析(续)稳定性分析}
\label{sec-3-7}

      \begin{eqnarray*}
      \Phi_0(s) & = &\frac{G_c(s)G(s)}{1+G_c(s)G(s)} \\
      \Phi(s) & = &\frac{(G_c(s)+G_r(s))G(s)}{1+G_c(s)G(s)} 
      \end{eqnarray*}
\begin{itemize}
\item <2-> 当 $G_r(s)$ 极点实部小于0时,校正后系统稳定性不变。
\end{itemize}
\end{frame}
\begin{frame}
\frametitle{按输入补偿的复合校正示例1:}
\label{sec-3-8}


\begin{tikzpicture}[node distance=2em,auto,>=latex', thick] 
%\path[use as bounding box] (-1,0) rectangle (10,-2); 
\path[->] node[] (r) {$R(s)$}; 
\path[->] node[ right =of r] (rr) {}; 
\path[->] node[ circle,inner sep=2pt,minimum size=1pt,draw,label=below left:$   $ ,right =of rr] (p1) {}; 
\path[->](r) edge node {} (p1) ; 
\path[->] node[draw, inner sep=5pt,right =of p1] (g1) {$\frac{K_1}{T_1 s+1}$}; 
\path[->] (p1) edge node[midway] {$E(s)$} (g1); 
\path[->] node[ circle,inner sep=2pt,minimum size=1pt,draw,label=below left:$   $ ,right =of g1] (p2) {}; 
\path[->](g1) edge  (p2) ; 
\path[red,->] node[draw, inner sep=5pt,right =of p2] (g2) {$\frac{K_2}{s(T_2 s+1)}$}; 
\path[->] (p2) edge node {} (g2); 
\path[->] node[ right =of g2] (o) {$C(s)$}; 
\path[->] (g2) edge node {} (o); 
\path[blue,->] node[draw, inner sep=5pt,above =of g1] (gr) {$G_r(s)$}; 
\path[->, draw] (rr.west) |-   (gr.west); 
\path[->, draw] (gr.east) -| node[very near end] {$+$} (p2); 
\path[->, draw] (g2.east)+(1em,0) -- +(1em,-3em) -| node[very near end] {$-$} (p1); 
\end{tikzpicture} 

设计  $G_r(s)$ 
\begin{itemize}
\item 实现完全补偿
\item 使系统等效为II型系统
\item 使系统等效为III型系统
\end{itemize}
\end{frame}
\begin{frame}
\frametitle{按输入补偿的复合校正示例1(续):}
\label{sec-3-9}

\begin{itemize}
\item 取  $G_r(s)=\lambda_1 s+\lambda_2 s^2$ ,得:
    \begin{align*}
    \Phi_e &=\frac{1-G_r(s)\frac{K_2}{s(T_2 s+1)}}{1+\frac{K_1}{T_1 s+1}\frac{K_2}{s(T_2 s+1)}}\\
     &=\frac{s(T_2 s+1)-G_r(s)K_2}{s(T_2 s+1)(1+\frac{K_1}{T_1 s+1}\frac{K_2}{s(T_2 s+1)})}\\
     &=\frac{s(T_2 s+1)-(\lambda_1 s+\lambda_2 s^2) K_2}{s(T_2 s+1)(1+\frac{K_1}{T_1 s+1}\frac{K_2}{s(T_2 s+1)})}
    \end{align*}
\item 取  $\lambda_{1}=\frac{1}{K_2},\lambda_2=0$  则系统为II型系统,
\item 取  $\lambda_{1}=\frac{1}{K_2},\lambda_2=\frac{T_2}{K_2}$ 则能实现完全补偿.
\end{itemize}
\end{frame}
\begin{frame}
\frametitle{按输入补偿的复合校正示例1(续):}
\label{sec-3-10}

\begin{itemize}
\item 取 $G_r(s)=\frac{\lambda_1 s+\lambda_2 s^2}{Ts+1}$ ,得:
    \begin{align*}
    \Phi_e &=\frac{1-G_r(s)\frac{K_2}{s(T_2 s+1)}}{1+\frac{K_1}{T_1 s+1}\frac{K_2}{s(T_2 s+1)}}\\
     &=\frac{s(T_2 s+1)(Ts+1)-(\lambda_1 s+\lambda_2 s^2)K_2}{s(Ts+1)(T_2 s+1)(1+\frac{K_1}{T_1 s+1}\frac{K_2}{s(T_2 s+1)})}\\
     &=\frac{T T_2 s^3+(T+T_2)s^2+s-(\lambda_1 s+\lambda_2 s^2)K_2}{s(Ts+1)(T_2 s+1)(1+\frac{K_1}{T_1 s+1}\frac{K_2}{s(T_2 s+1)})}
    \end{align*}
\item 取 $\lambda_1=\frac{1}{K_2},\lambda_2=\frac{T_2+T}{K_2}$ 则系统为III型系统.
\end{itemize}
\end{frame}
\begin{frame}
\frametitle{按输入补偿的复合校正示例1(续):}
\label{sec-3-11}

\begin{itemize}
\item 取 $G_r(s)=\frac{\lambda_1 s+\lambda_2 s^2}{(T_3s+1)(T_4s+1)}$ ,得:
    \begin{align*}
     \Phi_e &=\frac{1-G_r(s)\frac{K_2}{s(T_2 s+1)}}{1+\frac{K_1}{T_1 s+1}\frac{K_2}{s(T_2 s+1)}}\\
       &=\frac{s(T_2 s+1)(T_3s+1)(T_4s+1)-(\lambda_1s+\lambda_2 s^2)K_2}{s(T_3s+1)(T_4s+1)(T_2 s+1)(1+\frac{K_1}{T_1 s+1}\frac{K_2}{s(T_2 s+1)})}
      % &=\frac{T_4 T_3 T_2 s^3+(T_4T_3+T_4T_2+T_3T_2)s^2+(T_4+T_3+T_2)s+1-(\lambda_1+\lambda_2 s)K_2}{(T_3s+1)(T_4s+1)(T_2 s+1)(1+\frac{K_1}{T_1 s+1}\frac{K_2}{s(T_2 s+1)})}
    \end{align*}
\item 取 $\lambda_1=\frac{1}{K_2},\lambda_2=\frac{T_2+T_3+T_4}{K_2}$ 则系统为III型系统.
\end{itemize}
\end{frame}

\end{document}
