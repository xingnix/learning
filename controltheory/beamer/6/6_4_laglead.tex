% Created 2014-11-23 星期日 22:50
\documentclass[table]{beamer}
\usepackage{fixltx2e}
\usepackage{graphicx}
\usepackage{longtable}
\usepackage{float}
\usepackage{wrapfig}
\usepackage{soul}
\usepackage{textcomp}
\usepackage{marvosym}
\usepackage{wasysym}
\usepackage{latexsym}
\usepackage{amssymb}
\usepackage{hyperref}
\tolerance=1000
\usepackage{etex}
\usepackage{amsmath}
\usepackage{pstricks}
\usepackage{pgfplots}
\pgfplotsset{compat=1.8}
\usepackage{tikz}
\usepackage[europeanresistors,americaninductors]{circuitikz}
\usepackage{colortbl}
\usepackage{yfonts}
\usetikzlibrary{shapes,arrows}
\usetikzlibrary{positioning}
\usetikzlibrary{arrows,shapes}
\usetikzlibrary{intersections}
\usetikzlibrary{calc,patterns,decorations.pathmorphing,decorations.markings}
\usepackage[BoldFont,SlantFont,CJKchecksingle]{xeCJK}
\setCJKmainfont[BoldFont=Evermore Hei]{Evermore Kai}
\setCJKmonofont{Evermore Kai}
\usepackage{pst-node}
\usepackage{pst-plot}
\psset{unit=5mm}
\mode<beamer>{\usetheme{Frankfurt}}
\mode<beamer>{\usecolortheme{dove}}
\mode<article>{\hypersetup{colorlinks=true,pdfborder={0 0 0}}}
\mode<beamer>{\AtBeginSection[]{\begin{frame}<beamer>\frametitle{Topic}\tableofcontents[currentsection]\end{frame}}}
\setbeamercovered{transparent}
\subtitle{串联滞后-超前校正}
\providecommand{\alert}[1]{\textbf{#1}}

\title{线性系统校正方法}
\author{}
\date{}
\hypersetup{
  pdfkeywords={},
  pdfsubject={},
  pdfcreator={Emacs Org-mode version 7.9.3f}}

\begin{document}

\maketitle

\begin{frame}
\frametitle{Outline}
\setcounter{tocdepth}{3}
\tableofcontents
\end{frame}












\section{串联滞后-超前校正原理与方法}
\label{sec-1}
\begin{frame}
\frametitle{串联滞后-超前校正原理}
\label{sec-1-1}

\begin{itemize}
\item 滞后部分的幅值衰减特性,提高系统的稳定程度
\item 超前部分的相角超前特性,提高系统的动态性能
\end{itemize}

\begin{math}
G_c = \frac{(1+T_a s)(1+T_b s)}{(1+aT_a s)(1+\frac{T_b}{a}s)}
\end{math}
其中 $a>1$ 
\begin{columns}
\begin{column}{0.5\textwidth}
\begin{block}<2->{校正网络Bode图}
\label{sec-1-1-1}


\begin{tikzpicture}[scale=0.7]
\draw[->] (-1,0) -- (6,0);
\draw[->] (0,-1.1) -- (0,1.1);
\draw (0,0.5) node[above left] {$L_c(\omega)$};
\draw [red,thick] plot coordinates {(0,0) (1,0) (2,-1)  (4,-1) (5,0) (6,0) };
\draw (1,0) node[above] {$\frac{1}{aT_a}$};
\draw (2,0) node[above] {$\frac{1}{T_a}$};
\draw (4,0) node[above] {$\frac{1}{T_b}$};
\draw (5,0) node[above] {$\frac{a}{T_b}$};

\begin{scope}[shift={(0,-3)}]
\draw[->] (-1,0) -- (6,0);
\draw[->] (0,-1.1) -- (0,0.5);
\draw (0,0.5) node[above left] {$\phi_c(\omega)$};
\draw [red,thick] plot [smooth] coordinates {
(0,0) (0.5,-0.1) (1,-0.45) (1.5,-1) (2,-0.45) (2.5,-0.1) (3,0)
(3.5,0.1) (4,0.45) (4.5,1) (5,0.45) (5.5,0.1) (6,0)};
\draw (1,0) node[above] {$\frac{1}{aT_a}$};
\draw (2,0) node[above] {$\frac{1}{T_a}$};
\draw (4,0) node[below] {$\frac{1}{T_b}$};
\draw (5,0) node[below] {$\frac{a}{T_b}$};
\end{scope}
\end{tikzpicture}
\end{block}
\end{column}
\begin{column}{0.5\textwidth}
\begin{block}<3->{滞后校正示意图:}
\label{sec-1-1-2}

\begin{tikzpicture}[scale=0.7]
\coordinate (o) at (0,0);
\coordinate (ox) at (4.5,0);
\draw[->] (-1,0) -- (ox);
\draw[->] (0,-1.1) -- (0,1.5);
\draw (0,0.5) node[above left] {$L(\omega)$};
\draw (o) node[below left] {$o$};
%\draw [red,thick] plot coordinates {(0,0) (1,0) (2,-1)  (3,-1)};
\coordinate (a) at (0.5,1.5);
\coordinate (b) at ($(a) +(-20:2)$);
\coordinate (c) at ($(b)+(-40:2)$);
\coordinate (w1) at (intersection of b--c and o--ox);

\coordinate (c1) at ($(b)+(-20:2)$);

\coordinate (a1) at ($(a)+(0,-0.5)$);
\coordinate (c2) at ($(a1)+(-20:4)$);
\coordinate (an) at ($(a)+(-40:2)$);
\coordinate (a2) at (intersection of a--an and a1--c2);
\coordinate (d2) at (intersection of a2--c2 and b--c);
\coordinate (w2) at (intersection of a2--c2 and o--ox);

\draw[red] (a)--($(a)+(160:0.3)$);
\draw[red] (a)--(b);
\draw[red] (b)--(c);
\draw[blue] (b)--(c1);

\draw[blue] (a)--(a2);
\draw[blue] (a2)--(d2);
%\draw[blue] (d2)--(c);

\draw (w1) node[pin=-80:$\omega_c'$] {};
\draw (w2) node[pin=-100:$\omega_c''$] {};

\begin{scope}[shift={(0,-3)}]
\draw[->] (-1,0) -- (4.5,0);
\draw[->] (0,-1.1) -- (0,0.5);
\draw (0,0.5) node[above left] {$\phi(\omega)$};
\draw [red,thick] plot [smooth] coordinates {(0,0) (0.3,-0.1) (0.6,-0.25) (1.1,-0.5) (2.5,-0.65) (3,-0.7) (3.7,-0.9) (3.9,-1)};
\draw [blue,thick] plot [smooth] coordinates {(0,0) (0.3,-0.3) (0.6,-0.55) (1.1,-0.6) (2.5,-0.33) (3,-0.3) (3.7,-0.35) (3.9,-0.4) (4.1,-0.5)};
\draw[dashed,red] (0,-1) -- (4.5,-1);
\draw (0,-1) node[left] {$-180^\circ$};
\end{scope}
\end{tikzpicture}
\end{block}
\end{column}
\end{columns}
\end{frame}
\begin{frame}
\frametitle{适用泛围}
\label{sec-1-2}

\begin{itemize}
\item 系统不稳定,且要求的稳态性能与动态性能较高时
\end{itemize}
\end{frame}
\begin{frame}
\frametitle{设计步聚}
\label{sec-1-3}

\begin{itemize}
\item 由 $e_{ss}$ 要求确定开环增益
\item 绘制未校正系统Bode图,计算 $\omega_c',\gamma',h'$
\item <2->选择  $\frac{1}{T_b}$  为  $L(\omega)$  上斜率由  $-20dB/dec$  至 $-40dB/dec$ 的交接频率(校正后截止频率处斜率为 $-20dB/dec$ ,且有一定宽度)
\item <3->计算期望的截止频率 $\omega_c''$ ,求解  $L(\omega_c'')+20\lg T_b\omega_c''-20\lg a=0$  得 $a$
\item <4->计算期望相角裕度 $\gamma''$ ,求解:  $180^\circ+\phi_c(\omega_c'')+\phi(\omega_c'')=\gamma''$  ,得 $T_a$
\end{itemize}
\end{frame}
\section{串联滞后-超前校正示例}
\label{sec-2}
\begin{frame}
\frametitle{串联滞后-超前校正示例1}
\label{sec-2-1}

单位负反馈系统  $G(s)=\frac{K}{s(0.1s+1)(0.01s+1)}$  设计校正网絡,满足  $K_v \geq 100,\gamma'>40,\omega_c''\approx 20$ 

解:
\begin{itemize}
\item <2->由稳态性能指标得:  $K=100$ 
      \begin{eqnarray*}
      L(\omega) & = &\begin{cases}20\lg\frac{100}{\omega} & \omega<10 \\
                                  20\lg\frac{100}{0.1\omega^2} & 10\leq \omega<100 \\
                                  20\lg\frac{100}{0.001\omega^3} & \omega \geq 100 \end{cases} \\
      \omega_c' &=& 31.6 \\
      \gamma' &=& 180^{\circ}-90^{\circ}-\arctan0.1\omega_c'-\arctan0.01\omega_c' \\
       &=& 0 
      \end{eqnarray*}
\end{itemize}
\end{frame}
\begin{frame}
\frametitle{串联滞后-超前校正示例1(续):选用滞后-超前校正}
\label{sec-2-2}


\[G_c = \frac{(1+T_a s)(1+T_b s)}{(1+aT_a s)(1+\frac{T_b}{a}s)}\]

根据截止频率  $\omega_c''$  确定  $T_b,a$ 
\begin{eqnarray*}
T_b & = & 0.1\\
L(\omega_c'') +20\lg0.1\omega_c'' &=& 20\lg\frac{100}{\omega_c''} \\
&=& 20\lg5\\
20\lg a &=& 20\lg5 \\
a &=& 5 
\end{eqnarray*}
\end{frame}
\begin{frame}
\frametitle{串联滞后-超前校正示例1(续):验证相角裕度,确定 $T_{a}$}
\label{sec-2-3}

\begin{itemize}
\item 原系统: $\phi(\omega_c'') = -90^{\circ}-\arctan 0.1\omega_c''-\arctan 0.01\omega_c''=-165^{\circ}$
\item 超前校正后: 
      $180^{\circ}+\phi(\omega_c'')+\arctan0.1\omega_c''-\arctan(\frac{T_b}{a}\omega_c'')= 57^{\circ}\geq\epsilon+\gamma'',(\epsilon=6^{\circ})$
\item 参数确定: $\frac{1}{T_a} =0.1\omega_c'',T_a = 0.5$
\item 校正网络: 
       $G_c=\frac{(1+0.5s)(1+0.1s)}{(1+2.5s)(1+0.02s)}$
\end{itemize}
\end{frame}
\begin{frame}
\frametitle{串联滞后-超前校正示例2}
\label{sec-2-4}

单位负反馈系统开环对数幅频特性曲线如图所示,其中虚线表示校正前的实线表示校正后的.要求
\begin{enumerate}
\item 确定校正装置类型,写出传递函数
\item 确定校正后系统稳定时的开环增益
\item 当开环增益  $k=1$  时,求校正后系统的相位裕度  $\gamma$ , 幅值裕度  $h$
\end{enumerate}
\begin{columns}
\begin{column}{0.5\textwidth}
\begin{itemize}

\item Bode图
\label{sec-2-4-1}%
\begin{tikzpicture}[scale=0.8]
\coordinate (o) at (0,0);
\coordinate (ox) at (3.5,0);
\draw[->] (-1.1,0) -- (ox);
\draw[->] (0,-1.1) -- (0,3.5);
\draw (o) node[above left] {$o$};
\draw (o) node[below left] {$0.1$};
%\draw [red,thick] plot coordinates {(0,0) (1,0) (2,-1)  (3,-1)};
\draw[red,thick] (3,0.5)--+(-1,1)--+(-4,2.5);
\draw[red,thick] (3,0.5)--+(0.5,-1);
\draw[dashed,purple,thick] (0,2.5)--(1,2.5)--(2,1.5);
\draw[dashed,blue] (1,0)--(1,2.5);
\draw[dashed,blue] (2,0)--(2,1.5);
\draw[dashed,blue] (3,0)--(3,0.5);
\draw (1,0) node[below] {$1$};
\draw (2,0) node[below] {$10$};
\draw (3,0) node[below] {$100$};
\draw[red,thick] (0.7,1.5) node[] {$-20dB/dec$};
\draw[red,thick] (2.7,0.7) node[above right] {$-40dB/dec$};
\draw[red,thick] (3.3,0) node[below right] {$-60dB/dec$};
\end{tikzpicture}

\end{itemize} % ends low level
\end{column}
\begin{column}{0.5\textwidth}
\begin{block}<2->{解:由图得}
\label{sec-2-4-2}

\begin{itemize}
\item 采用了串联滞后-超前校正
\item 校正前传递函数:  $G_1(s)=\frac{K(10s+1)}{s(s+1)^2(0.01s+1)}$
\item 校正后传递函数:  $G_2(s)=\frac{K}{s(0.1s+1)(0.01s+1)}$
\item 校正装置传递函数:  $G_2(s)=\frac{(s+1)^2}{(0.1s+1)(10s+1)}$
\end{itemize}
\end{block}
\end{column}
\end{columns}
\end{frame}
\begin{frame}
\frametitle{串联滞后-超前校正示例2(续):确定开环增益}
\label{sec-2-5}

校正前系统闭环特征方程:  

\[D(s)=s(0.1s+1)(0.01s+1)+K\]

Routh表

\[
\begin{matrix}
s^3  & 1 & 1000 \\
s^2  & 110 & 1000K \\
s^1  & \frac{110000-1000K}{110} & 0 \\
s^0  & 1000K 
\end{matrix}\]

得:  $1<K<110$ 
\end{frame}
\begin{frame}
\frametitle{串联滞后-超前校正示例2(续):稳定裕度计算}
\label{sec-2-6}

当  $K=1$  时, 
\begin{eqnarray*}
G_2(s) &=& \frac{1}{s(0.1s+1)(0.01s+1)} \\
L(\omega) & = & \begin{cases} 20\lg\frac{1}{\omega} & \omega<10 \\
                              20\lg\frac{1}{0.1\omega^2} & 10\leq \omega <100 \\
                              20\lg\frac{1}{0.001\omega^3} & \omega\geq 100  \end{cases}\\
\omega_c &=& 1 \\
\gamma &=& 180^{\circ}+\phi(\omega_c) \\
 &=& 83.7^{\circ} \\
\phi(\omega_x) &=& -\pi \\
\omega_x &=& 31.6\\
h &=& -20\lg\frac{1}{0.1\omega_x^2} \\
 &=& 10 dB
\end{eqnarray*}
\end{frame}
\begin{frame}
\frametitle{串联滞后-超前校正示例3}
\label{sec-2-7}

设待校正系统开环传递函数为: $G_o(s)=\frac{K}{s(\frac{1}{6}s+1)(\frac{1}{2}s+1)}$ ,设计校正装置满足以下指标:
\begin{itemize}
\item 最大指令速度 $180^{\circ}/s$ 时,位置滞后误差不超过 $1^{\circ}$
\item 相角裕度为 $45^{\circ}\pm 3^{\circ}$
\item 幅值裕度不低于 10dB
\item 动态过程调节时间不超过 3s
\end{itemize}

解:
\begin{itemize}
\item <2-> 由稳态性能指标得:  $K=180$
       \begin{align*}
        \omega_c' &=12.6 \\
        \gamma' &= -55.5^{\circ}\\
        h' &=-30dB
       \end{align*}
\end{itemize}
\end{frame}
\begin{frame}
\frametitle{串联滞后-超前校正示例3(续)}
\label{sec-2-8}

\begin{itemize}
\item $G_c(j\omega)=\frac{(1+jT_a\omega)(1+jT_b\omega)}{(1+jaT_a\omega)(1+jT_b\omega/a)}$
\item 分析-20dB/dec与-40dB/dec转折点,得 $T_b=\frac{1}{2}$
\item 由 $t_s,\gamma''$ 指标与 -20dB/dec的范围,得: $\omega_c'' \in [3.2,6]$ 取 $\omega_c'\approx 3.5$
\item $L'(\omega_c'')+20\lg T_b\omega_c''-20\lg a=0\to a=50$
\item $\gamma'': 180^{\circ}-90^{\circ}-\arctan \frac{\omega_c''}{6}+\arctan T_a\omega_c''-\arctan 50T_a\omega_c''-\arctan\frac{\omega_c''}{100}\to T_a\approx 0.78$
\item $G_c(S)=\frac{(1+1.28s)(1+0.5s)}{(1+64s)(1+0.01s)}$ ,验证: $\gamma''=45.5^{\circ},h''=27dB$ ,满足要求。
\end{itemize}
\end{frame}

\end{document}
