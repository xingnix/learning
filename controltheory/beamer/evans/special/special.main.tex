% !TeX root = special.beamer.tex
\DeclareMathOperator*{\argmin}{arg\,min}
\usetikzlibrary{shapes,arrows,positioning,decorations}
\usetikzlibrary{decorations.pathmorphing} 
\usetikzlibrary{decorations.shapes} 
\usetikzlibrary{decorations.fractals} 
\usetikzlibrary{spy} % L A T EX and plain T E X
\begin{document}
\newcommand{\vect}[1]{\boldsymbol{#1}}

\def\lecturename{自动控制原理}

\title{\insertlecture}

\author{邢超}

\institute
{
  西北工业大学航天学院
}

%\mode<presentation>{\subject{嵌入式系统}}

%  start a lecture  --------------------------
\lecture[]{线性系统的根轨迹法}{}
\subtitle{特殊的根轨迹}
\date{2014}


%\setbeamertemplate{background}{\pgfimage[width=\paperwidth,height=\paperheight]{image/flower}}
%\setbeamercovered{transparent}
%\mode<presentation>{\beamerdefaultoverlayspecification{<+->}}

\begin{frame}
  \maketitle
\end{frame}

\section{一阶系统}

\begin{frame}{$G(s)H(s)=\frac{K^*}{s}$}
\begin{eqnarray}
1+\frac{K^*}{s} &=& 0 \\
\frac{K^*}{s} &=& -1 \\
\angle{s}&=&(2k+1)\pi \\
s+K^* &=& 0 \\
s &=& -K^* 
\end{eqnarray}
\end{frame}

\begin{frame}{$G(s)H(s)=\frac{K^*}{s-c}$}
\begin{eqnarray}
1+\frac{K^*}{s-c} &=& 0 \\
\frac{K^*}{s-c} &=& -1 \\
\angle{s-c}&=&(2k+1)\pi \\
s-c+K^* &=& 0 \\
s &=& c-K^*
\end{eqnarray}
\end{frame}

\section{二阶系统}
\subsection{无开环零点}
\begin{frame}{$G(s)H(s)\frac{K^*}{(s+a)(s+b)}$}
\begin{center}

\begin{tikzpicture}[scale=3]
% The graphic
\draw[style=help lines,step=0.2cm] (-1.2,-1) grid (0,1);
%\draw (0,0) circle (1cm);
\begin{scope}
\draw[->] (-1.2,0) -- (0.2,0) node[right] {$ $};
\draw[->] (0,-1) -- (0,1) node[above] {$j$};
\foreach \x/\xtext in {-1, -0.5, 0}
\draw[xshift=\x cm] (0pt,1pt) -- (0pt,-1pt) node[below,fill=white] {$\xtext$};
\foreach \y/\ytext in {-1, -0.5, 0.5, 1}
\draw[yshift=\y cm] (1pt,0pt) -- (-1pt,0pt) node[left,fill=white]{$\ytext$};
\end{scope}

\mode<beamer>{
\foreach \i/\a/\b in {1/-1/0,2/-0.8/-0.2,3/-0.6/-0.4,4/-0.5/-0.5}
{
\draw<\i>[thick,-,red] (\a,0)--(\b,0);
\draw<\i>[thick,<->,red] (-0.5,-1.3)--(-0.5,1.3);
\draw<\i> (\a,0) node[red]{$\times$};
\draw<\i> (\b,0) node[red]{$\times$};
\draw<\i> (0.7,0.3) node[green,fill=white]{$a=\a$};
\draw<\i> (0.7,0.1) node[green,fill=white]{$b=\b$};
}
\foreach \i/\a in {5/0.2,6/0.4,7/0.6,8/0.8,9/1}
{
\draw<\i>[->,red,thick] (-0.5,\a)--(-0.5,1.3);
\draw<\i>[->,red,thick] (-0.5,-\a)--(-0.5,-1.3);
\draw<\i> (-0.5,\a) node[red]{$\times$};
\draw<\i> (-0.5,-\a) node[red]{$\times$};
\draw<\i> (0.7,0.3) node[green,fill=white]{$a=-0.5+\a j$};
\draw<\i> (0.7,0.1) node[green,fill=white]{$b=-0.5-\a j$};
}
}
\mode<article>{
\draw[thick,-,red] (-1,0)--(0,0);
\draw[thick,<->,red] (-0.5,-1.3)--(-0.5,1.3);
\draw (-1,0) node[red]{$\times$};
\draw (0,0) node[red]{$\times$};
\draw (0.7,0.3) node[green,fill=white]{$a=-1$};
\draw (0.7,0.1) node[green,fill=white]{$b=0$};
}
\end{tikzpicture}
\end{center}
\end{frame}

\subsection{有开环零点}
\begin{frame}{$G(s)H(s)=\frac{K^*(s+c)}{(s+a)(s+b)}$}
\begin{center}
\begin{tikzpicture}[scale=1.2]
% The graphic
% \draw[style=help lines,step=0.2cm] (-3,-2) grid (4.2,2);
\begin{scope}
\draw[->] (-3,0) -- (4.2,0) node[right] {$ $};
\draw[->] (0,-2.1) -- (0,2.1) node[above] {$j$};
% \foreach \x/\xtext in {-2,  0, 1,2,4}
% \draw[xshift=\x cm] (0pt,1pt) -- (0pt,-1pt) node[below,fill=white] {$\xtext$};
% \foreach \y/\ytext in {-2, -1, 1, 2}
% \draw[yshift=\y cm] (1pt,0pt) -- (-1pt,0pt) node[left,fill=white]{$\ytext$};
\end{scope}

\mode<beamer>{
\foreach \i/\a/\b in {1/1/4,2/1.5/3,3/2/2}
{
\draw<\i>[thick,-,red] (\a,0)--(\b,0);
\draw<\i>[thick,<->,red] (-3,0)--(0,0);
\draw<\i>[thick,red] (0,0) circle (2cm);
\draw<\i> (\a,0) node[red,thick]{$\times$};
\draw<\i> (\b,0) node[red,thick]{$\times$};
\draw<\i> (0,0) node[red,thick]{$\circ$};
\draw<\i> (\a,0) node[below  left,green]{$a$};
\draw<\i> (\b,0) node[below  right,green]{$b$};
}
\foreach \i/\a in {4/30,5/60,6/90,7/145,8/180}
{
\draw<\i>[thick,<->,red] (-3,0)--(0,0);
\draw<\i>[-,red,thick] (\a:2cm)  arc (\a:360-\a:2cm);
\draw<\i> (\a:2cm) node[red]{$\times$};
\draw<\i> (-\a:2cm) node[red]{$\times$};
\draw<\i> (0,0) node[red,thick]{$o$};
\draw<\i> (\a:2cm) node[above  right,green]{$a$};
\draw<\i> (-\a:2cm) node[below  right,green]{$b$};
}
}
\mode<article>{
\draw[thick,-,red] (1,0)--(4,0);
\draw[thick,<->,red] (-3,0)--(0,0);
\draw[thick,red] (0,0) circle (2cm);
\draw (1,0) node[red,thick]{$\times$};
\draw (4,0) node[red,thick]{$\times$};
\draw (0,0) node[red,thick]{$\circ$};
\draw (1,0) node[below  left,green]{$a$};
\draw (4,0) node[below  right,green]{$b$};
}
\end{tikzpicture}
\end{center}
\end{frame}

\begin{frame}{根轨迹为圆的证明(重极点)}
\begin{center}
\begin{tikzpicture}[scale=1]
% The graphic
% \draw[style=help lines,step=0.2cm] (-3,-2) grid (4.2,2);
\begin{scope}
\draw[->] (-3,0) -- (4.2,0) node[right] {$ $};
\draw[->] (0,-2.1) -- (0,2.1) node[above] {$j$};
% \foreach \x/\xtext in {-2,  0, 1,2,4}
% \draw[xshift=\x cm] (0pt,1pt) -- (0pt,-1pt) node[below,fill=white] {$\xtext$};
% \foreach \y/\ytext in {-2, -1, 1, 2}
% \draw[yshift=\y cm] (1pt,0pt) -- (-1pt,0pt) node[left,fill=white]{$\ytext$};
\end{scope}

\draw[thick,red] (0,0) circle (2cm);
\draw (0:2cm) node[red,thick]{$\times$};
\draw (60:2cm) node[red, above right,thick]{$S$};
\draw (0,0) node[red,thick]{$o$};
\draw (0,0) node[red,below,thick]{$O$};
\draw[->] (0,0)--(60:2cm);
\draw[->] (2,0)--(60:2cm);
\draw<2->[->,dashed] (2,0)--(2,2);
\draw<2-> (2cm,2cm) node[red,right,thick]{$Y$};
\draw (4.2,0) node[red, right,thick]{$X$};
\draw (2,0) node[red, below ,thick,fill=white]{$A$};
\end{tikzpicture}
\end{center}
\begin{align*}
\angle{(s-a)}+\angle{(s-b)}-\angle{(s-O)} &= 2\angle{SAX}-\angle{SOX} \\
\uncover<2->{ \angle{SAX}} &\uncover<2->{=\angle{SAY}+\frac{\pi}{2} \\}
\uncover<2->{ 2\angle{SAY}} & \uncover<2->{=\angle{SOX}}
\end{align*}
\end{frame}

\begin{frame}{根轨迹为圆的证明(共轭极点)}
\begin{center}
\begin{tikzpicture}[scale=1]
% The graphic
% \draw[style=help lines,step=0.2cm] (-3,-2) grid (4.2,2);
\begin{scope}
\draw[->] (-3,0) -- (4.2,0) node[right] {$ $};
\draw[->] (0,-2.1) -- (0,2.1) node[above] {$j$};
% \foreach \x/\xtext in {-2,  0, 1,2,4}
% \draw[xshift=\x cm] (0pt,1pt) -- (0pt,-1pt) node[below,fill=white] {$\xtext$};
% \foreach \y/\ytext in {-2, -1, 1, 2}
% \draw[yshift=\y cm] (1pt,0pt) -- (-1pt,0pt) node[left,fill=white]{$\ytext$};
\end{scope}

\draw[red,thick] (60:2cm)  arc (60:300:2cm);
\draw (60:2cm) node[red,thick]{$\times$};
\draw (-60:2cm) node[red,thick]{$\times$};
\draw (100:2cm) node[red, above left,thick]{$S$};
\draw (0,0) node[red]{$o$};
\draw (0,0) node[red,below,thick]{$O$};
\draw[->] (60:2cm)--(100:2cm);
\draw[->] (-60:2cm)--(100:2cm);
\draw[dashed] (-60:2cm)--+(0,4.5cm);
\draw[dashed] (0,0)--(60:2cm);
\draw[dashed] (60:2cm)--+(2cm,0);
\draw[dashed] (-60:2cm)--+(2cm,0);
\draw (60:2cm)+(2cm,0) node[red,right]{$A'$};
\draw (-60:2cm)+(2cm,0) node[red,right]{$B'$};
\draw[->] (0,0)--(100:2cm);
\draw (-60:2cm)+(0,4.5cm) node[red,above right,thick]{$Y$};
\draw (4.2,0) node[red, right,thick]{$X$};
\draw (60:2cm) node[red, above right ,thick,fill=white]{$A$};
\draw (-60:2cm) node[red, below right ,thick,fill=white]{$B$};
\end{tikzpicture}
\end{center}
\begin{eqnarray*}
\angle{SAY}&=&\angle{SBY}+\angle{BSA} \\
\angle{BSA}&=&\angle{AOX}\\
2\angle{SBA}&=&\angle{SOA}
\end{eqnarray*}
\end{frame}

\section{开环重极点}
\subsection{原点为极点}
\begin{frame}{$G(s)H(s)=\frac{K^*}{s^n}$}
\begin{eqnarray}
1+K^{\ast}\frac{1}{s^n} &=& 0  \\
K^{\ast}\frac{1}{s^n} &=& -1  \\
\angle{s^n} &=& (2k+1)\pi \\
n\angle{s} &=& (2k+1)\pi \\
\angle{s} &=& \frac{(2k+1)\pi}{n} \\
s^n &=& -K^* \\
s &=& \sqrt[n]{K^*}e^{j\frac{(2k+1)\pi}{n}}
\end{eqnarray}
\end{frame}

\begin{frame}{示例$G(s)H(s)=\frac{K^*}{s^3}$}
\begin{center}
\begin{tikzpicture}[scale=1]
% The graphic
% \draw[style=help lines,step=0.2cm] (-3,-2) grid (4.2,2);
\begin{scope}
\draw[->] (-3,0) -- (4.2,0) node[right] {$ $};
\draw[->] (0,-2.1) -- (0,2.1) node[above] {$j$};
% \foreach \x/\xtext in {-2,  0, 1,2,4}
% \draw[xshift=\x cm] (0pt,1pt) -- (0pt,-1pt) node[below,fill=white] {$\xtext$};
% \foreach \y/\ytext in {-2, -1, 1, 2}
% \draw[yshift=\y cm] (1pt,0pt) -- (-1pt,0pt) node[left,fill=white]{$\ytext$};
\end{scope}

\draw[red,thick,->] (0:0)--(60:3cm);
\draw[red,thick,->] (0:0)--(180:3cm);
\draw[red,thick,->] (0:0)--(-60:3cm);
\draw (0:0cm) node[red,thick]{$\times$};
\end{tikzpicture}
\end{center}
\end{frame}

\begin{frame}{示例$G(s)H(s)=\frac{K^*}{s^3}, K^*\in (-\infty,0)$}
\begin{center}
\begin{tikzpicture}[scale=1]
\begin{scope}
\draw[->] (-3,0) -- (4.2,0) node[right] {$ $};
\draw[->] (0,-2.1) -- (0,2.1) node[above] {$j$};
\end{scope}

\draw[green,thick,->] (0:0)--(0:3cm);
\draw[green,thick,->] (0:0)--(120:3cm);
\draw[green,thick,->] (0:0)--(-120:3cm);
\draw (0:0cm) node[red,thick]{$\times$};
\end{tikzpicture}
\end{center}
\end{frame}

\subsection{极点平移}
\begin{frame}{$G(s)=\frac{K^*}{(s-c)^n}$}
\begin{eqnarray}
1+K^{\ast}\frac{1}{(s-c)^n} &=& 0  \\
K^{\ast}\frac{1}{(s-c)^n} &=& -1  \\
\angle{(s-c)^n} &=& (2k+1)\pi \\
n\angle{(s-c)} &=& (2k+1)\pi \\
\angle{(s-c)} &=& \frac{(2k+1)\pi}{n} \\
(s-c)^n &=& -K^* \\
s &=& c+\sqrt[n]{K^*}e^{j\frac{(2k+1)\pi}{n}}
\end{eqnarray}
\end{frame}

\begin{frame}{示例$G(s)H(s)=\frac{K^*}{(s-1)^3}$}
\begin{center}
\begin{tikzpicture}[scale=1]
% The graphic
% \draw[style=help lines,step=0.2cm] (-3,-2) grid (4.2,2);
\begin{scope}
\draw[->] (-3,0) -- (4.2,0) node[right] {$ $};
\draw[->] (0,-2.1) -- (0,2.1) node[above] {$j$};
% \foreach \x/\xtext in {-2,  0, 1,2,4}
% \draw[xshift=\x cm] (0pt,1pt) -- (0pt,-1pt) node[below,fill=white] {$\xtext$};
% \foreach \y/\ytext in {-2, -1, 1, 2}
% \draw[yshift=\y cm] (1pt,0pt) -- (-1pt,0pt) node[left,fill=white]{$\ytext$};
\end{scope}

\draw[red,thick,->] (0:1cm)-- +(60:3cm);
\draw[red,thick,->] (0:1cm)-- +(180:3cm);
\draw[red,thick,->] (0:1cm)-- +(-60:3cm);
\draw (0:1cm) node[red,thick]{$\times$};
\draw (0:1cm) node[red,below]{$1$};
\end{tikzpicture}
\end{center}
\end{frame}

\section{$G(s+a)H(s+a)$}
\begin{frame}{$G(s+a)H(s+a)$的根轨迹}
\begin{eqnarray*}
s' &=& s+a \\
1+K^* G(s')H(s') &=& 0 \\
s' &=& f(K^*) \\
s &=& f(K^*)-a
\end{eqnarray*}
\end{frame}

\section{$1+(K^*+a)G(s)H(s)$}

\begin{frame}{$1+(K^*+a)G(s)H(s)$的根轨迹}
\begin{eqnarray*}
K' &=& K^*+a \\
s &=& f(K') \\
s &=& f(K^*+a)
\end{eqnarray*}
\end{frame}


\begin{frame}{$\frac{K^*}{s^3+1}$的根轨迹}
\begin{center}
\begin{tikzpicture}[scale=1]
\begin{scope}
\draw[->] (-3,0) -- (3,0) node[right] {$ $};
\draw[->] (0,-2.1) -- (0,2.1) node[above] {$j$};
\end{scope}

\draw[red,thick,->] (60:1cm)--(60:3cm);
\draw[red,thick,->] (180:1cm)-- (180:3cm);
\draw[red,thick,->] (-60:1cm)-- (-60:3cm);
\draw (60:1cm) node[red,thick]{$\times$};
\draw (180:1cm) node[red,thick]{$\times$};
\draw (-60:1cm) node[red,thick]{$\times$};
\end{tikzpicture}
\end{center}
\end{frame}

\begin{frame}{ 当 $K^*\in(1,\infty)$ 时, $\frac{K^*}{s^3}$ 的根轨迹}
\begin{align*}
\frac{K^*}{s^3+1}&=-1 \\
K^* &=-s^3-1 \\
K^*+1 &=-s^3 \\
\frac{K^*+1}{s^3} &=-1 \qquad (K^*\in (0,+\infty))\\
\frac{K'}{s^3} &=-1 \qquad (K'\in (1,+\infty))\\
s &=\sqrt[3]{K'}e^{j\frac{(2k+1)\pi}{3}} \qquad (K'\in (1,+\infty))\\
\end{align*}
\end{frame}

\begin{frame}{$\frac{K^*}{s^3-1}$的根轨迹}
\begin{center}
\begin{tikzpicture}[scale=1]
\begin{scope}
\draw[->] (-3,0) -- (3,0) node[right] {$ $};
\draw[->] (0,-2.1) -- (0,2.1) node[above] {$j$};
\end{scope}

\draw[green,thick] (0:0cm)--(0:1cm);
\draw[green,thick] (0:0cm)-- (120:1cm);
\draw[green,thick] (0:0cm)-- (-120:1cm);
\draw[red,thick,->] (0:0cm)--(60:3cm);
\draw[red,thick,->] (0:0cm)-- (180:3cm);
\draw[red,thick,->] (0:0cm)-- (-60:3cm);
\draw (0:1cm) node[red,thick]{$\times$};
\draw (120:1cm) node[red,thick]{$\times$};
\draw (-120:1cm) node[red,thick]{$\times$};
\end{tikzpicture}
\end{center}
\end{frame}

\begin{frame}{当 $K^*\in(-1,\infty)$ 时 $\frac{K^*}{s^3}$ 的根轨迹}
\begin{align*}
\frac{K^*}{s^3-1}&=-1 \\
K^* &=-s^3+1 \\
K^* -1 &=-s^3 \\
\frac{K^*-1}{s^3} &=-1 \qquad (K^*\in (0,+\infty))\\
\frac{K'}{s^3} &=-1 \qquad (K'\in (-1,+\infty))\\
s &=\begin{cases}
\sqrt[3]{-K'}e^{j\frac{2k\pi}{3}} & \qquad (K'\in (-1,0))\\
\sqrt[3]{K'} e^{j\frac{(2k+1)\pi}{3}}  & \qquad (K'\in [0,+\infty))\\
\end{cases}
\end{align*}
\end{frame}

\end{document}
	

%%% Local Variables:
%%% TeX-master: "sepcial.beamer"
%%% End:
