% Created 2014-10-10 星期五 12:37
\documentclass{beamer}
\usepackage[utf8]{inputenc}
\usepackage[T1]{fontenc}
\usepackage{fixltx2e}
\usepackage{graphicx}
\usepackage{longtable}
\usepackage{float}
\usepackage{wrapfig}
\usepackage{soul}
\usepackage{textcomp}
\usepackage{marvosym}
\usepackage{wasysym}
\usepackage{latexsym}
\usepackage{amssymb}
\usepackage{hyperref}
\tolerance=1000
\usepackage{etex}
\usepackage{amsmath}
\usepackage{pstricks}
\PassOptionsToPackage{usenames}{color}
\usepackage{pgfplots}
\usepackage{tikz}
\usepackage[europeanresistors,americaninductors]{circuitikz}
\usepackage{colortbl}
\usepackage{yfonts}
\usetikzlibrary{shapes,arrows}
\usetikzlibrary{positioning}
\usetikzlibrary{arrows,shapes}
\usetikzlibrary{intersections}
\usetikzlibrary{calc,patterns,decorations.pathmorphing,decorations.markings}
\usepackage[BoldFont,SlantFont,CJKchecksingle]{xeCJK}
\setCJKmainfont[BoldFont=Evermore Hei]{Evermore Kai}
\setCJKmonofont{Evermore Kai}
\usepackage{pst-node}
\usepackage{pst-plot}
\psset{unit=5mm}
\newcommand*\diff{\mathop{}\!\mathrm{d}}
\allowdisplaybreaks
\mode<beamer>{\usetheme{Frankfurt}}
\mode<beamer>{\usecolortheme{dove}}
\mode<article>{\hypersetup{colorlinks=true,pdfborder={0 0 0}}}
\mode<beamer>{\AtBeginSection[]{\begin{frame}<beamer>\frametitle{Topic}\tableofcontents[currentsection]\end{frame}}}
\setbeamercovered{transparent}
\subtitle{示例}
\providecommand{\alert}[1]{\textbf{#1}}

\title{线性系统时域分析法}
\author{}
\date{}
\hypersetup{
  pdfkeywords={},
  pdfsubject={},
  pdfcreator={Emacs Org-mode version 7.9.3f}}

\begin{document}

\maketitle

\begin{frame}
\frametitle{Outline}
\setcounter{tocdepth}{3}
\tableofcontents
\end{frame}














\section{速度反馈}
\label{sec-1}
\begin{frame}
\frametitle{速度反馈分析}
\label{sec-1-1}
\begin{columns}
\begin{column}{0.5\textwidth}
%% 结构图
\label{sec-1-1-1}

\begin{psmatrix} [rowsep=0.4,colsep=0.5]
%         1    2   3  4          5    6    7
%                E(s)     .------------------------.
%         R-->o------>o-->|omegan/s(s+2\xi\omega_n)|--+--> C
%           _ ^     _ ^   '------------------------'  |
%             |       |                               |  
%             |       '------K_t s--------------------+
%             |                                       |
%             '---------------------------------------'
%
% 1                        2                          3        
$R(s)$  & \pscirclebox[framesep=-0.2em]{$\times$} & {\hskip 1em}    & \pscirclebox[framesep=-0.2em]{$\times$} & %
\psframebox{$\frac{\omega_n^2}{s(s+2\xi\omega_n)}$}   & \   & $C(s)$ \\
  &   &     &  & \psframebox{$ K_t s $} &  \ &    \\
\\
%link
\ncline{->}{1,1}{1,2}
\ncline{->}{1,2}{1,4}
\naput{$E(s)$}
\ncline{->}{1,4}{1,5}
\ncline{->}{1,5}{1,7}
\ncline{2,6}{2,5}
\ncangle[angleA=180,angleB=-90,armA=0.5em,armB=1em]{->}{2,5}{1,4}
\naput[npos=2.3]{$-$}
\ncangles[angleA=180,angleB=-90,armA=0em,armB=3.5em]{->}{1,6}{1,2}
\naput[npos=3.6]{$-$}
\end{psmatrix}
\end{column}
\begin{column}{0.5\textwidth}
\begin{block}<2->{分析:}
\label{sec-1-1-2}


\begin{align*}
\Phi(s) &= \frac{\omega_n^2}{s^2+2\xi_t\omega_n s+\omega_n^2} \\
\xi_t &= \xi+\frac{1}{2}K_t\omega_n 
\end{align*}
\end{block}
\end{column}
\end{columns}
\end{frame}
\begin{frame}
\frametitle{速度反馈示例}
\label{sec-1-2}
\begin{columns}
\begin{column}{0.5\textwidth}
%% 结构图
\label{sec-1-2-1}

\begin{psmatrix}[rowsep=0.4,colsep=0.5]
%         1    2   3  4          5    6    7
%                E(s)     .--------.
%         R-->o------>o-->|k/s(s+1)|--+--> C
%           _ ^     _ ^   '--------'  |
%             |       |               |  
%             |       '------As-------+
%             |                       |
%             '-----------------------'
%
% 1                        2                          3        
$R(s)$  & \pscirclebox[framesep=-0.2em]{$\times$} & {\hskip 1em}    & \pscirclebox[framesep=-0.2em]{$\times$} & %
\psframebox{$\frac{K}{s(s+1)}$}   & \   & $C(s)$ \\
  &   &     &  & \psframebox{$ As $} &  \ &    \\
\\
%link
\ncline{->}{1,1}{1,2}
\ncline{->}{1,2}{1,4}
\naput{$E(s)$}
\ncline{->}{1,4}{1,5}
\ncline{->}{1,5}{1,7}
\ncline{2,6}{2,5}
\ncangle[angleA=180,angleB=-90,armA=0.5em,armB=1em]{->}{2,5}{1,4}
\naput[npos=2.3]{$-$}
\ncangles[angleA=180,angleB=-90,armA=0em,armB=3.5em]{->}{1,6}{1,2}
\naput[npos=3.6]{$-$}
\end{psmatrix}

As为系统的速度反馈,求解:
\begin{itemize}
\item $A=\{0.8,0\}$ 时系统稳态误差及动态品质指标( $K=1$ ,求 $e_{ss}$ 时 $R(s)=\frac{1}{s^2}$ )
\item 若要求系统的 $\sigma\%=20\%$ , $t_s\leq 1s$ ,确定 $A,K$ 的值
\end{itemize}
\end{column}
\begin{column}{0.5\textwidth}
\begin{block}<2->{解:}
\label{sec-1-2-2}


\begin{align*}
G(s)&= \frac{K}{s(s+1)+KAs}\\
    &=  \frac{K}{s^2+(KA+1)s} \\
\Phi(s) &= \frac{K}{s^2+(KA+1)s+K} 
\end{align*}
系统稳定,为I型系统.
\end{block}
\end{column}
\end{columns}
\end{frame}
\begin{frame}
\frametitle{(续)计算稳态误差:}
\label{sec-1-3}

 $R(s)=\frac{1}{s^2}$ 时,
\begin{align*}
e_{ss} &= \frac{1}{Kv} \\
K_v &= \lim_{s\rightarrow 0}sG(s) \\
    &= \lim_{s\rightarrow 0}\frac{K}{s+KA+1} \\
    &= \frac{K}{KA+1} \\
\end{align*}

\begin{itemize}
\item $K=1,A=0.8$ 时, $e_{ss}=\frac{KA+1}{K}=1.8$
\item $K=1,A=0$ 时, $e_{ss}=\frac{KA+1}{K}=1$
\end{itemize}
\end{frame}
\begin{frame}
\frametitle{(续):计算动态品质:}
\label{sec-1-4}


\begin{eqnarray*}
\Phi(s) &=& \frac{\omega_n^2}{s^2+2\xi\omega_n s+\omega_n^2} \\
        &=& \frac{K}{s^2+(KA+1)s+K} \\
\sigma\% &=& e^{-\frac{\xi\pi}{\sqrt{1-\xi^2}}} \\
t_s &=& \frac{3.5}{\xi\omega_n}
\end{eqnarray*}

\begin{itemize}
\item <2-> $K=1,A=0$ 时, $\omega_n=1,\xi=0.5,\sigma\%=16.3\%,t_s=7$
\item <3-> $K=1,A=0.8$ 时, $\omega_n=1,\xi=0.9,\sigma\%=0.15\%,t_s=3.89$
\end{itemize}
\end{frame}
\begin{frame}
\frametitle{(续)确定 $A,K$ 值}
\label{sec-1-5}



\begin{eqnarray*}
e^{\frac{-\xi\pi}{\sqrt{1-\xi^2}}} &=&20\% \\
\frac{3.5}{\xi\omega_n} &=& 1
\end{eqnarray*}

得: $\xi=0.456,\omega_n=7.68$

\begin{eqnarray*}
2\xi\omega_n & = & AK+1\\
K &=& \omega_n^2
\end{eqnarray*}

得: $A=0.102,K=58.9$
\end{frame}
\section{比例-微分控制}
\label{sec-2}
\begin{frame}
\frametitle{传递函数}
\label{sec-2-1}

\begin{psmatrix}[rowsep=0.4,colsep=0.5]
%         1    2   3  4      5                6          7    8
%                E(s)                          
%         R-->o---+--1------>o--> wn2/s(s+2\x\wn2)--+--> C
%           _ ^   |           ^                      |
%             |   !--Tds------!                      |  
%             |                                      |
%             '--------------------------------------'
%
% 1                        2                          3        
$R(s)$  & \pscirclebox[framesep=-0.2em]{$\times$} & {\hskip 1em} & \psframebox{$1$} 
 & \pscirclebox[framesep=-0.2em]{$\times$} & %
\psframebox{$\frac{\omega_n^2}{s(s+2\xi\omega_n)}$}   & \   & $C(s)$ \\
    &     &  & \psframebox{$ T_d s $} &  \ &   \\
\\
%link
\ncline{->}{1,1}{1,2}
\ncline{->}{1,2}{1,4}
\naput{$E(s)$}
\ncline{->}{1,4}{1,5}
\ncline{->}{1,5}{1,6}
\ncline{->}{1,6}{1,8}
\ncangle[angleA=0,angleB=180,armA=0.5em,armB=1em]{->}{1,3}{2,4}
\ncangle[angleA=0,angleB=-90,armA=0.5em,armB=1em]{->}{2,4}{1,5}
\naput[npos=2.3]{$+$}
\ncangles[angleA=180,angleB=-90,armA=0em,armB=3.5em]{->}{1,7}{1,2}
\naput[npos=3.6]{$-$}
\end{psmatrix}

比较原系统与PD控制系统的稳态性能与动态性能

解:

\begin{eqnarray*}
G(s) & = &\frac{\omega_n^2(1+T_d s)}{s^2+2\xi\omega_n s} \\
\Phi(s) &=&\frac{\omega_n^2(1+T_d s)}{s^2+(2\xi+T_d\omega_n)\omega_n s+\omega_n^2}
\end{eqnarray*}
\end{frame}
\begin{frame}
\frametitle{稳态误差}
\label{sec-2-2}

系统稳定,为I型系统.
\begin{eqnarray*}
e_{ss} & =& \frac{1}{K_v} \\
K_v &=& \lim_{s\rightarrow 0}sG(s) \\
   &=& \lim_{s\rightarrow 0}\frac{\omega_n^2(1+T_d s)}{s+2\xi\omega_n} \\
   &=&\frac{\omega_n}{2\xi}
\end{eqnarray*}

$e_{ss}$ 与 $T_d$ 无关.
\end{frame}
\begin{frame}
\frametitle{动态性能分析}
\label{sec-2-3}

考虑如下三个系统:
\begin{eqnarray*}
\Phi_1(s) &=&\frac{\omega_n^2}{s^2+2\xi\omega_n s+\omega_n^2}\\
\Phi_2(s) &=&\frac{\omega_n^2(1+T_d s) }{s^2+(2\xi+T_d\omega_n)\omega_n s+\omega_n^2}\\
\Phi_3(s) &=&\frac{\omega_n^2}{s^2+(2\xi+T_d\omega_n)\omega_n s+\omega_n^2}
\end{eqnarray*}           

\begin{itemize}
\item <2->与系统1相比,系统3的阻尼比较大,两者的自然频率相同.
\item <3->因为:
        \begin{eqnarray*}
        \Phi_2(s) & = & (1+T_d s)\Phi_3(s) \\
        c_2(t) &=& c_3(t)+T_d\frac{dc_3(t)}{dt}
        \end{eqnarray*}
       由于存在微分作用,因此对高频噪声有放大作用.
\end{itemize}
\end{frame}
\begin{frame}
\frametitle{动态性能计算:阶跃响应}
\label{sec-2-4}

\begin{align*}
\Phi &=  \frac{\omega_n^2}{z}\left(\frac{s+z}{s^2+2\xi_d\omega_n s+\omega_n^2}\right)\qquad (\xi_d=\xi+\frac{\omega_n}{2z},z=\frac{1}{T_d})\\
C(s) &= \frac{\omega_n^2}{s(s^2+2\xi_d\omega_n s+\omega_n^2)}+\frac{1}{z}\frac{s\omega_n^2}{s(s^2+2\xi_d\omega_n s+\omega_n^2)}\\
h(t) &= 1+re^{-\xi_d\omega_n t}\sin(\omega_n\sqrt{1-\xi_d^2}t+\psi) \\
r &= \frac{\sqrt{z^2-2\xi_d\omega_n z+\omega_n^2}}{z\sqrt{1-\xi_d^2}}\\
\psi &= -\pi+\arctan\frac{\omega_n\sqrt{1-\xi_d^2}}{z-\xi_d\omega_n}+\arctan \frac{\sqrt{1-\xi_d^2}}{\xi_d}
\end{align*}
\end{frame}
\begin{frame}
\frametitle{动态性能计算:峰值时间}
\label{sec-2-5}


\begin{eqnarray*}
\frac{\diff h(t)}{\diff t} & =& 0\\
\tan(\omega_n\sqrt{1-\xi_d^2}t_p+\psi) &=& \frac{\sqrt{1-\xi_d^2}}{\xi_d}\\
t_p &=& \frac{\beta_d-\psi}{\omega_n\sqrt{1-\xi_d^2}}\qquad \left(\beta_d=\arctan \frac{\sqrt{1-\xi_d^2}}{\xi_d}\right)\\
\end{eqnarray*}
\end{frame}
\begin{frame}
\frametitle{动态性能计算:超调量}
\label{sec-2-6}

\begin{eqnarray*}
h(t_p) & =& 1+re^{-\xi_d\omega_n t_p}\sin(\omega_n\sqrt{1-\xi_d^2}t_p+\psi) \\
 &=& 1+re^{-\xi_d\omega_n t_p}\sin(\beta_d) \\
\sigma\% &=& r\sqrt{1-\xi_d^2}e^{-\xi_d\omega_n t_p} \times 100\%
\end{eqnarray*}
\end{frame}
\begin{frame}
\frametitle{动态性能计算:调节时间}
\label{sec-2-7}

\begin{align*}
\Delta  &= |h(t)-h(\infty)|\\
&= |re^{-\xi_d\omega_n t}\sin(\omega_n\sqrt{1-\xi_d^2}t+\psi) |\\
&\leqslant re^{-\xi_d\omega_n t} \\
t_s &= \frac{3+\ln r}{\xi_d \omega_n}\qquad (\Delta=0.05)
\end{align*}
\end{frame}
\begin{frame}
\frametitle{比例-微分控制示例}
\label{sec-2-8}

设单位反馈系统开环传递函数: $G(s)=\frac{K(T_d s+1)}{s(1.67s+1)}$
若要求单位斜坡函数输入时 $e_{ss}\leq 0.2, \xi_d=0.5$ 求
$K,T_d$ 的值,并估算系统在阶跃函数作用下的动态性能。

解:
\begin{columns}
\begin{column}{0.3\textwidth}
\begin{block}{求 $K$ :}
\label{sec-2-8-1}

\begin{eqnarray*}
e_{ss} & =& \frac{1}{K} \leq 0.2\\
 K &=& 5 
\end{eqnarray*}
\end{block}
\end{column}
\begin{column}{0.4\textwidth}
\begin{block}{$T_d=0$}
\label{sec-2-8-2}

\begin{eqnarray*}
s^2+0.6s+3 & =& 0\\
\xi &=& 0.173 \\
\omega_n &=& 1.732 \\
t_p &=& 1.84s \\
\sigma\% &=& 57.6\% \\
t_s &=& 11.7s
\end{eqnarray*}
\end{block}
\end{column}
\begin{column}{0.3\textwidth}
\begin{block}{$T_d\neq0$}
\label{sec-2-8-3}

\begin{eqnarray*}
\xi_d & =& 0.5\\
T_d &=& \frac{2(\xi_d-\xi)}{\omega_n}\\
t_p &=& 1.63s \\
\sigma\% &=& 22\% \\
t_s &=& 3.49s
\end{eqnarray*}
\end{block}
\end{column}
\end{columns}
\end{frame}
\section{比例-积分控制}
\label{sec-3}
\begin{frame}
\frametitle{比例-积分控制分析}
\label{sec-3-1}

\begin{psmatrix}[rowsep=0.4,colsep=0.5]
%         1    2   3  4      5                6          7    8
%                E(s)            |N(s)              
%         R-->o--K_1(1+1/T_i s)->o-> K_2/s(T_2 s+1)---+--> C
%           _ ^                                       |
%             |                                       |  
%             |                                       |
%             '---------------------------------------'
%
% 1                        2                          3        
 & & & & N(S)\\
$R(s)$  & \pscirclebox[framesep=-0.2em]{$\times$} & {\hskip 1em} 
& \psframebox{$K_1(1+\frac{1}{T_i s})$} 
 & \pscirclebox[framesep=-0.2em]{$\times$} & %
\psframebox{$\frac{K_2}{s(T_2 s+1)}$}   & \   & $C(s)$ \\
\\
%link
\ncline{->}{2,1}{2,2}
\ncline{->}{2,2}{2,4}
\naput{$E(s)$}
\ncline{->}{2,4}{2,5}
\ncline{->}{2,5}{2,6}
\ncline{->}{2,6}{2,8}
\ncline{->}{1,5}{2,5}
\ncangles[angleA=180,angleB=-90,armA=0em,armB=3.5em]{->}{2,7}{2,2}
\naput[npos=3.6]{$-$}
\end{psmatrix}


\begin{align*}
E_n(s) &=-\frac{K_2 T_i s}{T_i T_2 s^3+T_i s^2+K_1K_2T_i s+K_1K_2}N(s)\\
\end{align*}
系统稳定时:
\begin{align*}
e_{ssn} &=\lim_{s\to 0}sE_n(s) = 0 & (N(s)=\frac{n_0}{s})\\
e_{ssn} &=\lim_{s\to 0}sE_n(s) = -\frac{n_1T_i}{K_1} & (N(s)=\frac{n_1}{s^2})
\end{align*}
\end{frame}
\begin{frame}
\frametitle{比例-积分控制示例}
\label{sec-3-2}

\begin{psmatrix}[rowsep=0.4,colsep=0.5]
%         1    2   3  4      5                6          7    8
%                E(s)                          
%         R-->o---+--1------>o--> wn2/s(s+2\x\wn2)--+--> C
%           _ ^   |          ^                      |
%             |   !--K_1/s---!                      |  
%             |                                     |
%             '-------------------------------------'
%
% 1                        2                          3        
$R(s)$  & \pscirclebox[framesep=-0.2em]{$\times$} & {\hskip 1em} & \psframebox{$1$} 
 & \pscirclebox[framesep=-0.2em]{$\times$} & %
\psframebox{$\frac{\omega_n^2}{s(s+2\xi\omega_n)}$}   & \   & $C(s)$ \\
    &     &  & \psframebox{$ \frac{K_1}{s} $} &  &   \\
\\
%link
\ncline{->}{1,1}{1,2}
\ncline{->}{1,2}{1,4}
\naput{$E(s)$}
\ncline{->}{1,4}{1,5}
\ncline{->}{1,5}{1,6}
\ncline{->}{1,6}{1,8}
\ncangle[angleA=0,angleB=180,armA=0.5em,armB=1em]{->}{1,3}{2,4}
\ncangle[angleA=0,angleB=-90,armA=0.5em,armB=1em]{->}{2,4}{1,5}
\naput[npos=2.3]{$+$}
\ncangles[angleA=180,angleB=-90,armA=0em,armB=4.5em]{->}{1,7}{1,2}
\naput[npos=3.6]{$-$}
\end{psmatrix}

已知参数 $\xi=0.2,\omega_n=86.6$ 求:
\begin{itemize}
\item 使闭环系统稳定的 $K_1$ 范围
\item 使闭环系统极点实部全部小于-1的 $K_1$ 范围
\end{itemize}

解:

\begin{align*}
\Phi &=\frac{\omega_n(s+K_1)}{s^3+2\xi\omega_n s^2+\omega_n^2s+K_1\omega_n^2} \\
D(s) &=s^3+2\xi\omega_n s^2+\omega_n^2s+K_1\omega_n^2=0\\
D(s) &=s^3+ 34.6 s^2 + 7500 s + 7500 K_1=0\\
\end{align*}
\end{frame}
\begin{frame}
\frametitle{比例-积分控制示例(稳定性)}
\label{sec-3-3}

Routh表:
\[
\begin{matrix}
s^3 & 1    & 7500 \\
s^2 & 34.6 & 7500K_1 \\
s^1 & \frac{34.6\times 7500-7500K_1}{34.6} & 0 \\
s^0 & 7500K_1 
\end{matrix}
\]

得:
\[
0<K_1<34.6
\]
\end{frame}
\begin{frame}
\frametitle{比例-积分控制示例(相对稳定性)}
\label{sec-3-4}

设: $s=s_1-1$

\begin{align*}
(s_1-1)^3+34.6(s_1-1)^2+7500(s_1-1)+7500K_1 &=0 \\
s_1^3 + 31.6 s_1^2+7433.8s_1+(7500K_1-7466.4)&=0
\end{align*}
Routh表:
\[
\begin{matrix}
s^3 & 1    & 7433.8 \\
s^2 & 31.6 & 7500K_1-7466.4 \\
s^1 & \frac{31.6\times 7433.8-(7500K_1-7466.4)}{31.6} & 0 \\
s^0 & 7500K_1-7466.4
\end{matrix}
\]
得: 
\[
1<K_1<32.3
\]
\end{frame}

\end{document}
