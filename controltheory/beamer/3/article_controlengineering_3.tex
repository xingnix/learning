% Created 2013-10-20 Sun 19:10
\documentclass{article}
\usepackage[utf8]{inputenc}
\usepackage[T1]{fontenc}
\usepackage{fixltx2e}
\usepackage{graphicx}
\usepackage{longtable}
\usepackage{float}
\usepackage{wrapfig}
\usepackage{soul}
\usepackage{textcomp}
\usepackage{marvosym}
\usepackage{wasysym}
\usepackage{latexsym}
\usepackage{amssymb}
\usepackage{hyperref}
\tolerance=1000
\usepackage{amsmath}
\usepackage[usenames]{color}
\usepackage{pstricks}
\usepackage{pgfplots}
\usepackage{tikz}
\usepackage[europeanresistors,americaninductors]{circuitikz}
\usepackage{colortbl}
\usepackage{yfonts}
\usetikzlibrary{shapes,arrows}
\usetikzlibrary{positioning}
\usetikzlibrary{arrows,shapes}
\usetikzlibrary{intersections}
\usetikzlibrary{calc,patterns,decorations.pathmorphing,decorations.markings}
\usepackage[BoldFont,SlantFont,CJKchecksingle]{xeCJK}
\setCJKmainfont[BoldFont=Evermore Hei]{Evermore Kai}
\setCJKmonofont{Evermore Kai}
\xeCJKsetup{CJKglue=\hspace{0pt plus .08 \baselineskip }}
\usepackage{pst-node}
\usepackage{pst-plot}
\psset{unit=5mm}
\usepackage{beamerarticle}
\mode<beamer>{\usetheme{Frankfurt}}
\mode<beamer>{\usecolortheme{dove}}
\mode<article>{\hypersetup{colorlinks=true,pdfborder={0 0 0}}}
\mode<beamer>{\AtBeginSection[]{\begin{frame}<beamer>\frametitle{Topic}\tableofcontents[currentsection]\end{frame}}}
\setbeamercovered{transparent}
\subtitle{}
\providecommand{\alert}[1]{\textbf{#1}}

\title{线性系统时域分析法}
\author{}
\date{}
\hypersetup{
  pdfkeywords={},
  pdfsubject={},
  pdfcreator={Emacs Org-mode version 7.9.3f}}

\begin{document}

\maketitle

\begin{frame}
\frametitle{Outline}
\setcounter{tocdepth}{3}
\tableofcontents
\end{frame}








\section{动态性能分析}
\label{sec-1}






\mode<article>{分析 $\sigma\%,t_s$ 等指标, $r(t)=1,R(s)=\frac{1}{s}$ }
\subsection{一阶系统动态性能}
\label{sec-1-1}
\begin{frame}
\frametitle{惯性环节动态性能}
\label{sec-1-1-1}

\begin{psmatrix}[rowsep=0.4,colsep=0.5]
%              
%              .------.
% R-->o----- ->| 1/Ts |--+--> C
%   _ ^        '------'  |
%     |                  |  
%     '------------------'
%
%
% 1                        2                        3             4              5    6
$R(s)$ &  \pscirclebox[framesep=-0.2em]{$\times$} &   &  \psframebox{$\frac{1}{Ts}$}   & {\hskip 1em}   & $C(s)$ \\
%link
\ncline{->}{1,1}{1,2}
\ncline{->}{1,2}{1,4}
\ncline{->}{1,4}{1,6}
%\ncangle[angleA=0,angleB=0,armA=0.5em,armB=0.5em]{1,4}{2,4}
\ncangles[angleA=0,angleB=-90,armA=1em,armB=2em]{->}{1,4}{1,2}
\naput[npos=3.6]{$-$}
\end{psmatrix}

\begin{eqnarray*}
G(s) & = & \frac{1}{Ts}\\
\Phi(s) &=& \frac{1}{Ts+1} \\
C(s) &=& \Phi(s)R(s) \\
     &=& \frac{-T}{Ts+1}+\frac{1}{s} \\
c(t) &=& 1-e^{-t/T}
\end{eqnarray*}
\end{frame}
\subsection{二阶系统时域分析}
\label{sec-1-2}
\begin{frame}
\frametitle{传递函数}
\label{sec-1-2-1}

\begin{psmatrix}[rowsep=0.4,colsep=0.5]
%              
%              .----------------------.
% R-->o----- ->| w_n^2/s^2+2\xi\w_n s |--+--> C
%   _ ^        '----------------------'  |
%     |                                  |  
%     '----------------------------------'
%
% 1                        2                        3             4              5    6
$R(s)$ &  \pscirclebox[framesep=-0.2em]{$\times$} &   &  \psframebox{$\frac{\omega_n^2}{s^2+2\xi\omega_n s}$}   & {\hskip 1em}   & $C(s)$ \\
%link
\ncline{->}{1,1}{1,2}
\ncline{->}{1,2}{1,4}
\ncline{->}{1,4}{1,6}
%\ncangle[angleA=0,angleB=0,armA=0.5em,armB=0.5em]{1,4}{2,4}
\ncangles[angleA=0,angleB=-90,armA=1em,armB=2em]{->}{1,4}{1,2}
\naput[npos=3.6]{$-$}
\end{psmatrix}

\begin{itemize}
\item $\xi$: 阻尼比
\item $\omega_n$:自然频率,无阻尼振荡频率
\end{itemize}
\begin{eqnarray*}
r(t) &=& 1 \\
R(s) &=& \frac{1}{s}\\
G(s) & =& \frac{\omega_n^2}{s^2+2\xi\omega_n s} \\
\Phi(s) &=& \frac{\omega_n^2}{s^2+2\xi\omega_n s+\omega_n^2}\\
p_{1,2} &=& -\xi\omega_n\pm\omega_n\sqrt{\xi^2-1}
\end{eqnarray*}
\end{frame}
\begin{frame}
\frametitle{$\xi\leq 0$}
\label{sec-1-2-2}

\begin{itemize}
\item <2-> $\xi< 0$ 时有正实根,不稳定
\item <3-> $\xi=0$ 时有两个纯虚根,无阻尼,临界稳定,等幅振荡,频率为$\omega_n$,
	 \begin{eqnarray*}
	 C(s) & = & \frac{\omega_n^2}{s^2+\omega_n^2}\cdot \frac{1}{s}  \\
	      & =& \frac{-s}{s^2+\omega_2}+\frac{1}{s} \\
	 c(t) &=& 1-\cos\omega_n t
	 \end{eqnarray*}
\end{itemize}
\end{frame}
\begin{frame}
\frametitle{$\xi>1$}
\label{sec-1-2-3}

    系统闭环极点为两个不同的实根.过阻尼,相当于两个一阶系统并联, $\sigma\%=0$
       \begin{eqnarray*}
       \Phi(s) & = & \frac{\omega_n^2}{(s-p_1)(s-p_2)} \\
	       & = & \frac{K_1}{s-p_1}+\frac{K_2}{s-p_2}\\
       c(t)    &=& 1-\frac{e^{p_1 t}}{1-\frac{p_1}{p_2}}-\frac{e^{p_2 t}}{1-\frac{p_2}{p_1}}
       \end{eqnarray*}
\end{frame}
\begin{frame}
\frametitle{$\xi=1$}
\label{sec-1-2-4}

\begin{itemize}
\item <1-> 闭环极点有两个相同的负实根$p_{1,2}=-\xi\omega_n=-\omega_n$
       \begin{eqnarray*}
       C(s) & = &\frac{\omega_n^2}{(s+\omega_n^2)}\cdot\frac{1}{s} \\
       c(t) &=& 1-e^{-\omega_n t}(1+\omega_n t)
       \end{eqnarray*}
\item <2-> 且有:
       \begin{eqnarray*}
       \frac{dc(t)}{dt} &=& \omega_ne^{-\omega_n t}(1+\omega_n t)-\omega_n e^{-\omega_n t}
       	=  \omega_n^2 te^{-\omega_n t} 
       	>  0 \\
       c(0) &=&0 \\
       c(\infty)&=&1\\
       \sigma \% &=& 0\\
       t_s &=& 4.75T \\
       T &=&\frac{1}{\omega_n}
       \end{eqnarray*}
\end{itemize}
\end{frame}
\begin{frame}
\frametitle{$0<\xi<1$}
\label{sec-1-2-5}

   系统有一对实部小于零的共轭复根, $p_{1,2}  =  -\xi\omega_n\pm j\omega_n\sqrt{1-\xi^2}$
\begin{eqnarray*}
C(s) &=& \frac{\omega_n^2}{s^2+2\xi\omega_n s+\omega_n^2}\cdot\frac{1}{s} \\
     &=& \frac{1}{s}-\frac{s+\xi\omega_n}{(s+\xi\omega_n)^2+(1-\xi^2)\omega_n^2}-\frac{\xi\omega_n}{(s+\xi\omega_n)^2+(1-\xi^2)\omega_n^2} \\
c(t) &=& 1-\frac{1}{\sqrt{1-\xi^2}}e^{-\xi\omega_n t}(\sqrt{1-\xi^2}\sin\sqrt{1-\xi^2}\omega_n t \\
     & &+\xi\cos\sqrt{1-\xi^2}\omega_n t)\\
     &=& 1-\frac{1}{\sqrt{1-\xi^2}}e^{-\xi\omega_n t}\sin(\omega_d t+\beta)\\
\beta & = & \tan^{-1}\frac{\sqrt{1-\xi^2}}{\xi} \\
\omega_d &=& \sqrt{1-\xi^2}\omega_n
\end{eqnarray*}
\end{frame}
\subsection{二阶系统阶跃响应指标计算}
\label{sec-1-3}
\begin{frame}
\frametitle{二阶欠阻尼系统阶跃响应指标}
\label{sec-1-3-1}


\begin{eqnarray*}
   c(t)  &=& 1-\frac{1}{\sqrt{1-\xi^2}}e^{-\xi\omega_n t}\sin(\omega_d t+\beta)\\
\end{eqnarray*}

\begin{itemize}
\item 欠阻尼. $\omega_d$ 称为有阻尼振荡频率.最佳阻尼比 $\xi=0.707$
\item 指标: $\sigma\% , t_s , t_p , t_r$ 等
\end{itemize}
\end{frame}
\begin{frame}
\frametitle{上升时间 $t_r$}
\label{sec-1-3-2}

\begin{itemize}
\item $100\%$ 的 $t_r$ : $c(t)$ 首次达到 $c(\infty)$ 的时间
\item $90\%$ 的 $t_r$ : $c(t)$ 首次达到 $90\%c(\infty)$ 的时间
\item $70\%$ 的 $t_r$ : $c(t)$ 首次达到 $70\%c(\infty)$ 的时间
\end{itemize}

\begin{eqnarray*}
c(t) & = & c(\infty) \\
1-\frac{1}{\sqrt{1-\xi^2}}e^{-\xi\omega_n t}\sin(\omega_d t+\beta) &=& 1 \\
sin(\omega_d t+\beta) &=& 0 \\
\omega_d t+\beta &=& k\pi \\
t_r &=& \frac{\pi-\beta}{\omega_d}
\end{eqnarray*}
\end{frame}
\begin{frame}
\frametitle{峰值时间 $t_p$}
\label{sec-1-3-3}


\mode<article>{$c(t)$ 达到最大值的时间}

\begin{eqnarray*}
\frac{dc(t)}{dt} &=& 0 \\
-\xi\omega_n e^{-\xi\omega_n t}\sin(\omega_d t+\beta)+e^{-\xi\omega_n t}\omega_d\cos(\omega_d t+\beta) & = & 0 \\
\omega_d\cos(\omega_d t+\beta) &=& \xi\omega_n \sin(\omega_d t+\beta) \\
\tan(\omega_d t+\beta) &=& \frac{\sqrt{1-\xi^2}}{\xi} \\
\tan(\omega_d t+\beta) &=& \tan\beta \\
\omega_d t &=& k\pi\\
t_p &=& \frac{\pi}{\omega_d}
\end{eqnarray*}
\end{frame}
\begin{frame}
\frametitle{超调量 $\sigma \%$}
\label{sec-1-3-4}

\begin{eqnarray*}
\sigma \% & = & \frac{c_{max}-c(\infty)}{c(\infty)}\times 100\% 
         = (c(t_p)-1) \\
         &=& -\frac{1}{\sqrt{1-\xi^2}}e^{-\xi\omega_n t_p}\sin(\omega_d t_p+\beta) \\
         &=& -\frac{1}{\sqrt{1-\xi^2}}e^{-\frac{\xi\omega_n\pi}{\omega_d}}\sin(\pi+\beta) \\
         &=& \frac{1}{\sqrt{1-\xi^2}}e^{-\frac{\xi\pi}{\sqrt{1-\xi^2}}}\sin(\beta) \\
         &=& e^{-\frac{\xi\pi}{\sqrt{1-\xi^2}}}\times 100\% \\
\end{eqnarray*}

\mode<article>{分析:}
\begin{itemize}
\item <2->$\sigma\%$ 只与 $\xi$ 有关,两者成反比关系
\item <3->工程上一般取 $\sigma\%\in[0.4,0.8]$
\item <4->最佳阻尼比 $\xi=0.707,\sigma\%=4.3\%$
\end{itemize}
\end{frame}
\begin{frame}
\frametitle{调节时间 $t_s$}
\label{sec-1-3-5}


\mode<article>{近似估算:}

\begin{eqnarray*}
c(t) & = & 1-\frac{1}{\sqrt{1-\xi^2}}e^{-\xi\omega_n t}\sin(\omega_d t+\beta)\\
     &\approx & 1-\frac{1}{\sqrt{1-\xi^2}}e^{-\xi\omega_n t} \\
%     &\approx & 1-e^{-\xi\omega_n t} \\
e(t) &=& c(\infty)-c(t) \\
    &\approx& \frac{1}{\sqrt{1-\xi^2}}e^{-\xi\omega_n t}\\ 
%     &\approx& e^{-\xi\omega_n t}
\end{eqnarray*}

\begin{itemize}
\item <2-> $t_s$ 与 $\omega_n,\xi$ 有关:通常取 $\xi\omega_n t_s = 3.5,t_s=\frac{3.5}{\xi\omega}$
\end{itemize}
       
\end{frame}
\begin{frame}
\frametitle{二阶过阻尼系统阶跃响应指标}
\label{sec-1-3-6}

\begin{itemize}
\item <2->$\sigma\%=0$
\item <3->$\xi=1$ 时, 
       	\[t_s=\frac{4.75}{\omega_n}\]
\item <4->$\xi>1,|p_1|\gg |p_2|$ 时,系统降阶,去掉极点 $p_2$ , 
       \[t_s=\frac{3}{|p_1|}\]
\end{itemize}
\end{frame}
\subsection{高阶系统时域分析(3阶及以上系统)}
\label{sec-1-4}
\begin{frame}
\frametitle{三阶系统}
\label{sec-1-4-1}

\begin{itemize}
\item <2-> 根的几种情况
\begin{itemize}
\item <3-> 3个负实根 $p_1,p_2,p_3$
\item <4-> 1个负实根,一对共轭复根 
       \[-s_0,-\xi\omega_n\pm j\omega_n\sqrt{1-\xi^2},(0<\xi<1)\]
\end{itemize}
\item <5-> 重点考虑有复根的情况.
\end{itemize}
\end{frame}
\begin{frame}
\frametitle{三阶系统($\Phi(s)$) 单位阶跃响应($C(s)$)}
\label{sec-1-4-2}

\begin{eqnarray*}
 \Phi(s) & = & \frac{s_0\omega_n^2}{(s+s_0)(s^2+2\xi\omega_n s+\omega_n^2)} \\
 C(s) &=& \frac{s_0\omega_n^2}{s(s+s_0)(s^2+2\xi\omega_n s+\omega_n^2)} \\
 c(t) &=& 1-\frac{e^{-s_0 t}}{b\xi^2(b-2)+1}-\frac{e^{-\xi\omega_n t}}{b\xi^2(b-2)+1} \\
     & & \left(b\xi^2(b-2)\cos\omega_d t + \frac{b\xi(\xi^2(b-2)+1)}{\sqrt{1-\xi^2}}\sin\omega_d t\right) \\
 \omega_d &=& \omega_n\sqrt{1-\xi^2} \\
 b &=& \frac{s_0}{\xi\omega_n}
\end{eqnarray*}
\end{frame}
\begin{frame}
\frametitle{$b$ 对 $c(t)$ 的影响}
\label{sec-1-4-3}


\begin{itemize}
\item <2->复根比实根离虚轴近得多
      \begin{eqnarray*}
      b & \gg & 1\\
      c(t) &\approx & 1-e^{-\xi\omega_n t}\left(\cos\omega_d t + \frac{1}{\sqrt{1-\xi^2}}\sin\omega_d t\right) 
      \end{eqnarray*}
      近似看作2阶欠阻尼系统.
\item <3->实根比复根离虚轴近得多
      \begin{eqnarray*}
      b & \approx & 0\\
      c(t) &\approx & 1-e^{-s_0 t}
      \end{eqnarray*}
      近似看作1阶系统
\item <4->实根与复根与虚轴距离同
      \begin{eqnarray*}
      b & \approx & 1\\
      c(t) &\approx & 1-\frac{e^{-\xi\omega_n t}}{1-\xi^2}\left(1+\xi\sin(\omega_d t-\beta)\right) 
      \end{eqnarray*}
\end{itemize}
\end{frame}
\begin{frame}
\frametitle{主导极点法}
\label{sec-1-4-4}

\begin{itemize}
\item <2->目的:分析高阶系统的性能
\item <3->内容:系统有多个极点,其中某些极点决定了整个系统的性能,对系统起主导作用,称这些极点为主导极点.
\item <4->确定方法:主导极点离虚轴距离为 $a$ ,其它极点离虚轴距离 $\geq 5a$
\end{itemize}
\end{frame}
\section{稳定性分析}
\label{sec-2}
\subsection{稳定性的概念}
\label{sec-2-1}
\begin{frame}
\frametitle{稳定性}
\label{sec-2-1-1}


\psset{unit=1em}
\begin{pspicture}(0,0)(30em,5em)
%\begin{pspicture}[showgrid](0,0)(30em,10em)
%
%       o    \   o   /  \     /
%       ^     \  ^  /    \ o /
%      / \     \/ \/      \_/   
%
\psset{unit=1em}
%\multips(1.5,2.5)(4,0){3}{\pscircle(0,0){0.5}}
\psset{linecolor=red}
%\psline(0,0)(50,6)
\pscircle(4.5,4){1}
\pscurve(0,0)(4.5,3)(9,0)
\psset{linecolor=violet}
\pscircle(14.5,2){1}
\pscurve(10,0)(12,2)(14.5,1)(17.5,2)(19,0)
\psset{linecolor=blue}
\pscircle(24.5,1){1}
\pscurve(20,2)(24.5,0)(29,2)
%\psplot{0}{6}{x sqrt}
%\rput(3,3){中文}
%\pscustom{\dim{1}
%\code{
%dup dup  scale currentlinewidth exch div setlinewidth 
%0 0 moveto 3 6 lineto stroke}}
\end{pspicture}

定义:系统处于平衡状态时,若有干扰使系统偏离平衡状态,当扰动消除后,系统仍能回到原来的平衡状态,则称该系统是稳定的,反之称为不稳定.
\end{frame}
\begin{frame}
\frametitle{稳定的充要条件}
\label{sec-2-1-2}


\begin{itemize}
\item 系统的传递函数
      \begin{eqnarray*}
      \Phi(s) & = & \frac{C(s)}{R(s)} = \frac{k_{g}\prod_{i=1}^{m}(s-z_{i})}{\sum_{j=1}^{n}(s-p_{j})} 
      \end{eqnarray*}
\item 设系统输入为单位脉冲信号: $r(t)=\delta(t),R(s)=1$
      \begin{eqnarray*}
      C(s) & = & \Phi(s)R(s) =  \frac{k_{g}\prod_{i=1}^{m}(s-z_{i})}{\sum_{j=1}^{n}(s-p_{j})} \\
	   & = &  k_{g}\sum_{j=1}^{n}\frac{k_{j}}{s-p_{j}} \\
      c(t) & = & k_{g}\sum_{j=1}^{n}k_{j}e^{p_{j}t}
      \end{eqnarray*}
\item 当 $\Re(p_{j})<0$ 时,有 $\lim_{t\rightarrow\infty}c(t) = 0$
\item 稳定性充要条件:系统的闭环极点均具有负实部
\end{itemize}
\end{frame}
\subsection{古尔维茨判据及劳斯判据}
\label{sec-2-2}
\begin{frame}
\frametitle{特征多项式}
\label{sec-2-2-1}

\begin{eqnarray*}
G(s) & = & \frac{M(s)}{N(s)}\\
M(s) &=&  b_{o}s^{m}+\cdots+b_{n} \\
N(s) & =& a_{0}s^{n}+\cdots+a_{n}
\end{eqnarray*}
其中:
\begin{itemize}
\item $N(s)$ 称为特征多项式
\item $N(s)=0$ 称为特征方程(只与 $a_{0},\cdots,a_{n}$ 有关)
\end{itemize}
\end{frame}
\begin{frame}
\frametitle{古尔维茨判据(代数判据):}
\label{sec-2-2-2}

\begin{itemize}
\item <2-> 稳定的必要条件:特征多项式中各项系数大于零.
\item <3-> 稳定的充要条件:特征多项式中各项系数所构成的主行列式及顺序主子式全部大于零.
\item <3-> 主行列式:
      \begin{equation*}
      \Delta_{n}=\left|
      \begin{matrix}
      a_{1}  & a_{3} & a_{5} & \cdots & 0  \\
      a_{0}  & a_{2} & a_{4} & \cdots & 0  \\
      0      & a_{1} & a_{3} & \cdots & 0  \\
      \vdots & a_{0} & a_{2} & \cdots & 0  \\
      \vdots &\vdots &\vdots &        & 0   \\
       	0    &  0    & \cdots & a_{n-2}& a_{n}
      \end{matrix}
      \right|
      \end{equation*}
\item <4-> 顺序主子式有n-1个: $\Delta_{1},\cdots,\Delta_{n-1}$
\item <5-> 李纳德-戚帕特判据:若所有奇次古尔维茨行列式为正,则偶次古尔维茨行列式必为正,反之亦然
\end{itemize}
\end{frame}
\begin{frame}
\frametitle{劳斯-古尔维茨判据,简称劳斯判据}
\label{sec-2-2-3}

\begin{itemize}
\item <2->构造劳斯表判断系统是否稳定
      \begin{equation*}
      \begin{matrix}
      s^{n}   &  a_{0}  &  a_{2}  & a_{4}  & \cdots  \\
      s^{n-1} &  a_{1}  & a_{3}   & a_{5}  & \cdots  \\
      s^{n-2} &  c_{1}=\frac{a_1 a_2 - a_0 a_3}{a_1} &  c_{2}=\frac{a_1 a_4 - a_0 a_5}{a_1} & \cdots \\
      s^{n-3} &  d_{1}=\frac{c_1 a_3 - a_1 c_2}{c_1} &  d_{2}=\frac{c_1 a_5 - a_1 c_3}{c_1} & \cdots \\
      \vdots  &   \vdots                             &                                      & \cdots  \\
      s^{0}   &   a_{n}                              &                                      &
      \end{matrix}
      \end{equation*}
\item <3->劳斯判据:
\begin{itemize}
\item <4->系统稳定的充要条件:劳斯表中第一列元素均大于零
\item <5->若第一列元素有小于零的,则系统不稳定,且不稳定根的个数等于第一列元素变号的次数.
\end{itemize}
\end{itemize}
\end{frame}
\subsection{劳斯判据示例}
\label{sec-2-3}
\begin{frame}
\frametitle{Routh 判据示例1: $3s^{4}+10s^{3}+5s^{2}+s+2=0$}
\label{sec-2-3-1}


解:
\begin{itemize}

\item 劳斯表:
\label{sec-2-3-1-1}%
\begin{equation*}
\begin{matrix}
s^{4} & 3   &  5  &  2 \\
s^{3} & 10  &  1        \\
s^{2} & 4.7 &  2       \\
s^{1} & -\frac{15.3}{4.7}  \\
s^{0} & 2   
\end{matrix}
\end{equation*}


\item 结论\\
\label{sec-2-3-1-2}%
系统不稳定,有2个不稳定根.

\end{itemize} % ends low level
\end{frame}
\begin{frame}
\frametitle{Routh 判据示例2:}
\label{sec-2-3-2}

\begin{psmatrix}[rowsep=0.4,colsep=0.5]
%              .---------------.
%              |     s+1       |
% R-->o--> K ->| ------------- |--+--> C
%   _ ^        |  (s(s+2)(s+3) |  |
%     |        '---------------'  |
%     '---------------------------'
%
%
% 1                        2                        3                           4              5    6
$r$ &  \pscirclebox[framesep=-0.2em]{$\times$}& \psframebox{$K$} &  \psframebox{$\frac{s+1}{s(s+2)(s+3)}$}&  & $y$ \\
%link
\ncline{->}{1,1}{1,2}
\ncline{->}{1,2}{1,3}
\ncline{->}{1,3}{1,4}
\ncline{->}{1,4}{1,6}
\ncangles[angleA=0,angleB=-90,armA=0.5em,armB=1em]{->}{1,4}{1,2}
\naput[npos=3.6]{$-$}
\end{psmatrix}

求使系统稳定的 $K$ 的范围

\mode<article>{解:}
\begin{itemize}

\item 传递函数\\
\label{sec-2-3-2-1}%
\begin{eqnarray*}
G(s)     & = & \frac{K(s+1)}{s(s+2)(s+3)}\\
\Phi(s)  & = & \frac{K(s+1)}{s^{3}+5s^{2}+(6+K)s+K}
\end{eqnarray*}


\item 劳斯表:\\
\label{sec-2-3-2-2}%
\[
\begin{matrix}
s^{3}  &    1   &  6+K  \\
s^{2}  &    5   &  K  \\
s^{1}  &   \frac{30+4K}{5}  & 0 \\
s^{0}  &  K
\end{matrix}
\]

稳定条件:
\begin{eqnarray*}
30+4K & > & 0 \\
K & >  & 0
\end{eqnarray*}

得: $K>0$

\end{itemize} % ends low level
\end{frame}
\subsection{劳斯判据应用中的特殊情况}
\label{sec-2-4}
\begin{frame}
\frametitle{Routh表特殊情况:第一列元素有零元,系统不稳定:}
\label{sec-2-4-1}

\begin{itemize}
\item 先用 $\epsilon$ 代替,再令 $\epsilon\rightarrow 0$
\item 系统升阶: $(s+a)D(s),a>0$
\end{itemize}
\begin{itemize}

\item 例:\\
\label{sec-2-4-1-1}%
系统特征方程: $D(s)=s^{4}+2s^{3}+s^{2}+2s+1=0$ 判断系统的稳定性,求出不稳定根的个数.


\item Roth表:\\
\label{sec-2-4-1-2}%
\[
\begin{matrix}
s^{4}   &  1  &   1  \\
s^{3}   &  2  &  2   \\
s^{2}   &  0(\epsilon)  &  1   \\
s^{1}   &  \frac{2\epsilon-2}{\epsilon} & 0 \\
s^{0}   &  1   \\
\end{matrix}
\]


\item $(s+1)D(s)=s^{5}+3s^{4}+3s^{3}+3s^{2}+3s+1$\\
\label{sec-2-4-1-3}%
\[
\begin{matrix}
s^{5} & 1 & 3 & 3 \\
s^{4} & 3 & 3 & 1 \\
s^{3} & 2 & \frac{8}{3} \\
s^{2} & -1 & 1 \\
s^{1} & \frac{14}{3} & 0 \\
s^{0} & 1
\end{matrix}
\]

\end{itemize} % ends low level
\end{frame}
\begin{frame}
\frametitle{特殊情况:全行为零时:}
\label{sec-2-4-2}

\begin{itemize}
\item 全零行的数值由上一行求导代替
\item 辅助方程:全零行的上一行可构成辅助方程,辅助方程的解全部为系统特征根
\end{itemize}
\begin{itemize}

\item 例:\\
\label{sec-2-4-2-1}%
$D(s)=s^{4}+5s^{3}+5s^{2}-5s-6$ 求系统不稳定根的个数


\item Routh表\\
\label{sec-2-4-2-2}%
\[
\begin{matrix}
s^{4} &  1 &  5 & -6 \\
s^{3} &  5 & -5 \\
s^{2} &  6 & -6 \\
s^{1} &  0(12)  & 0 \\
s^{0} &  -6
\end{matrix}
\]


\item 辅助方程
\label{sec-2-4-2-3}%
\begin{itemize}
\item 对 $6s^{2}-6$ 求导得 $12s$
\item 辅助方程 $6s^2-6=0$ ,得 $s_{1,2}=\pm 1$ 系统不稳定根为1
\end{itemize}

\end{itemize} % ends low level
\end{frame}
\subsection{劳斯判据求解系统参数}
\label{sec-2-5}
\begin{frame}
\frametitle{例: $D(s)=s^{4}+2s^{3}+ks^{2}+s+2$ 求 $K$ 值稳定范围}
\label{sec-2-5-1}


解:
\begin{itemize}

\item Routh表\\
\label{sec-2-5-1-1}%
\[
\begin{matrix}
s^{4} & 1 & K & 2\\
s^{3} & 2 & 1 \\
s^{2} & \frac{2K-1}{2} & 2\\
s^{1} & 1-\frac{8}{2K-1} & 0 \\
s^{0} & 2 \\
\end{matrix}
\]
\begin{eqnarray*}
2K-1 & > & 0 \\
%   K & > 1 \\
1-\frac{8}{2K-1} &>& 0  
%   K &> \frac{9}{2} 
\end{eqnarray*}
得: $K>\frac{9}{2}$


\item 当 $K=\frac{9}{2}$ 时:\\
\label{sec-2-5-1-2}%
Routh表:

$$
\begin{matrix}
s^{4} & 1 & 4.5 & 2\\
s^{3} & 2 & 1 \\
s^{2} & 4 & 2\\
s^{1} & 0(8) & 0 \\
s^{0} & 2 \\
\end{matrix}
$$
其中辅助方程为 $4s^{2}+2=0$ ,可解得 $s=\pm\frac{\sqrt{2}}{2}j$ 

\end{itemize} % ends low level
\end{frame}
\begin{frame}
\frametitle{例: $D(s)=Ts^{3}+s^{2}+K=0$}
\label{sec-2-5-2}
\begin{itemize}

\item Routh表\\
\label{sec-2-5-2-1}%
\[
\begin{matrix}
s^{3} & T  & 0\\
s^{2} & 1  & K \\
s^{1} & -TK \\
s^{0} & K 
\end{matrix}
\]


\item 无解
\label{sec-2-5-2-2}%
\begin{eqnarray*}
T & > & 0 \\
K & > & 0 \\
TK& < & 0 
\end{eqnarray*}

\end{itemize} % ends low level
\end{frame}
\subsection{相对稳定性}
\label{sec-2-6}
\begin{frame}
\frametitle{例:$D(s)=s^{3}+5.5s^{2}+8.5s+3$}
\label{sec-2-6-1}

\begin{itemize}
\item 判断系统是否具有相对稳定性:$\sigma=1$
\end{itemize}

解:
\begin{itemize}

\item Routh表\\
\label{sec-2-6-1-1}%
\[
\begin{matrix}
s^{3} & 1 &  8.5 \\
s^{2} & 5.5  & 3 \\
s^{1} & 8.5-\frac{3}{5.5} & 0 \\
s^{0} & 3
\end{matrix}
\]


\item 将 $s=z-\sigma$ 代入 $D(s)$ 中\\
\label{sec-2-6-1-2}%
得 $D(z)=z^{3}+2.5z^{2}+0.5z-1$ ,不稳定.

\begin{itemize}

\item Routh表:\\
\label{sec-2-6-1-2-1}%
\[
\begin{matrix}
s^{3} & 1 &  0.5 \\
s^{2} & 2.5  & -1 \\
s^{1} & 0.5+\frac{1}{2.5} & 0 \\
s^{0} & -1
\end{matrix}
\]


\item 结论\\
\label{sec-2-6-1-2-2}%
有一个不稳定极点.


\end{itemize} % ends low level
\end{itemize} % ends low level
\end{frame}
\section{稳态误差}
\label{sec-3}
\subsection{误差传递函数}
\label{sec-3-1}
\begin{frame}
\frametitle{系统误差}
\label{sec-3-1-1}
\begin{itemize}

\item ignore
\label{sec-3-1-1-1}%
\begin{psmatrix}[rowsep=0.4,colsep=0.5]
%              
%         E(s) .------.
% R-->o----- ->| G(s) |--+--> C
%   _ ^        '------'  |
%     |                  |  
%     '--------[ H(s) ]--'
%
%
% 1                        2                        3             4              5    6
$R(s)$ &  \pscirclebox[framesep=-0.2em]{$\times$} &$\cdots $   &  \psframebox{$G(s)$}   &   & $C(s)$ \\
       &                                          &       &  \psframebox{$H(s)$}&  &        \\
%link
\ncline{->}{1,1}{1,2}
\ncline{->}{1,2}{1,4}
\naput{$E(s)$}
\ncline{->}{1,4}{1,6}
\ncangle[angleA=0,angleB=0,armA=0.5em,armB=0.5em]{1,4}{2,4}
\ncangle[angleA=180,angleB=-90,armA=0.5em,armB=1em]{->}{2,4}{1,2}
\naput[npos=2.3]{$-$}
\end{psmatrix}

\mode<article>{系统误差有两种:}

\begin{itemize}
\item <2->输入端定义:$E_{2}(s)=E(s)$
\item <3->输出端定义:$E_{1}(s)=C_{expect}-C_{real}$
\item <4->不加特别说明,系统误差指的是输入端定义.
\end{itemize}


\item $E(s)$ 与 $E_1(s)$\\
\label{sec-3-1-1-2}%
\begin{eqnarray*}
C_{expect} & = & \frac{R(s)}{H(s)}\\
E_{1}(s)   & = & \frac{R(s)}{H(s)}-C(s) \\
           & = & \frac{R(s)-C(s)H(s)}{H(s)}\\
           & =& \frac{E(s)}{H(s)}
\end{eqnarray*}

\end{itemize} % ends low level
\end{frame}
\begin{frame}
\frametitle{误差传递函数:}
\label{sec-3-1-2}

\begin{eqnarray*}
\Phi_{e}(s) & = & \frac{E(s)}{R(s)}\\
            & = & \frac{1}{1+G(s)H(s)} \\
            & = & \frac{R(s)-H(s)C(s)}{R(s)} \\
            & = & 1-H(s)\Phi(s)
\end{eqnarray*}

\begin{itemize}
\item <2-> 系统误差:$E(s)=\Phi_{e}(s)R(s)$
\end{itemize}
\end{frame}
\begin{frame}
\frametitle{稳态误差:}
\label{sec-3-1-3}

\begin{eqnarray*}
e_{ss} &=& \lim_{t\rightarrow \infty}e(t) \\
       &=& \lim_{s\rightarrow 0}sE(s)  \\
       &= & \lim_{s\rightarrow 0}s\Phi_{e}(s)R(s)
\end{eqnarray*}

\begin{itemize}
\item <2->说明:
\begin{itemize}
\item <2->稳态误差与输入信号密切相关
\item <3->求稳态误差前要判断系统稳定性
\end{itemize}
\end{itemize}
\end{frame}
\begin{frame}
\frametitle{扰动作用下的稳态误差}
\label{sec-3-1-4}

\begin{psmatrix}[rowsep=0.4,colsep=0.5]
%          1    2  3     4    5    6    7 
%                            | N(s)
%               E(s)         v +  
%         R---->o--> G_1(s)--o- G_2(s)--+--> C
%             _ ^                       |
%               |                       |  
%               '-----------H(s)--------+
%           
%
% 1                         2                           3                  4                     5
       &                                       &               &                        &         $N(s)$                                  &   \\  
$R(s)$  &\pscirclebox[framesep=-0.2em]{$\times$}& {\hskip 1em} &  \psframebox{$G_1(s)$} &  \pscirclebox[framesep=-0.2em]{$\times$} &  \psframebox{$G_2(s)$}  & \    & $C(s)$ \\
       &                                         &              &     &   \psframebox{$ H(s) $} 
%link
\ncline{->}{1,5}{2,5}
%\naput{$N(s)$}
\ncline{->}{2,1}{2,2}
\ncline{->}{2,2}{2,4}
\naput{$E(s)$}
\ncline{->}{2,4}{2,5}
\ncline{->}{2,5}{2,6}
\ncline{->}{2,6}{2,8}
\ncangles[angleA=180,angleB=0,armA=0em,armB=0em]{->}{2,7}{3,5}
\ncangle[angleA=180,angleB=-90,armA=0em,armB=0em]{->}{3,5}{2,2}
\naput[npos=1.6]{$-$}
\end{psmatrix}
\begin{itemize}

\item 输入端定义:\\
\label{sec-3-1-4-1}%
\begin{eqnarray*}
E(s) & = &E_R(s)+E_N(s) \\
E_R(s)&=& \Phi_e(s)R(s) \\
E_N(s)&=& \Phi_{en}(s)N(s) \\
\end{eqnarray*}


\item 输出端定义:\\
\label{sec-3-1-4-2}%
令 $R(s)=0$ ,计算 $N(s)$ 单独引起的 $e_{ss}$ ,此时 $C_{expect}(s)=0$ 

\begin{eqnarray*}
E(s) & = & 0-C(s) \\
     & = & -\Phi_N(s)N(s)\\
\Phi_N(s) &=& \frac{G_2}{1+G_1G_2}\\
e_{ss}&=&\lim_{s\rightarrow 0}s(-\Phi_N(s)N(s)) 
\end{eqnarray*}


\end{itemize} % ends low level
\end{frame}
\subsection{系统类型与静态误差系数}
\label{sec-3-2}
\begin{frame}
\frametitle{阶跃输入:}
\label{sec-3-2-1}

\begin{eqnarray*}
r(t) & = & A \\
R(s) & = & \frac{A}{s} \\
e_{ss}&=& \lim_{s\rightarrow 0}s \cdot\frac{1}{1+G_{open}(s)}\cdot\frac{A}{s} \\
      &=& \lim_{s\rightarrow 0}\frac{A}{1+G_{open}(s)}
\end{eqnarray*}
\end{frame}
\begin{frame}
\frametitle{速度输入}
\label{sec-3-2-2}

\begin{eqnarray*}
r(t) & = & vt \\
R(s) & = & \frac{v}{s^{2}} \\
e_{ss}&=& \lim_{s\rightarrow 0}s \cdot\frac{1}{1+G_{open}(s)}\cdot\frac{v}{s^{2}} \\
      &=& \lim_{s\rightarrow 0}\frac{A}{s+sG_{open}(s)}\\
      &=& \lim_{s\rightarrow 0}\frac{A}{sG_{open}(s)}
\end{eqnarray*}
\end{frame}
\begin{frame}
\frametitle{加速度输入}
\label{sec-3-2-3}

\begin{eqnarray*}
r(t) & = & \frac{1}{2}at^{2} \\
R(s) & = & \frac{a}{s^{2}} \\
e_{ss}&=& \lim_{s\rightarrow 0}s \cdot\frac{1}{1+G_{open}(s)}\cdot\frac{a}{s^{3}} \\
      &=& \lim_{s\rightarrow 0}\frac{A}{s^{2}+s^{2}G_{open}(s)}\\
      &=& \lim_{s\rightarrow 0}\frac{A}{s^{2}G_{open}(s)}
\end{eqnarray*}
\end{frame}
\begin{frame}
\frametitle{系统类型}
\label{sec-3-2-4}

\begin{itemize}
\item <2->由开环传递函数定义
      \begin{eqnarray*}
       G_{open} & = & G(s)H(s) \\
	       	& = & \frac{K\prod_{i=1}^{m}(\tau_{i}s+1)}{s^{\nu}\prod_{j=1}^{n-\nu}(T_{j}s+1)}
      \end{eqnarray*}
\item <2->其中 $K$ 为开环增益.
\item <3->定义:
\begin{itemize}
\item $\nu=0$ 称为0型系统
\item $\nu=1$ 称为I型系统
\item $\nu=2$ 称为II型系统
\end{itemize}
\end{itemize}
\end{frame}
\begin{frame}
\frametitle{静态误差系数}
\label{sec-3-2-5}

\begin{itemize}
\item <2->静态位置误差系数
       	\begin{eqnarray*}
       	r(t) &=& A\\
       	e_{ss}&=&\frac{A}{1+K_{p}}, \qquad
       	K_{p}=\lim_{s\rightarrow 0} G_{open}(s)
       	\end{eqnarray*}
\item <3->静态速度误差系数 
       	\begin{eqnarray*}
       	r(t)&=&vt\\
       	e_{ss}&=&\frac{v}{K_{v}}, \qquad
       	K_{v}=\lim_{s\rightarrow 0} sG_{open}(s)
       	\end{eqnarray*}
\item <4->静态加速度误差系数 
       	\begin{eqnarray*}
       	r(t)&=&\frac{1}{2}at^{2}\\
       	e_{ss}&=&\frac{a}{K_{a}}, \qquad
       	K_{v}=\lim_{s\rightarrow 0} s^{2}G_{open}(s)
       	\end{eqnarray*}
\end{itemize}
\end{frame}
\begin{frame}
\frametitle{零型系统($\nu=0$)}
\label{sec-3-2-6}

\begin{itemize}
\item <2-> $r(t)=A$ 时:
      \begin{eqnarray*}
      K_p &=& \lim_{s\rightarrow 0}G_o(s) 
	  = \lim_{s\rightarrow 0}\frac{K\prod_{i=0}^m(\tau_i s+1)}{\prod_{j=1}^n (\tau_j s+1)} 
	  = K \\
      e_{ss1} &=& \frac{A}{1+K_p}
      \end{eqnarray*}
      \mode<article>{称为有差系统.}
\item <3-> $r(t)=vt$ 时:
      \begin{eqnarray*}
      K_v &=& \lim_{s\rightarrow 0}sG_o(s) 
	  = 0 \\
      e_{ss2} &=& \infty 
      \end{eqnarray*}
\item <4> $r(t)=\frac{1}{2}at^2$ 时:
      \begin{eqnarray*}
      K_a &=& \lim_{s\rightarrow 0}s^2 G(s)_o(s) 
	  = 0 \\
      e_{ss3} &=& \infty
      \end{eqnarray*}
\end{itemize}
\end{frame}
\begin{frame}
\frametitle{I型系统($\nu=1$)}
\label{sec-3-2-7}

\begin{itemize}
\item <2->$r(t)=A$ 时:
      \begin{eqnarray*}
      K_p &=& \lim_{s\rightarrow 0}G_o(s) 
	  = \lim_{s\rightarrow 0}\frac{K\prod_{i=0}^m(\tau_i s+1)}{s\prod_{j=1}^{n-1}(\tau_j s+1)} 
	  = \infty \\
      e_{ss1} &=& \frac{1}{1+K_p}
	    = 0
      \end{eqnarray*}
      \mode<article>{无差系统.}
\item <3->$r(t)=vt$ 时:
      \begin{eqnarray*}
      K_v &=& \lim_{s\rightarrow 0}sG_o(s) 
	  = K \\
      e_{ss2} &=& \frac{v}{K_v} 
	      =\frac{v}{K}
      \end{eqnarray*}
\item <4->$r(t)=\frac{1}{2}at^2$ 时:
      \begin{eqnarray*}
      K_a &=& \lim_{s\rightarrow 0}s^2 G(s)_o(s) 
	  = 0 \\
      e_{ss3} &=& \infty
      \end{eqnarray*}
\end{itemize}
\end{frame}
\begin{frame}
\frametitle{II型系统($\nu=2$)}
\label{sec-3-2-8}

\begin{eqnarray*}
K_p & = & \infty\\
e_{ss1} &=& 0 \\
K_v & = & \infty \\
e_{ss2} &=& 0 \\
K_a &=& K \\
e_{ss3} &=& \frac{a}{K}
\end{eqnarray*}
\end{frame}
\begin{frame}
\frametitle{小结:}
\label{sec-3-2-9}

\begin{itemize}
\item <2->零型:
      \[e_{ss1}=\frac{A}{1+K},e_{ss2}=e_{ss3}=\infty\]
\item <3->I型:
      \[e_{ss1}=0,e_{ss2}=\frac{v}{K},e_{ss3}=\infty\]
\item <4->II型:
      \[e_{ss1}=e_{ss2}=0,e_{ss3}=\frac{a}{K}\]
\end{itemize}
\end{frame}
\begin{frame}
\frametitle{例:}
\label{sec-3-2-10}


若 $G(s)H(s) =\frac{10K_h}{s+1},K_h\in\{0.1,1\}$ ,求单位阶跃下的 $e_{ss}$ .

\mode<article>{解:}
\begin{itemize}

\item 解法1\\
\label{sec-3-2-10-1}%
零型系统, $r(t)=1,e_{ss}=\frac{1}{1+K_p}$

\begin{eqnarray*}
K_p &=  &\lim_{s\rightarrow 0}G(s)H(s) \\
    &=& 10K_h \\
    &=&
\begin{cases}
1  & K_h =0.1 \\
10 & K_h = 1
\end{cases}\\
e_{ss} &=&
\begin{cases}
0.5 & K_h=0.1 \\
\frac{1}{11} & K_h=1
\end{cases}
\end{eqnarray*}


\item 解法2:\\
\label{sec-3-2-10-2}%
\begin{eqnarray*}
e_{ss} &=& \lim_{s\rightarrow 0}s\Phi_e(s)R(s)\\
    &=&\lim_{s\rightarrow 0}s\frac{1}{1+G(s)H(s)}R(s)\\
    &=&\lim_{s\rightarrow 0}s\frac{s+1}{s+1+10K_h}\frac{1}{s}\\
    &=& \frac{1}{1+10K_h} \\
    &=&
\begin{cases}
0.5 & K_h=0.1 \\
\frac{1}{11} & K_h=1
\end{cases}
\end{eqnarray*}

\end{itemize} % ends low level
\end{frame}
\begin{frame}
\frametitle{例:  求 $r(t)=2+3t$ 时的 $e_{ss}$}
\label{sec-3-2-11}

\begin{psmatrix}[rowsep=0.4,colsep=0.5]
% 1    2   3  4   5   6    7       8   9
%           R*    E(s)    .------.
% R-->2/s+1-->o------>o-->| G(s) |--+--> C
%           _ ^     _ ^   '------'  |
%             |       |             |  
%             |       '----0.8s-----+
%             |                     |
%             '---------------------'
%        
%
% 1                        2                 3        
$R(s)$ &  \psframebox{$\frac{2}{s+1}$} & ${\hskip 1em}  $ & %
\pscirclebox[framesep=-0.2em]{$\times$} &$ $   & \pscirclebox[framesep=-0.2em]{$\times$} & %
\psframebox{$\frac{5}{s(5s+1)}$}   & \   & $C(s)$ \\
  &   &     &  & & & \psframebox{$ 0.8s $} &  \ &  \\
\\
%link
\ncline{->}{1,1}{1,2}
\ncline{->}{1,2}{1,4}
\naput{$R^{*}(s)$}
\ncline{->}{1,4}{1,6}
\naput{$E(s)$}
\ncline{->}{1,6}{1,7}
\ncline{->}{1,7}{1,9}
\ncline{2,8}{2,7}
\ncangle[angleA=180,angleB=-90,armA=0.5em,armB=1em]{->}{2,7}{1,6}
\naput[npos=2.3]{$-$}
\ncangles[angleA=180,angleB=-90,armA=0em,armB=4.5em]{->}{1,8}{1,4}
\naput[npos=3.6]{$-$}
\end{psmatrix}

解:

\begin{eqnarray*}
G(s) & = \frac{C(s)}{E(s)} \\
     &=& \frac{\frac{5}{s(5s+1)}}{1+\frac{4s}{s(5s+1)}} \\
    & = & \frac{5}{5s^2+5s} \\
    & = & \frac{1}{s(s+1)} 
\end{eqnarray*}
\end{frame}
\begin{frame}
\frametitle{例:计算稳态误差}
\label{sec-3-2-12}

\mode<article>{判断稳定性:}

\begin{eqnarray*}
\Phi(s) &=& \frac{C(s)}{R^{*}(s)}
        = \frac{1}{s(s+1)+1} \\
\Phi_e(s) &=& \frac{s(s+1)}{s(s+1)+1}
\end{eqnarray*}
系统稳定.

\begin{eqnarray*}
R(s) &=& \frac{2s+3}{s^2}\\
e_{ss} & = &\lim_{s\rightarrow 0}s\Phi_e(s)R^{*}(s) \\
       &=& \lim_{s\rightarrow 0}s\cdot\frac{s(s+1)}{s(s+1)+1}\cdot\frac{2}{s+1}\cdot\frac{2s+3}{s^2} \\
       &=& 6
\end{eqnarray*}
\end{frame}
\subsection{动态误差系数}
\label{sec-3-3}
\begin{frame}
\frametitle{动态误差系数}
\label{sec-3-3-1}

动态误差系数完整地描述系统稳态误差随时间变化的规律,而不是指误差信号随时间的变化规律,静态误差可看作动态误差的一个特例.

\begin{eqnarray*}
E(s) & = & \Phi_e(s)R(s)\\
\Phi_e(s) &=& \frac{E(s)}{R(s)}\\
         &=&\frac{1}{1+G(s)H(s)} \\
         &=& \frac{M(s)}{N(s)} 
\end{eqnarray*}
\end{frame}
\begin{frame}
\frametitle{在 $s=0$ 处展开,得:}
\label{sec-3-3-2}


\begin{eqnarray*}
\phi_e(s)  &=& \Phi_e(0)+\dot{\Phi}_e(0)s+\cdots+\frac{\Phi_e^{(n)}(0)s^n}{n!}+\cdots \\
E(s) & = & \Phi_e(0)R(s)+\dot{\Phi}_e(0)sR(s)+\cdots+\frac{\Phi_e^{(n)}(0)s^nR(s)}{n!}+\cdots \\
e_{ss}(t) & = & \Phi_e(0)r(t)+\dot{\Phi}_e(0)\dot{r}(t)+\cdots+\frac{\Phi_e^{(n)}(0)r^{(n)}(t)}{n!}+\cdots \\
          &= & \sum_{i=1}^{\infty}C_ir^{(i)}(t) ,\qquad
C_i = \frac{\Phi_e^{(i)}(0)}{i!}
\end{eqnarray*}

\begin{itemize}
\item 其中 $C_i$ 称为动态误差系数.
\begin{itemize}
\item $C_0$ 动态位置误差系数
\item $C_1$ 动态速度误差系数
\item $C_2$ 动态加速度误差系数
\end{itemize}
\end{itemize}
\end{frame}
\begin{frame}
\frametitle{动态误差系数示例:}
\label{sec-3-3-3}

\begin{itemize}
\item <2-> 零型系统 $r(t)=1$ 则 \[e_{ss}(t)=C_0 ,C_0=\frac{1}{1+K_p}\]
\item <3-> I型系统 $r(t)=t$ 则 \[e_{ss}(t)=C_0 t+C_1,C_0=0,C_1=\frac{1}{K_v}\]
\item <4-> II型系统 $r(t)=t$ 则 \[e_{ss}(t)=C_0 \frac{1}{2}at^2+C_1at+C_2 a,C_0=C_1=0,C_2=\frac{1}{K_a}\]
\end{itemize}
\end{frame}
\begin{frame}
\frametitle{讨论: $C_i$ 的计算}
\label{sec-3-3-4}

\begin{eqnarray*}
\Phi_e(s) &=& \frac{M(s)}{N(s)} \\
 & = & C_0+C_1s+C_2s^2+\cdots
\end{eqnarray*}
\end{frame}
\begin{frame}
\frametitle{例: $G(s)H(s)=\frac{1}{s(s+1)}$}
\label{sec-3-3-5}


综合除法:

\[
\begin{matrix}
\text{divident}      &      &   \text{divisor}  &    & \text{quotient} &   & \text{remainder} \\  
s^2+s             &\div  &    s^2+s+1     & \rightarrow &  s       &  & s^2+s-s(1+s+s^2) \\
-s^3             &\div  &   s^2+s+1       & \rightarrow  & -s^3     &   & -s^3-(-s^3)(1+s+s^2)\\
 s^4+s^5         &\div  &  s^ 2+s+1       & \rightarrow & s^4  &   & \cdots  \\
\cdots           &\div   &  s^2+s+1       &\rightarrow  & \cdots   &    &\cdots   
\end{matrix}
\]

得:
\begin{eqnarray*}
\Phi_e(s)  &=& s-s^3+s^4+\cdots
\end{eqnarray*}
\end{frame}
\begin{frame}
\frametitle{例: $G(s)H(s)=\frac{1}{s(s+1)}$}
\label{sec-3-3-6}


另一种写法:

\begin{eqnarray*}
\frac{s^2+s}{s^2+s+1} & = & s + \frac{-s^2+s-s(1+s+s^2)}{s^2+s+1}  \\
\frac{-s^3}{s^2+s+1}  & = & -s^3+\frac{-s^3-(-s^3)(1+s+s^2)}{s^2+s+1}\\
\frac{s^4+s^5}{s^2+s+1} &=& s^4 +\cdots \\
\cdots                  &=& \cdots \\
\Phi_e(s)               &=& s-s^3+s^4+\cdots
\end{eqnarray*}
\end{frame}
\begin{frame}
\frametitle{例:}
\label{sec-3-3-7}

单位负反馈系统开环传递函数: $G_o(s)=\frac{100}{s(0.s1+1)}$ 求输入信号为 $\sin(5 t)$ 时的稳态误差.

解:系统稳定,
\begin{eqnarray*}
r(t) &=& sin(\omega t),\omega=5 \\
E(s) &=& \Phi_e(s)R(s) \\
e_{ss}&=& \sum_{i=0}^{\infty}C_i r^{(i)} \\
\Phi_e(s)&=& \frac{1}{1+G_o(s)} \\
         &=& \frac{0.1s^2+s}{0.1s^2+s+100} 
\end{eqnarray*}
\end{frame}
\begin{frame}
\frametitle{解法1}
\label{sec-3-3-8}





\begin{itemize}
\item $\frac{0.1s^2+s}{0.1s^2+s+100}  =0.01s+\frac{0.1s^2+s-0.01s(0.1s^2+s+100)}{0.1s^2+s+100}$
\item $\frac{-10^{-3}s^3+0.09s^2}{0.1s^2+s+100} = 9\times 10^{-4}s^2+\frac{-10^{-3}s^3+0.09s^2-9\times 10^{-4}s^2(0.1s^2+s+100)}{0.1s^2+s+100}$
\item $\frac{-9\times 10^{-5}s^4-1.9\times 10^{-3}s^3}{0.1s^2+s+100}  = -1.9\times 10^{-5}s^3 + \cdots$
\end{itemize}
所以
\begin{itemize}
\item $\Phi_e(s) = 0+0.01s+9\times 10^{-4}s^2-1.9\times 10^{-5}s^3+\cdots$
\item $e_{ss}(t) = (C_0-C_2\omega^2+C_4\omega^4+\cdots)\sin(\omega t)+(C_1-C_3\omega^3+C_5\omega^5+\cdots)\cos(\omega t)$
\item $e_{ss}(t) = -0.055\cos(5t-249^{\circ})$
\end{itemize}
\end{frame}
\begin{frame}
\frametitle{解法2:}
\label{sec-3-3-9}

\begin{eqnarray*}
E(s) & = & \Phi_e(s)R(s) \\
     &=& \frac{s^2+10S}{s^2+10S+1000}\cdot\frac{5}{s^2+25}\\
     &=&\frac{-0.0498s-0.1115}{s^2+25}+\frac{as+b}{s^2+10s+1000}\\
e_{ss}(t)&=& -0.055\cos(5t-249^{\circ})+\Delta 
\end{eqnarray*}
其中: $\lim_{t\rightarrow\infty}\Delta = 0$
\end{frame}
\subsection{减小稳态误差的措施}
\label{sec-3-4}
\begin{frame}
\frametitle{减小 $e_{ss}$ 的措施}
\label{sec-3-4-1}

\begin{itemize}
\item <2->增大开环增益
\item <3->增加系统类型
\item <4->串级控制抑制扰动
\item <5->复合控制
\end{itemize}
\end{frame}
\section{示例}
\label{sec-4}
\subsection{例1}
\label{sec-4-1}
\begin{frame}
\frametitle{例1:速度反馈}
\label{sec-4-1-1}
\begin{itemize}

\item 结构图
\label{sec-4-1-1-1}%
\begin{psmatrix}[rowsep=0.4,colsep=0.5]
%         1    2   3  4          5    6    7
%                E(s)     .--------.
%         R-->o------>o-->|k/s(s+1)|--+--> C
%           _ ^     _ ^   '--------'  |
%             |       |               |  
%             |       '------As-------+
%             |                       |
%             '-----------------------'
%
% 1                        2                          3        
$R(s)$  & \pscirclebox[framesep=-0.2em]{$\times$} & {\hskip 1em}    & \pscirclebox[framesep=-0.2em]{$\times$} & %
\psframebox{$\frac{K}{s(s+1)}$}   & \   & $C(s)$ \\
  &   &     &  & \psframebox{$ As $} &  \ &    \\
\\
%link
\ncline{->}{1,1}{1,2}
\ncline{->}{1,2}{1,4}
\naput{$E(s)$}
\ncline{->}{1,4}{1,5}
\ncline{->}{1,5}{1,7}
\ncline{2,6}{2,5}
\ncangle[angleA=180,angleB=-90,armA=0.5em,armB=1em]{->}{2,5}{1,4}
\naput[npos=2.3]{$-$}
\ncangles[angleA=180,angleB=-90,armA=0em,armB=3.5em]{->}{1,6}{1,2}
\naput[npos=3.6]{$-$}
\end{psmatrix}

As为系统的速度反馈,求解:
\begin{itemize}
\item $A=\{0.8,0\}$ 时系统稳态误差及动态品质指标( $K=1$ ,求 $e_{ss}$ 时 $R(s)=\frac{1}{s^2}$ )
\item 若要求系统的 $\sigma\%=20\%$ , $t_s\leq 1s$ ,确定 $A,K$ 的值
\end{itemize}


\item 解:\\
\label{sec-4-1-1-2}%
\begin{eqnarray*}
G(s)&=& \frac{K}{s(s+1)+KAs}\\
    &=&  \frac{K}{s^2+(KA+1)s} \\
\Phi(s) &=& \frac{K}{s^2+(KA+1)s+K} 
\end{eqnarray*}
系统稳定,为I型系统.

\end{itemize} % ends low level
\end{frame}
\begin{frame}
\frametitle{例1(续):计算稳态误差:}
\label{sec-4-1-2}

 $R(s)=\frac{1}{s^2}$ 时,
\begin{eqnarray*}
e_{ss} &=& \frac{1}{Kv} \\
K_v &=& \lim_{s\rightarrow 0}sG(s) \\
    &=& \lim_{s\rightarrow 0}\frac{K}{s+KA+1} \\
    &=& \frac{K}{KA+1} \\
\end{eqnarray*}

\begin{itemize}
\item $K=1,A=0.8$ 时, $e_{ss}=\frac{KA+1}{K}=1.8$
\item $K=1,A=0$ 时, $e_{ss}=\frac{KA+1}{K}=1$
\end{itemize}
\end{frame}
\begin{frame}
\frametitle{例1(续):计算动态品质:}
\label{sec-4-1-3}


\begin{eqnarray*}
\Phi(s) &=& \frac{\omega_n^2}{s^2+2\xi\omega_n s+\omega_n^2} \\
        &=& \frac{K}{s^2+(KA+1)s+K} \\
\sigma\% &=& e^{-\frac{\xi\pi}{\sqrt{1-\xi^2}}} \\
t_s &=& \frac{3.5}{\xi\omega_n}
\end{eqnarray*}

\begin{itemize}
\item <2-> $K=1,A=0$ 时, $\omega_n=1,\xi=0.5,\sigma\%=16.3\%,t_s=7$
\item <3-> $K=1,A=0.8$ 时, $\omega_n=1,\xi=0.9,\sigma\%=0.15\%,t_s=3.89$
\end{itemize}
\end{frame}
\begin{frame}
\frametitle{例1(续):确定 $A,K$ 值}
\label{sec-4-1-4}



\begin{eqnarray*}
e^{\frac{-\xi\pi}{\sqrt{1-\xi^2}}} &=&20\% \\
\frac{3.5}{\xi\omega_n} &=& 1
\end{eqnarray*}

得: $\xi=0.456,\omega_n=7.68$

\begin{eqnarray*}
2\xi\omega_n & = & AK+1\\
K &=& \omega_n^2
\end{eqnarray*}

得: $A=0.102,K=58.9$
\end{frame}
\subsection{例2}
\label{sec-4-2}
\begin{frame}
\frametitle{例2:比例-微分控制}
\label{sec-4-2-1}

\begin{psmatrix}[rowsep=0.4,colsep=0.5]
%         1    2   3  4      5                6          7    8
%                E(s)                          
%         R-->o---+--1------>o--> wn2/s(s+2\x\wn2)--+--> C
%           _ ^   |           ^                      |
%             |   !--Tds------!                      |  
%             |                                      |
%             '--------------------------------------'
%
% 1                        2                          3        
$R(s)$  & \pscirclebox[framesep=-0.2em]{$\times$} & {\hskip 1em} & \psframebox{$1$} 
 & \pscirclebox[framesep=-0.2em]{$\times$} & %
\psframebox{$\frac{\omega_n^2}{s(s+2\xi\omega_n)}$}   & \   & $C(s)$ \\
    &     &  & \psframebox{$ T_d s $} &  \ &   \\
\\
%link
\ncline{->}{1,1}{1,2}
\ncline{->}{1,2}{1,4}
\naput{$E(s)$}
\ncline{->}{1,4}{1,5}
\ncline{->}{1,5}{1,6}
\ncline{->}{1,6}{1,8}
\ncangle[angleA=0,angleB=180,armA=0.5em,armB=1em]{->}{1,3}{2,4}
\ncangle[angleA=0,angleB=-90,armA=0.5em,armB=1em]{->}{2,4}{1,5}
\naput[npos=2.3]{$+$}
\ncangles[angleA=180,angleB=-90,armA=0em,armB=3.5em]{->}{1,7}{1,2}
\naput[npos=3.6]{$-$}
\end{psmatrix}

比较原系统与PD控制系统的稳态性能与动态性能

解:

\begin{eqnarray*}
G(s) & = &\frac{\omega_n^2(1+T_d s)}{s^2+2\xi\omega_n s} \\
\Phi(s) &=&\frac{\omega_n^2(1+T_d s)}{s^2+(2\xi+T_d\omega_n)\omega_n s+\omega_n^2}
\end{eqnarray*}
\end{frame}
\begin{frame}
\frametitle{例2(续):稳态误差}
\label{sec-4-2-2}

系统稳定,为I型系统.
\begin{eqnarray*}
e_{ss} & =& \frac{1}{K_v} \\
K_v &=& \lim_{s\rightarrow 0}sG(s) \\
   &=& \lim_{s\rightarrow 0}\frac{\omega_n^2(1+T_d s)}{s+2\xi\omega_n} \\
   &=&\frac{\omega_n}{2\xi}
\end{eqnarray*}

$e_{ss}$ 与 $T_d$ 无关.
\end{frame}
\begin{frame}
\frametitle{例2(续):比较动态性能}
\label{sec-4-2-3}

考虑如下三个系统:
\begin{eqnarray*}
\Phi_1(s) &=&\frac{\omega_n^2}{s^2+2\xi\omega_n s+\omega_n^2}\\
\Phi_2(s) &=&\frac{\omega_n^2(1+T_d s) }{s^2+(2\xi+T_d\omega_n)\omega_n s+\omega_n^2}\\
\Phi_3(s) &=&\frac{\omega_n^2}{s^2+(2\xi+T_d\omega_n)\omega_n s+\omega_n^2}
\end{eqnarray*}           

\begin{itemize}
\item <2->与系统1相比,系统3的阻尼比较大,两者的自然频率相同.
\item <3->因为:
        \begin{eqnarray*}
        \Phi_2(s) & = & (1+T_d s)\Phi_3(s) \\
        c_2(t) &=& c_3(t)+T_d\frac{dc_3(t)}{dt}
        \end{eqnarray*}
       由于存在微分作用,因此对高频噪声有放大作用.
\end{itemize}
\end{frame}

\end{document}
