% Created 2015-09-08 星期二 17:43
\documentclass{beamer}
\usepackage{fixltx2e}
\usepackage{graphicx}
\usepackage{longtable}
\usepackage{float}
\usepackage{wrapfig}
\usepackage{soul}
\usepackage{textcomp}
\usepackage{marvosym}
\usepackage{wasysym}
\usepackage{latexsym}
\usepackage{amssymb}
\usepackage{hyperref}
\tolerance=1000
\usepackage{etex}
\usepackage{amsmath}
\usepackage{pstricks}
\usepackage{pgfplots}
\usepackage{tikz}
\usepackage[europeanresistors,americaninductors]{circuitikz}
\usepackage{colortbl}
\usepackage{yfonts}
\usetikzlibrary{shapes,arrows}
\usetikzlibrary{positioning}
\usetikzlibrary{arrows,shapes}
\usetikzlibrary{intersections}
\usetikzlibrary{calc,patterns,decorations.pathmorphing,decorations.markings}
\usepackage[BoldFont,SlantFont,CJKchecksingle]{xeCJK}
\setCJKmainfont[BoldFont=Evermore Hei]{Evermore Kai}
\setCJKmonofont{Evermore Kai}
\usepackage{pst-node}
\usepackage{pst-plot}
\psset{unit=5mm}
\mode<beamer>{\usetheme{Frankfurt}}
\mode<beamer>{\usecolortheme{dove}}
\mode<article>{\hypersetup{colorlinks=true,pdfborder={0 0 0}}}
\mode<beamer>{\AtBeginSection[]{\begin{frame}<beamer>\frametitle{Topic}\tableofcontents[currentsection]\end{frame}}}
\setbeamercovered{transparent}
\subtitle{用微分方程描述的数学模型}
\mode<article>{\renewcommand{\labelitemii}{$\cdot$}}
\providecommand{\alert}[1]{\textbf{#1}}

\title{自动控制系统的数学模型}
\author{}
\date{}
\hypersetup{
  pdfkeywords={},
  pdfsubject={},
  pdfcreator={Emacs Org-mode version 7.9.3f}}

\begin{document}

\maketitle

\begin{frame}
\frametitle{Outline}
\setcounter{tocdepth}{3}
\tableofcontents
\end{frame}














\section{简介}
\label{sec-1}
\begin{frame}
\frametitle{数学模型分类}
\label{sec-1-1}

\begin{itemize}
\item <2->数学模型:描述系统内部物理量/变量之间关系的数学表达式
\item <3->数学模型种类:
\begin{itemize}
\item <4->时域(t):微分方程、差分方程、状态方程
\item <5->复域(s):传递函数、结构图
\item <6->频域(w):频率特性
\end{itemize}
\end{itemize}
\end{frame}
\begin{frame}
\frametitle{建模步骤}
\label{sec-1-2}

\begin{enumerate}
\item <2->列出系统的输入变量、输出变量
\item <3->列写系统的原始方程
\item <4->对原始方程进行线性化
\item <5->消去中间变量,得到仅有输入/输出变量的微分方程
\end{enumerate}
\end{frame}
\begin{frame}
\frametitle{线性化原理与方法}
\label{sec-1-3}

原理:在系统工作点处,将非线性函数展开成泰勒级数,忽略高次项,得到线性化方程
\begin{itemize}
\item <2->$y=f(x)$ 在 $x_0$ 处展开
\begin{itemize}
\item <3->$y =f(x_0)+f'(x_0)(x-x_0)+\cdots$
\item <3->$y \approx   y_0 + k(x-x_0)$
\end{itemize}
\item <4->$y=f(x_1,x_2)$ 在 $(x_{10},x_{20})$ 处展开
\begin{itemize}
\item <5->$y = f(x_{10},x_{20})+f_{x1}(x_{10},x_{20})(x-x_{10})+f_{x2}(x_{10},x_{20})(x-x_{20})+\cdots$
\item <5->$y = y_0+K_1(x-x_{10})+K_2(x-x_{20})$
\end{itemize}
\item <6-> 增量化方程
\begin{itemize}
\item <7-> $dY=y-y_0$
\item <7-> $dX=x-x_0$
\item <8-> $dY=K dX$
\end{itemize}
\end{itemize}
\end{frame}
\section{线性微分方程与线性系统}
\label{sec-2}
\begin{frame}
\frametitle{线性微分方程的解}
\label{sec-2-1}

  解=特解+通解

\begin{itemize}
\item <2->通解由微分方程的特征根决定
\begin{itemize}
\item <3->微分方程: $a_0 x''(t)+a_1 x'(t)+a_2 x(t) =b$
\item <4->特征方程:  $a_0s^2+a_1 s + a_2=0$
\end{itemize}
\item <5-> n阶方程有n个互异特征根 $\lambda_1,...,\lambda_n$ 时,通解为
     $y(t)=c_1e^{\lambda_1 t}+...+c_ne^{\lambda_n t}$
\item <6-> 若有重根  $\lambda=\lambda_1=\lambda_2$,则
     $y(t)=c_1 t e^{\lambda t}+c_2e^{\lambda t}+...+c_ne^{\lambda_n t}$
\end{itemize}
\end{frame}
\begin{frame}
\frametitle{线性系统的定义与特点}
\label{sec-2-2}

\begin{itemize}
\item <2->条件:
\begin{itemize}
\item 可加性: 若 $r_1(t)->c_1(t)$, $r_2(t)->c_2(t)$ 则: $r_1(t)+r_2(t)->c_1(t)+c_2(t)$
\item 齐次性: $r_1(t)->c_1(t)$ 则: $a r_1(t) -> a c_1(t)$
\end{itemize}
\item <3-> 微积分特性: $r_1(t)\rightarrow c_1(t)$  则  $\int r_1(t) \rightarrow \int c_1(t), r_1(t)'\rightarrow c_1(t)'$
\end{itemize}
\end{frame}
\section{机械系统示例(因果关系:力与位移的关系)}
\label{sec-3}
\begin{frame}
\frametitle{机械系统示例1}
\label{sec-3-1}
\begin{columns}
\begin{column}{0.5\textwidth}
\begin{block}<1->{弹簧-质量-阻尼系统}
\label{sec-3-1-1}


\begin{tikzpicture}[every node/.style={outer sep=0pt,thick}]
%#+begin_example
%  \|                         (F)->       (-fv)    |/
%  \|        (-kY)            .---.       ----.    |/
%  \|-----'\/\/\/\/\/\/\/'----+ m +--------+] +----|/
%  \|                         '---' (Y)   ----'    |/
%  \|==============================================|/
%#+end_example

\tikzstyle{spring}=[thick,decorate,decoration={zigzag,pre length=0.3cm,post length=0.3cm,segment length=6}]
\tikzstyle{damper}=[thick,decoration={markings,  
  mark connection node=dmp,
  mark=at position 0.5 with 
  {
    \node (dmp) [thick,inner sep=0pt,transform shape,rotate=-90,minimum width=15pt,minimum height=3pt,draw=none] {};
    \draw [thick] ($(dmp.north east)+(2pt,0)$) -- (dmp.south east) -- (dmp.south west) -- ($(dmp.north west)+(2pt,0)$);
    \draw [thick] ($(dmp.north)+(0,-5pt)$) -- ($(dmp.north)+(0,5pt)$);
  }
}, decorate]
%\tikzstyle{ground}=[fill,pattern=north east lines,draw=none,minimum width=0.75cm,minimum height=0.3cm]
\tikzstyle{ground}=[draw=none,minimum width=0.75cm,minimum height=0.3cm]

\node (M) [draw,minimum width=1cm, minimum height=2.5cm] {$m$};

\node (ground) [draw,ground,anchor=north,yshift=-0.25cm,minimum width=1.5cm] at (M.south) {};
\draw (ground.north east) -- (ground.north west);
\draw [thick] (M.south west) ++ (0.2cm,-0.125cm) circle (0.125cm)  (M.south east) ++ (-0.2cm,-0.125cm) circle (0.125cm);
\draw [->,red,thick] (M.north east)++(0,-1em)-- + (3em,0);
\draw [red,ultra thick] (M.north east)++(2em,-1em) node[above] {$F$};

\node (wall) [ground, rotate=-90, minimum width=3cm,yshift=-3cm] {};
\draw (wall.north east) -- (wall.north west);

\draw [spring] (wall.170) -- ($(M.north west)!(wall.170)!(M.south west)$);
\draw [damper] (wall.10) -- ($(M.north west)!(wall.10)!(M.south west)$);

\draw [red,-latex,ultra thick] (M.east) ++ (0.2cm,0) -- +(1cm,0);
\draw [red,thick] (M.east)++(0.5cm,0) node[above] {$Y$};

\end{tikzpicture}
\end{block}
\end{column}
\begin{column}{0.5\textwidth}
\begin{block}<2->{解:}
\label{sec-3-1-2}


\begin{itemize}
\item 输入: $F(t)$ , 输出: $Y(t)$
\item 原始方程: $F- k Y(t) -f v = ma$
\item 消去中间变量:
\begin{itemize}
\item $v = Y'(t)$
\item $a = v'$
\item $mY"+fY'+kY=F$
\end{itemize}
\end{itemize}
\end{block}
\end{column}
\end{columns}
\end{frame}
\section{电气系统示例}
\label{sec-4}
\begin{frame}
\frametitle{电气及机电系统示例1}
\label{sec-4-1}
\begin{columns}
\begin{column}{0.5\textwidth}
\begin{block}<1->{电阻、电容网絡}
\label{sec-4-1-1}


\begin{circuitikz}[american voltages]
%#+begin_example
%             C  (I_1) -->
%        .----||---.
%        |         |
%  o-----+--[R_1]--+--+-----o
%  +       (I_2) -->  |
%                     |
%  (E_r)      (I) | [R_2] (E_c)
%                 v   |
%  -                  |
%  o------------------+-----o
%#+end_example
\draw
  % rotor circuit
  (0,0) to  [short, o-o] (5,0)
  to [open, v^>=$E_c$,-o](5,3)
  to [short] (3,3)
  to [R, l=$R_2$,i_={$I$}] (3,0)

  (0,0) to [open, v>=$E_r$,-o] (0,3)
  to [R,l=$R_1$ ,i_={$I_2$}] (3,3)

  (0.5,3) to [short] (0.5,4.5) to [C, l=$C$ ,i_={$I_1$}] (3,4.5) to [short] (3,3);
\end{circuitikz}
\end{block}
\end{column}
\begin{column}{0.5\textwidth}
\begin{block}<2->{解:}
\label{sec-4-1-2}


\begin{itemize}
\item 输入:$E_r$ ,输出 $E_c$
\item 原始方程:
\begin{itemize}
\item $I=I_1+I_2$
\item $E_c=R_2 I$
\item $E_r=R_1 I_2+E_c$
\item $C (E_r-E_c)' = I_1$
\end{itemize}
\item 消去I,I$_1$,I$_2$得:
\begin{itemize}
\item $R_1 C E_c'+\frac{R_1+R_2}{R_2} E_c = R_1 C E_r'+ E_r$
\end{itemize}
\end{itemize}
\end{block}
\end{column}
\end{columns}
\end{frame}
\begin{frame}
\frametitle{电气及机电系统示例2}
\label{sec-4-2}
\begin{columns}
\begin{column}{0.5\textwidth}
\begin{block}<1->{电阻、电容网络}
\label{sec-4-2-1}


\begin{circuitikz}[american voltages]
%#+begin_example
%  
%  o--------[R]-------+-----o
%  +       (I) -->    |
%                     |
%  (U_r)          C ----- 
%                   ----- U_c
%                     |
%  -                  |
%  o------------------+-----o
%#+end_example

\draw
  (0,0) to  [short, o-o] (5,0)
  to [open, v^>=$U_c$,-o](5,3)
  to [short] (3,3)
  to [C, l=$C$] (3,0)

  (0,0) to [open, v>=$U_r$,-o] (0,3)
  to [R,l=$R$ ,i_={$I$}] (3,3);

\end{circuitikz}
\end{block}
\end{column}
\begin{column}{0.5\textwidth}
\begin{block}<2->{解:}
\label{sec-4-2-2}


\begin{itemize}
\item 输入: $U_r$ ,输出: $U_c$
\item $U_r=R I +U_c$ , $C U_c' = I$
\item 消去 $I$ ,  $RC U_c' +U_c = U_r$
\end{itemize}
    
\end{block}
\end{column}
\end{columns}
\end{frame}
\section{非线性系统示例}
\label{sec-5}
\begin{frame}[t]
\frametitle{倒立摆系统线性化模型}
\label{sec-5-1}
\begin{columns}
\begin{column}{0.23\textwidth}
\begin{block}<1->{倒立摆}
\label{sec-5-1-1}


\begin{tikzpicture}[every node/.style={outer sep=0pt,thick}]
%#+begin_example
%                   ^          //
%                   | \theta  //
%                   |        //
%                   |       //
%                   |      //
%                   |     //
%                   |    //
%                   |   /+
%                   |  //| mg
%                 N | // v
%                   |//  -----> H
%                   //
%            .------+------.
%            |             |
%            |      +------------>  F
%            |      |      |
%            '------|------'
%                O  |   O
%  =================|==================
%                   | Mg
%                   v 
%#+end_example
%\tikzstyle{ground}=[draw=none,minimum width=0.75cm,minimum height=0.3cm]
\node (M) [draw,minimum width=5em, minimum height=3em] {};
%\node (ground) [draw,anchor=north,yshift=-0.25cm,minimum width=10em] at (M.south) {};
%\draw (ground.north east) -- (ground.north west);
\draw [thick] (M.south west) ++ (0.2cm,-0.125cm) circle (0.125cm)  (M.south east) ++ (-0.2cm,-0.125cm) circle (0.125cm);
\draw [thick] (M.south west) ++ (-0.2cm,-0.25cm) --  ($(M.south east) + (0.2cm,-0.25cm)$);
\draw [thick] (M.north) circle (0.3em) ;
%\draw [->,red,thick] (M.north east)++(0,-1em)-- + (3em,0));
\draw [->,red,thick] (M.center)-- + (3em,0);
\draw [red,ultra thick] (M.center)++(3em,0) node[below] {$F$};
\draw [->,red,thick] (M.center)-- + (0,-5em);
\draw [red,ultra thick] (M.center)++(0,-5em) node[right] {$Mg$};
\node (stick) [anchor=west,draw, rotate=60, minimum height=0.1cm,minimum width=7em,] at (M.north) {};
\draw [->,red,thick] (stick.center)-- + (0,-2.3em);
\draw [red,ultra thick] (stick.center)++(0,-2.3em) node[right] {$mg$};
\draw [red,thick,<->] (stick.west)++(0,5em)  arc (90:60:5em) ;
\draw [red] (stick.west)++(75:5em) node[above] {$\theta$};
\draw [red,dashed,thick] (stick.west)--+(0,7em);
\draw [->,red,thick] (stick.west)--+(0,3em) node[left] {$N$};
\draw [->,red,thick] (stick.west)--+(3em,0) node[below] {$H$};
%\draw (wall.north east) -- (wall.north west);
%\draw [red,-latex,ultra thick] (M.east) ++ (0.2cm,0) -- +(1cm,0);
%\draw [red,thick] (M.east)++(0.5cm,0) node[above] {$Y$};
\end{tikzpicture}

\mode<article>{
\begin{itemize}
\item $M$: 小车质量
\item $m$:杆质量
\item $l$:摆杆半长
\item $J$:摆杆转动惯量(绕质心)
\end{itemize}
}
\end{block}
\end{column}
\begin{column}{0.77\textwidth}
\begin{block}<2->{解:}
\label{sec-5-1-2}


\begin{itemize}
\item 输入$F$,输出 $\theta,r$
\item 原始方程:
\begin{itemize}
\item 小车水平方向: $Mr''=F-H$
\item 杆水平方向: $m(r+l sin(\theta))'' = H$
\item 杆竖直方向: $m (l cos(\theta))'' = N-mg$
\item 杆转动: $J\theta''=N l sin(\theta) - H l cos(\theta)$
\end{itemize}
\item 整理后:
\begin{itemize}
\item $(M+m)r''+ml\cos(\theta)\theta''-ml\sin(\theta)(\theta')^2=F$
\item $ml\cos(\theta)r''+(J+m l^2)\theta''=mgl\sin(\theta)$
\end{itemize}
\item 线性化 ( $\theta\rightarrow 0,\sin(\theta)\approx \theta,cos(\theta)\approx 1$ )
\begin{itemize}
\item $(M+m)r''+ml\theta'' = F$
\item $ml r'' +(J+ml^2)\theta''=mgl\theta$
\end{itemize}
\end{itemize}
\end{block}
\end{column}
\end{columns}
\end{frame}

\end{document}
