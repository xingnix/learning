\documentclass{article}
\usepackage[english]{babel}
\usepackage{amsmath}

%%%%%%%%%% Start TeXmacs macros
\catcode`\|=\active \def|{
\fontencoding{T1}\selectfont\symbol{124}\fontencoding{\encodingdefault}}
\newcommand{\nin}{\not\in}
\newcommand{\tmaffiliation}[1]{\thanks{\textit{Affiliation:} #1}}
\newcommand{\tmop}[1]{\ensuremath{\operatorname{#1}}}
%%%%%%%%%% End TeXmacs macros

\begin{document}

\title{Notes on VC Dimension of Sin/Cos Function}

\author{
  Xing Chao
  \tmaffiliation{xingnix@live.com}
}

\maketitle

\section{Problem}

can be assigned to arbitrary values according to VC dimension definition. The
problem is how to choose these values to meat the all kinds of dichotomy with
sine or cosine funtion?

Consider three point, $x_{1} =0,x_{2} =2,x_{3} =3$ , and the corrysponding
category label $c ( x_{i} ) \in \{ 0,1 \}$. It is eany to draw a sine/cosine
function cure to shatter the three ponts:
\begin{eqnarray*}
  c ( x_{1} ) =1,c ( x_{2} ) =0,c ( x_{3} ) =0 & \rightarrow & \cos \left( 2
  \pi   \frac{x}{5} \right) -1+ \Delta\\
  c ( x_{1} ) =1,c ( x_{2} ) =1,c ( x_{3} ) =0 & \rightarrow & \cos \left( 2
  \pi   \frac{x}{2} \right) -1+ \Delta\\
  c ( x_{1} ) =1,c ( x_{2} ) =0,c ( x_{3} ) =1 & \rightarrow & \cos \left( 2
  \pi   \frac{x}{3} \right) -1+ \Delta
\end{eqnarray*}
where $\Delta$ is a small number.

\section{Basic Idea}

In order to extend the previous example of three points to more points, let's
choose the value of three point again to see if there is any clue. If we can
choose period of cosine function to be exactly divisible by some $x_{i} \in \{
x_{i} |c ( x_{i} ) =1 \}$, and not exactly divisible by other $x_{i} \in \{
x_{i} |c ( x_{i} ) =0 \}$, the cosine function can be used to classify the
points according to $c ( x_{i} )$. It is possible to shatter these points if
we choose $x_{i}$ carefully to make such period exist for evry category label
$c ( x_{i} )$.

Let
\begin{eqnarray*}
  x_{1} & = & 2 \times 7 \times 13\\
  x_{2} & = & 3 \times 7 \times 11\\
  x_{3} & = & 5 \times 11 \times 13
\end{eqnarray*}
then
\begin{eqnarray*}
  c ( x_{1} ) =1,c ( x_{2} ) =0,c ( x_{3} ) =0 & \rightarrow & \cos \left( 2
  \pi   \frac{x}{2} \right) -1+ \Delta\\
  c ( x_{1} ) =1,c ( x_{2} ) =1,c ( x_{3} ) =0 & \rightarrow & \cos \left( 2
  \pi   \frac{x}{7} \right) -1+ \Delta\\
  c ( x_{1} ) =1,c ( x_{2} ) =0,c ( x_{3} ) =1 & \rightarrow & \cos \left( 2
  \pi   \frac{x}{13} \right) -1+ \Delta
\end{eqnarray*}

\section{Algorithm}

Each $x_{i}$ is a production of prime numbers and for each subset $\{ x_{i} |c
( x_{i} ) =1 \}$ of $\{ x_{i} \}$,there is a prime number in each $x_{i} \in
\{ x_{i} |c ( x_{i} ) =1 \}$ and not in $x_{i} \nin \{ x_{i} |c ( x_{i} ) =1
\}$. That prime number is the period of the cosine function.

Now it is time to find a way to generate $x_{i}$. First let's count the amount
of prime numbers we need. Each subset of $\{ x_{i} \}$ need one prime number
and there is $2^{N}$ subset, where $N= | \{ x_{i} \} |$ is the number of
elements in $\{ x_{i} \}$. The number ``2'' is a clue that binary number may
give some help. A dichotomy can be represented by $c ( x_{1} ) c ( x_{2} )
\cdots c ( x_{n} )$, where each $c ( x_{i} )$ is 0 or 1 and it can be used to
represent the index of \ the dichotomy. Then a corresponding prime is choosen
to put in $x_{i}$ whose $c ( x_{i} ) =1$ in the binary number.

Example
\begin{eqnarray*}
  c ( x_{1} ) =1,c ( x_{2} ) =0,c ( x_{3} ) =0 & \rightarrow & 001\\
  &  & x_{1} = \tmop{prod} ( [ \tmop{prime} ( 1 )   ] ) = \tmop{prod} ( [ 2 ]
  )\\
  &  & x_{2} = [ ]\\
  &  & x_{3} = [ ]\\
  c ( x_{1} ) =0,c ( x_{2} ) =1,c ( x_{3} ) =0 & \rightarrow & 010\\
  &  & x_{1} = \tmop{prod} ( [ 2  ] )\\
  &  & x_{2} = \tmop{prod} ( [ \tmop{prime} ( 2 )   ] ) = \tmop{prod} ( [ 3 ]
  )\\
  &  & x_{3} = [ ]\\
  c ( x_{1} ) =1,c ( x_{2} ) =1,c ( x_{3} ) =0 & \rightarrow & 011\\
  &  & x_{1} = \tmop{prod} ( [ 2, \tmop{prime} ( 3 ) ] ) = \tmop{prod} ( [
  2,5 ] )\\
  &  & x_{2} = \tmop{prod} ( [ 3, \tmop{prime} ( 3 ) ] ) = \tmop{prod} ( [
  3,5 ] )\\
  &  & x_{3} = [ ]\\
  c ( x_{1} ) =0,c ( x_{2} ) =0,c ( x_{3} ) =1 & \rightarrow & 100\\
  &  & x_{1} = \tmop{prod} ( [ 2,5 ] )\\
  &  & x_{2} = \tmop{prod} ( [ 3,5 ] )\\
  &  & x_{3} = \tmop{prod} ( [ \tmop{prime} ( 4 )   ] ) = \tmop{prod} ( [ 7 ]
  )\\
  c ( x_{1} ) =1,c ( x_{2} ) =0,c ( x_{3} ) =1 & \rightarrow & 101\\
  &  & x_{1} = \tmop{prod} ( [ 2,5, \tmop{prime} ( 5 ) ] ) = \tmop{prod} ( [
  2,5,11 ] )\\
  &  & x_{2} = \tmop{prod} ( [ 3,5 ] )\\
  &  & x_{3} = \tmop{prod} ( [ 7, \tmop{prime} ( 5 )   ] ) = \tmop{prod} ( [
  7,11 ] )\\
  c ( x_{1} ) =0,c ( x_{2} ) =1,c ( x_{3} ) =1 & \rightarrow & 101\\
  &  & x_{1} = \tmop{prod} ( [ 2,5,11 ] )\\
  &  & x_{2} = \tmop{prod} ( [ 3,5, \tmop{prime} ( 7 ) ] ) = \tmop{prod} ( [
  3,5,13 ] )\\
  &  & x_{3} = \tmop{prod} ( [ 7,11, \tmop{prime} ( 7 )   ] ) = \tmop{prod} (
  [ 7,11,13 ] )
\end{eqnarray*}
then
\begin{eqnarray*}
  x_{1} & = & 2 \times 5 \times 11\\
  x_{2} & = & 3 \times 5 \times 13\\
  x_{3} & = & 7 \times 11 \times 13
\end{eqnarray*}
For each dichotomy a period can be choosen according to the binary number of
$c ( x_{1} ) c ( x_{2} ) \cdots c ( x_{n} )$
\[ \tmop{period} = \tmop{prime} ( c ( x_{1} ) c ( x_{2} ) \cdots c ( x_{n} ) )
\]


\end{document}
