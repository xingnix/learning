% Created 2016-08-20 Sat 15:29
\documentclass{article}
\usepackage{fixltx2e}
\usepackage{graphicx}
\usepackage{longtable}
\usepackage{float}
\usepackage{wrapfig}
\usepackage{soul}
\usepackage{textcomp}
\usepackage{marvosym}
\usepackage{wasysym}
\usepackage{latexsym}
\usepackage{amssymb}
\usepackage{hyperref}
\tolerance=1000
\usepackage{etex}
\usepackage{amsmath}
\usepackage{pstricks}
\usepackage{pgfplots}
\usepackage{tikz}
\usepackage[europeanresistors,americaninductors]{circuitikz}
\usepackage{colortbl}
\usepackage{yfonts}
\usetikzlibrary{shapes,arrows}
\usetikzlibrary{positioning}
\usetikzlibrary{arrows,shapes}
\usetikzlibrary{intersections}
\usetikzlibrary{calc,patterns,decorations.pathmorphing,decorations.markings}
\usepackage[BoldFont,SlantFont,CJKchecksingle]{xeCJK}
\setCJKmainfont[BoldFont=Evermore Hei]{Evermore Kai}
\setCJKmonofont{Evermore Kai}
\usepackage{pst-node}
\usepackage{pst-plot}
\psset{unit=5mm}
\usepackage{beamerarticle}
\mode<beamer>{\usetheme{Frankfurt}}
\mode<beamer>{\usecolortheme{dove}}
\mode<article>{\hypersetup{colorlinks=true,pdfborder={0 0 0}}}
\AtBeginSection[]{\begin{frame}<beamer>\frametitle{Topic}\tableofcontents[currentsection]\end{frame}}
\setbeamercovered{transparent}
\providecommand{\alert}[1]{\textbf{#1}}

\title{人工神经网络}
\author{}
\date{}
\hypersetup{
  pdfkeywords={},
  pdfsubject={},
  pdfcreator={Emacs Org-mode version 7.9.3f}}

\begin{document}

\maketitle

\begin{frame}
\frametitle{Outline}
\setcounter{tocdepth}{3}
\tableofcontents
\end{frame}










\section{简介}
\label{sec-1}




人工神经网络(Artificial Neural Networks——ANNs)提供了一种普遍而且实用的方法,来从样例中学习值为实数、离散或向量的函数。像反向传播(BackPropagation)这样的算法使用梯度下降来调节网络参数以最佳拟合由输入-输出对组成的训练集合。ANN学习对于训练数据中的错误鲁棒性很好,且已经成功地应用到很多领域,例如视觉场景分析(interpreting visual scenes)、语音识别、以及机器人控制等。
\subsection{示例}
\label{sec-1-1}


\href{file:///run/media/xing/000751B6000C49A9/github/learning/machinelearning/beamer/4/image/alvinn1.gif}{./image/alvinn1.gif}
\href{file:///run/media/xing/000751B6000C49A9/github/learning/machinelearning/beamer/4/image/alvinn2.gif}{./image/alvinn2.gif}
\href{file:///run/media/xing/000751B6000C49A9/github/learning/machinelearning/beamer/4/image/alvinn3.gif}{./image/alvinn3.gif}

Pomerleau(1993)的ALVINN系统是ANN学习的一个典型实例,这个系统使用一个学习到的ANN以正常的速度在高速公路上驾驶汽车。ANN的输入是一个3032像素的网格,像素的亮度来自一个安装在车辆上的前向摄像机。ANN的输出是车辆行进的方向。这个ANN通过观察人类驾驶时的操纵命令进行训练,训练过程大约5分钟。ALVINN用学习到的网络在高速公路上以70英里时速成功地驾驶了90Pomerleau(1993)的ALVINN系统是ANN学习的一个典型实例,这个系统使用一个学习到的ANN以正常的速度在高速公路上驾驶汽车。ANN的输入是一个3032像素的网格,像素的亮度来自一个安装在车辆上的前向摄像机。ANN的输出是车辆行进的方向。这个ANN通过观察人类驾驶时的操纵命令进行训练,训练过程大约5分钟。ALVINN用学习到的网络在高速公路上以70英里时速成功地驾驶了90英里(在分行公路的左车道行驶,同时有其他车辆)。英里(在分行公路的左车道行驶,同时有其他车辆)。
\section{人工神经网络适用问题}
\label{sec-2}


\begin{itemize}
\item 实例是用很多“属性-值”对表示的。
  要学习的目标函数是定义在可以用向量描述的实例之上的,向量由预先定义的特征组成,例如ALVINN例子中的像素值。这些输入属性之间可以高度相关,也可以相互独立。输入值可以是任何实数。
\item 目标函数的输出可能是离散值、实数值或者由若干实数属性或离散属性组成的向量。
  例如,在ALVINN系统中输出的是30个属性的向量,每一个分量对应一个建议的驾驶方向。每个输出值是0和1之间的某个实数,对应于在预测相应驾驶方向时的置信度(confidence)。我们也可以训练一个单一网络,同时输出行驶方向和建议的加速度,这只要简单地把编码这两种输出预测的向量连接在一起就可以了。
\item 训练数据可能包含错误。ANN学习算法对于训练数据中的错误有非常好的鲁棒性。
\item 可容忍长时间的训练。
  网络训练算法通常比像决策树学习这样的算法需要更长的训练时间。训练时间可能从几秒钟到几小时,这要看网络中权值的数量、要考虑的训练实例的数量、以及不同学习算法参数的设置等因素。
\item 可能需要快速求出目标函数值。
  尽管ANN的学习时间相对较长,但对学习的网络求值,以便把网络应用到后续的实例,通常是非常快速的。例如,ALVINN在车辆向前行驶时,每秒应用它的神经网络若干次,以不断地更新驾驶方向。
\item 人类能否理解学到的目标函数是不重要的。
  神经网络方法学习到的权值经常是人类难以解释的。学到的神经网络比学到的规则难于传达给人类。
\end{itemize}
\section{感知器}
\label{sec-3}


\includegraphics[width=.9\linewidth]{./image/perceptron.png}

\[o(x_{1}, \ldots, x_{n}) = \left\{ \begin{array}{rl}
     1 & \mbox{if $w_{0} + w_{1}x_1 + \cdots + w_n x_n > 0$}\\
     -1 & \mbox{otherwise.}  
\end{array}
\right. \]

简化表示:

\[
o(\vec{x}) = \left\{ \begin{array}{rl}
     1 & \mbox{if $\vec{w} \cdot \vec{x} > 0$}\\
     -1 & \mbox{otherwise.}  
\end{array}
\right. 
\]
\section{两输入感知器的决策平面}
\label{sec-4}


\includegraphics[width=.9\linewidth]{./image/ann-linearly-separable.png}
\section{感知器训练法则}
\label{sec-5}



\[w_i \leftarrow w_i + \Delta w_i \]
where
\[ \Delta w_{i} = \eta (t - o) x_{i} \]

其中:

\begin{itemize}
\item $t=c(\vec{x})$ 是当前训练样例的目标输出
\item $o$ 是感知器的输出
\item $\eta$ 是一个正的常数称为学习速率(learning rate)
\end{itemize}


在有限次使用感知器训练法则后,训练过程会收敛到一个能正确分类所有训练样例的权向量,前提:
\begin{itemize}
\item 训练样例线性可分,
\item 使用了充分小的 $\eta$ (参见Minskey \& Papert 1969)
\end{itemize}
\subsection{梯度下降和delta法则}
\label{sec-5-1}


考虑线性单元:

\[ o = w_{0} + w_{1}x_1 + \cdots + w_n x_n \]

学习使均方误差

\[ E[\vec{w}] \equiv  \frac{1}{2}\sum_{d \in D}(t_{d} - o_{d})^{2} \]

最小的  $w_{i}$ 。其中 $D$ 训练样例集合。

梯度下降算法

\[ \nabla E[\vec{w}] \equiv \left[\frac{\partial E}{\partial w_{0}},
\frac{\partial E}{\partial w_{1}}, \cdots \frac{\partial E}{\partial
w_{n}}\right] \]

训练法则:

\[\Delta \vec{w} = -\eta \nabla E[\vec{w}] \]

或:

\[\Delta w_{i} = -\eta  \frac{\partial E}{\partial w_{i}}\]

推导:

\begin{eqnarray}
\frac{\partial E}{\partial w_{i}} & = & \frac{\partial}{\partial w_{i}} \frac{1}{2}\sum_{d}(t_{d} - o_{d})^{2} \nonumber\\
 & = & \frac{1}{2}\sum_{d}\frac{\partial}{\partial w_{i}}
  (t_{d} - o_{d})^{2} \nonumber\\
 & = & \frac{1}{2}\sum_{d} 2 (t_{d} - o_{d}) 
\frac{\partial}{\partial w_{i}}(t_{d} - o_{d}) \nonumber\\
 & = & \sum_{d} (t_{d} - o_{d}) 
\frac{\partial}{\partial w_{i}}(t_{d} - \vec{w} \cdot \vec{x_{d}}) \nonumber\\
\frac{\partial E}{\partial w_{i}} & = & \sum_{d} (t_{d} - o_{d}) (- x_{i,d}) \nonumber
\end{eqnarray}
\subsection{训练线性单元的梯度下降算法}
\label{sec-5-2}


Gradient-Descent( $training\_examples$ , $eta$)
training$_{\mathrm{examples}}$中每一个训练样例形式为序偶 $\langle \vec{x}, t \rangle$ ,其中 $\vec{x}$ 是输入值向量, $t$ 是目标输出值。$eta$ 是学习速率(例如0.05)。
\begin{itemize}
\item 初始化每个 $w_{i}$ 为某个小的随机值
\item 遇到终止条件之前,做以下操作:
\begin{itemize}
\item 初始化每个 $\Delta w_{i}$ 为0
\item 对于训练样例 $training\_examples$ 中的每个 $\langle \vec{x},t \rangle$ ,做:
\begin{itemize}
\item 把实例 $\vec{x}$ 输入到此单元,计算输出 $o$
\item 对于线性单元的每个权 $w_{i}$ ,做
            \[\Delta w_{i} \leftarrow \Delta w_{i} + \eta (t - o) x_{i}\]
\end{itemize}
\item 对于线性单元的每个权 $w_{i}$ ,做
            \[w_{i} \leftarrow w_{i} + \Delta w_{i}\]
\end{itemize}
\end{itemize}
\subsection{增量(随机)梯度下降算法}
\label{sec-5-3}


\begin{itemize}
\item 批量梯度下降:
\begin{itemize}
\item 计算梯度 $\nabla E_{D}[\vec{w}]$
\item $\vec{w} \leftarrow \vec{w} -\eta \nabla E_{D}[\vec{w}]$
\end{itemize}
\item 增量梯度下降:
\item 对训练集 $D$ 中的样例 $d$
\begin{itemize}
\item 计算梯度 $\nabla E_{d}[\vec{w}]$
\item $\vec{w} \leftarrow \vec{w} -\eta \nabla E_{d}[\vec{w}]$
\end{itemize}
\end{itemize}

其中
       \[E_{D}[\vec{w}] \equiv  \frac{1}{2}\sum_{d \in D}(t_{d} - o_{d})^{2}\]
       \[E_{d}[\vec{w}] \equiv  \frac{1}{2}(t_{d} - o_{d})^{2}\]
\section{多层网络和反向传播算法}
\label{sec-6}


  \includegraphics[width=.9\linewidth]{./image/ann-lippmann.png}
\subsection{Sigmoid 单元}
\label{sec-6-1}


\includegraphics[width=.9\linewidth]{./image/ann-sigmoid.png}

$$\sigma(x)= \frac{1}{1 + e^{-x}} $$

$$\frac{d \sigma(x)}{dx} = \sigma(x) (1 - \sigma(x))$$


可得梯度下降法则用于训练:
\begin{itemize}
\item 单个 sigmoid 单元
\item 由 sigmoid 单元构成的多层网络  $\rightarrow$ 反向传播(Backpropagation)
\end{itemize}
\subsection{Sigmoid 单元的误差梯度}
\label{sec-6-2}


\begin{eqnarray}
\frac{\partial E}{\partial w_{i}} & = & \frac{\partial}{\partial w_{i}}\ 
\frac{1}{2}\sum_{d\in D}(t_{d} - o_{d})^{2} \nonumber\\
 & = & \frac{1}{2}\sum_{d}\frac{\partial}{\partial w_{i}}
  (t_{d} - o_{d})^{2} \nonumber\\
 & = & \frac{1}{2}\sum_{d} 2 (t_{d} - o_{d}) \ 
\frac{\partial}{\partial w_{i}}(t_{d} - o_{d}) \nonumber\\
 & = & \sum_{d} (t_{d} - o_{d}) \left( - \frac{\partial o_{d}}{\partial
w_{i}}\right) \nonumber\\
& = & - \sum_{d} (t_{d} - o_{d})\ \frac{\partial o_{d}}{\partial
net_{d}}\ \frac{\partial net_{d}}{\partial w_{i}} \nonumber
\end{eqnarray}

已知:
\[\frac{\partial o_{d}}{\partial net_{d}} = \frac{\partial
\sigma(net_{d})}{\partial net_{d}} =  o_{d}(1 -  o_{d})  \]

\[\frac{\partial net_{d}}{\partial w_{i}} = \frac{\partial (\vec{w} \cdot
\vec{x}_{d})}{\partial w_{i}} = x_{i,d} \]

得:
\begin{eqnarray}
\frac{\partial E}{\partial w_{i}} & = & - \sum_{d \in D} (t_{d} - o_{d})
o_{d}(1-o_{d}) x_{i,d} \nonumber
\end{eqnarray}
\subsection{反向传播算法}
\label{sec-6-3}


Backpropagation( $training\_examples$ , $\eta$ , $n_{in}$ , $n_{out}$ , $n_hidden$ )
trainning$_{\mathrm{exaples}}$中每一个训练样例是形式为\$<x$_d$,t>\$的序偶,其中是网络输入值向量,是目标输出值。
 $\eta$ 是学习速率(例如0.05)。 $n_{in}$ 是网络输入的数量, $n_hidden$ 是隐藏层单元数, $n_out$ 是输出单元数。
从单元i到单元j的输入表示为 $x_{ji}$ ,单元i到单元j的权值表示为 $w_{ij}$ 。
\begin{itemize}
\item 创建具有nin个输入,nhidden个隐藏单元,nout个输出单元的网络
\item 初始化所有的网络权值为小的随机值(例如-0.05和0.05之间的数)
\item 在遇到终止条件前,做
\begin{itemize}
\item 对于训练样例training$_{\mathrm{examples}}$中的每个<,>,做
\item 把输入沿网络前向传播
\begin{itemize}
\item 把实例输入网络,并计算网络中每个单元u的输出ou。
\end{itemize}
\item 使误差沿网络反向传播
\begin{itemize}
\item 对于网络的每个输出单元k,计算它的误差项 $\delta_{k}$
                        \[\delta_{k} \leftarrow o_{k}(1-o_{k})(t_{k}-o_{k})\]
\end{itemize}
\item 对于网络的每个隐藏单元 \$h\$,计算它的误差项 $\delta_h$
                      \[\delta_{h} \leftarrow o_{h}(1-o_{h})\sum_{k \in outputs}w_{h,k}\delta_{k}\]
\item 更新每个网络权值 $w_{i,j}$
                      \[w_{i,j} \leftarrow w_{i,j} + \Delta w_{i,j}\]
            其中
                      \[\Delta w_{i,j} = \eta \delta_{j} x_{i,j}\]
\end{itemize}
\end{itemize}

\end{document}
