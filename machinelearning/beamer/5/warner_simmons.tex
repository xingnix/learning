\documentclass{article}
\usepackage{CJK}
\usepackage{amsmath}

%%%%%%%%%% Start TeXmacs macros
\newcommand{\tmmisc}[1]{\thanks{\textit{Misc:} #1}}
\newcommand{\tmop}[1]{\ensuremath{\operatorname{#1}}}
\newcommand{\tmstrong}[1]{\textbf{#1}}
%%%%%%%%%% End TeXmacs macros

\begin{document}
\begin{CJK*}{UTF8}{gbsn}

\title{
  {\tmstrong{Note on Randomized Response Technique,RRT}}
  \tmmisc{xingnix@live.com}
}

\author{Xing Chao}

\maketitle

对敏感性问题的调查方案,为了能保护个人秘密,
使被调查者愿意做出真实回答,产生了相应的统计分析方法------随机化回答技术,包括:
\begin{itemize}
  \item Warner model
  
  \item Simmons model
  
  \ 
\end{itemize}

\section{Warner model}

\ \ \ 由美国统计学家Warner提出,
针对要调查的敏感性问题,列出正反两个问题。如:调查考试作弊问题时,出示两个问题:

A:你考试作弊?

B:你考试没有作弊?

然后由被调查者按照给定的概率随机选择一个问题,与自己情况一致则回答"是",否则回答"否"。
调查者不知道被调查者在回答哪个问题,保证了被调查者个人秘密不被泄露。需要注意的是:选择问题A的概率是确定的。

理论分析如下:

设:

$P (A)$: 选择问题A的概率,

$P (B)$: 选择问题B的概率,

则有:
\begin{eqnarray*}
  P (A) + P (B) & = & 1
\end{eqnarray*}
\ \ \ \ \ 设:

$P (T)$: 回答''是''的概率,

$P (T|A)$: 选择问题A并回答''是''的概率,

$P (T|B)$: 选择问题B并回答''是''的概率,
\begin{eqnarray*}
  P (T|A) + P (T|B) & = & 1
\end{eqnarray*}
\ \ \ \ \
如果对n个人进行调查,调查结果中有$m$个人回答"是",有$n
- m$个人回答"否",可得:
\begin{eqnarray*}
  P (T|A) P (A) + P (T|B) P (B) & = & P (T)\\
  P (T|A) P (A) + (1 - P (T|A)) (1 - P (A)) & = & P (T)\\
  P (T|A) & = & \frac{P (T) + P (A) - 1}{2 P (A) - 1}\\
  \hat{P} (T|A) & = & \frac{\hat{P} (T) + P (A) - 1}{2 P (A) - 1}\\
  & = & \frac{\frac{_m}{n} + P (A) - 1}{2 P (A) - 1}
\end{eqnarray*}


根据二项分布性质可得:

期望
\begin{eqnarray*}
  E [\hat{P} (T)] & = & P (T)\\
  E [\hat{P} (T|A)] & = & P (T|A)
\end{eqnarray*}
\ \ \ \ \ 方差
\begin{eqnarray*}
  \tmop{Var} [\hat{P} (T)] & = & \frac{P (T) (1 - P (T))}{n}\\
  \tmop{Var} [\hat{P} (T|A)] & = & \frac{P (T) (1 - P (T))}{n (2 P (A) - 1)^2}
\end{eqnarray*}

\section{Simmons model}

\ \ \
Warner的方法中回答A的人数比例不能为$\frac{1}{2}$。1967年Simmons对Warner模型进行了改进。调查人员提出两个不相关的问题,其中一个为敏感性问题A,另一个为非敏感性问题B。如:调查考试作弊问题时,出示两个问题:

A:你考试作弊?

B:在心里想一个数字,它是否是奇数?

然后由被调查者按照给定的概率随机选择一个问题。

理论分析如下:

设:

$P (A)$: 选择问题A的概率,

$P (B)$: 选择问题B的概率,

得:
\begin{eqnarray*}
  P (A) + P (B) & = & 1
\end{eqnarray*}
\ \ \ \ \ 设:

$P (T)$: 回答''是''的概率,

$P (T|A)$: 选择问题A并回答''是''的概率,

$P (T|B)$: 选择问题B并回答''是''的概率,

得:
\begin{eqnarray*}
  P (T|A) P (A) + P (T|B) P (B) & = & P (T)\\
  P (T|A) P (A) + P (T|B) (1 - P (A)) & = & P (T)\\
  P (T|A) & = & \frac{P (T) - P (T|B) (1 - P (A))}{P (A)}\\
  \hat{P} (T | A |) & = & \frac{\hat{P} (T) - P (T|B) (1 - P (A))}{P (A)}
\end{eqnarray*}


根据二项分布性质可得:

期望
\begin{eqnarray*}
  E [\hat{P} (T | A |)] & = & P (T|A)
\end{eqnarray*}
\ \ \ \ \ \ 方差
\begin{eqnarray*}
  \tmop{Var} [\hat{P} (T)] & = & \frac{P (T) (1 - P (T))}{n}\\
  \tmop{Var} [\hat{P} (T | A |)] & = & \frac{P (T) (1 - P (T))}{n (P (A))^2}
\end{eqnarray*}

\end{CJK*}
\end{document}
