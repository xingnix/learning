\documentclass{article}
\usepackage[english]{babel}
\usepackage{amsmath,amssymb}

%%%%%%%%%% Start TeXmacs macros
\catcode`\<=\active \def<{
\fontencoding{T1}\selectfont\symbol{60}\fontencoding{\encodingdefault}}
\newcommand{\mathd}{\mathrm{d}}
\newcommand{\tmop}[1]{\ensuremath{\operatorname{#1}}}
%%%%%%%%%% End TeXmacs macros

\begin{document}

\title{Mean \& Median}

\author{Xing Chao}

\maketitle

How to interprete the data $\{ x_i \}_{i = 1}^n$ with only one value $c$ ?
\begin{eqnarray*}
  e_1 & = & \sum_{i = 1}^n (x_i - c)^2\\
  e_2 & = & \sum_{i = 1}^n | x_i - c |
\end{eqnarray*}
$e_1, e_2$ are two kinds of error functions. Different $\hat{c}$ (estimated
value) should be assigned to $c$ in order to get the minimum of $e_1, e_2$.

In order to minimize $e_1$
\begin{eqnarray*}
  \frac{\mathd e_1}{\mathd c} & = & \sum_{i = 1}^n 2 (c - x_i) = 0\\
  c & = & \frac{1}{n} \sum_{i = 1}^n x_i
\end{eqnarray*}
In order to minimize $e_2$, let's first sort $\{ x_i \}_{i = 1}^n$ in ascend
order. When n is odd,
\begin{eqnarray*}
  e_2 & = & \sum_{i = 1}^{\frac{n + 1}{2} - 1} | x_i - c | + \sum_{i = \frac{n
  + 1}{2} + 1}^n | x_i - c | + \left| x_{\frac{n + 1}{2}} - c \right|\\
  & \geqslant & \sum_{i = 1}^{\frac{n + 1}{2} - 1} | x_i - x_{n + 1 - i} | +
  \left| x_{\frac{n + 1}{2}} - c \right|
\end{eqnarray*}
When $c = x_{\frac{n + 1}{2}}$ , $e_2$ reaches minimum. There is a very simple
example with $\{ 1, 2, 10 \}$ as
\begin{eqnarray*}
  e_2 & = & | 1 - c | + | 2 - c | + | 10 - c |
\end{eqnarray*}
When $c \in [1, 10]$
\begin{eqnarray*}
  e_2 & = & (c - 1 + 10 - c) + | 2 - c |
\end{eqnarray*}
which is smaller than cases when $c < 1$ or $c > 10$.
\begin{eqnarray*}
  \arg \min_c e_2 & = & 2
\end{eqnarray*}
Median is superior than mean when there are outliers in the data. But mean
value as its own merits, such as with some distributions mean estimate has
less variance than that of median estimate.

Suppose uniform random variable $x \in [- 1, 1]$. Three samples are observed
as $\{ x_1, x_2, x_3 \}$. The distribution density function of median is
\begin{eqnarray*}
  f_{x_{(2)}} (x) & = & 3 \left( \frac{x + 1}{2} \right) \left( \frac{1 -
  x}{2} \right)
\end{eqnarray*}
which is deduced from distribution function of k-th order statistics
\begin{eqnarray*}
  f_{x_{(k)}} (x) & = & \frac{n!}{(k - 1) ! (n - k) !} (F (x))^{k - 1} (1 - F
  (x))^{n - k} f (x)
\end{eqnarray*}
Expectation of median estimate is 0 and variance is
\begin{eqnarray*}
  \tmop{var}_{x_{(k)}} (x) & = & \int_{- 1}^1 x^2 f_{x_{(2)}} (x) \mathd x\\
  & = & \int_{- 1}^1 \frac{3 x^2 (1 - x^2)}{4} \mathd x\\
  & = & \frac{1}{5}
\end{eqnarray*}
Compare it with mean estimate variance,
\begin{eqnarray*}
  \tmop{var}_{\tmop{mean}} (x) & = & \tmop{var} \left( \frac{\sum_{i = 1}^3
  x_i}{3} \right)\\
  & = & \frac{\sum_{i = 1}^3 \tmop{var} (x_i)}{9}\\
  & = & \frac{1}{9}
\end{eqnarray*}
where
\begin{eqnarray*}
  \tmop{var} (x_i) & = & \int_{- 1}^1 \frac{x^2}{2} \mathd x\\
  & = & \frac{1}{3}
\end{eqnarray*}
The result of comparison is
\begin{eqnarray*}
  \tmop{var}_{\tmop{mean}} (x) & < & \tmop{var}_{x_{(2)}} (x)
\end{eqnarray*}

\end{document}
