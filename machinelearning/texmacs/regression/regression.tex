\documentclass{beamer}
\usepackage{CJK}
\usepackage{amsmath,amssymb,graphicx,enumerate}

%%%%%%%%%% Start TeXmacs macros
\newcommand{\Tau}{\mathrm{T}}
\newcommand{\tmmathbf}[1]{\ensuremath{\boldsymbol{#1}}}
\newcommand{\tmop}[1]{\ensuremath{\operatorname{#1}}}
\newenvironment{enumeratenumeric}{\begin{enumerate}[1.] }{\end{enumerate}}
%%%%%%%%%% End TeXmacs macros

\begin{document}
\begin{CJK*}{UTF8}{gbsn}

{\screens{\begin{frame}
  \
  
  \
  
  \
  
  \
  
  \
  
  \title{机器学习}
  
  \maketitle
  
  \ 
\end{frame}}{\begin{frame}
  \frametitle{最小二乘学习法}
  
  \
  
  \qquad以$d$维实向量$\tmmathbf{x}$作为输入、以实数值$y$作为输出的函数$y
  = f (\tmmathbf{x})$的学习问题进行说明。
  
  \qquad这里的真实关系$f$是未知的,通过学习过程中作为训练集而输入输出的训练样本$\{
  \tmmathbf{x}_i, y_i \}_{i = 1}^n$来对其进行学习。但是
  在一般情况下,输出样本$y_i$的直实值$f
  (\tmmathbf{x}_i)$中经常会观测到噪声。
\end{frame}}{\begin{frame}
  \frametitle{均方误差损失}
  
  最小二乘法习法是 对模型 的输出$f_{\tmmathbf{\theta}}
  (\tmmathbf{x}_i)$和训练集输出$\{ y_i \}_{i = 1}^n$的平方误差
  \[ J_{\tmop{LS}} (\tmmathbf{\theta}) = \frac{1}{2} \sum_{i = 1}^n
     (f_{\tmmathbf{\theta}} (\tmmathbf{x}_i) - y_i)^2 \]
  为最小时的参数$\tmmathbf{\theta}$进行学习。
  \[ \hat{\tmmathbf{\theta}}_{\tmop{LS}} = \arg \min_{\tmmathbf{\theta}}
     J_{\tmop{LS}} (\tmmathbf{\theta}) \]
  平方误差$(f_{\tmmathbf{\theta}} (\tmmathbf{x}_i) - y_i)^2$是残差
  \[ f_{\tmmathbf{\theta}} (\tmmathbf{x}_i) - y_i \]
  的$\mathcal{l}_2$范数,因此最小二乘学习法有时也称为$\mathcal{l}_2$损失最小化学习法。
\end{frame}}{\begin{frame}
  \frametitle{线性模型}
  
  如果使用线性模型
  \[ f_{\tmmathbf{\theta}} (\tmmathbf{x}) = \sum_{j = 1}^b \theta_i \phi_i
     (\tmmathbf{x}) =\tmmathbf{\theta}^T \phi (\tmmathbf{x}) \]
  则训练样本的平方差$J_{\tmop{LS}}$就能够表示为下述形式。
  \[ J_{\tmop{LS}} (\tmmathbf{\theta}) = \frac{1}{2} \|
     \tmmathbf{\Phi}\tmmathbf{\theta}-\tmmathbf{y} \|^2 \]
  在这里,$\tmmathbf{y}= (y_1, \cdots,
  y_n)^T$是训练输出的$n$维向量, $\tmmathbf{\Phi}$是$n \times
  b$矩阵,也称为设计矩阵,如下所示:
  \[ \tmmathbf{\Phi}= \left(\begin{array}{ccc}
       \phi_1 (\tmmathbf{x}_1) & \cdots & \phi_b (\tmmathbf{x}_1)\\
       \vdots & \ddots & \vdots\\
       \phi_1 (\tmmathbf{x}_n) & \cdots & \phi_b (\tmmathbf{x}_n)
     \end{array}\right) \]
\end{frame}}{\begin{frame}
  \frametitle{求解}
  
  训练样本的平方差$J_{\tmop{LS}}$的参数向量$\tmmathbf{\theta}$的偏微分$\nabla_{\tmmathbf{\theta}}
  J_{\tmop{LS}}$为
  \[ \nabla_{\tmmathbf{\theta}} J_{\tmop{LS}} = \left( \frac{\partial
     J_{\tmop{LS}}}{\partial \theta_1}, \cdots, \frac{\partial
     J_{\tmop{LS}}}{\partial \theta_b} \right)^T =\tmmathbf{\Phi}^T
     \tmmathbf{\Phi}\tmmathbf{\theta}-\tmmathbf{\Phi}^T \tmmathbf{y} \]
  令此微分为0,得最小二乘解满足
  \[ \tmmathbf{\Phi}^T \tmmathbf{\Phi}\tmmathbf{\theta}=\tmmathbf{\Phi}^T
     \tmmathbf{y} \]
  此方程的解$\hat{\tmmathbf{\theta}}_{\tmop{LS}}$使用设计矩阵$\tmmathbf{\Phi}$的广义逆矩阵$\tmmathbf{\Phi}^{\dag}$来进行计算,得
  \[ \hat{\tmmathbf{\theta}}_{\tmop{LS}} =\tmmathbf{\Phi}^{\dag} \tmmathbf{y}
  \]
\end{frame}}{\begin{frame}
  \frametitle{加权最小二乘}
  
  对顺序为$i$的训练样本的平方差通过权重$w_i \geqslant
  0$进行加权,然后再采用最小二乘法学习,称为加权最小二乘学习法。
  \[ \min_{\tmmathbf{\theta}} \frac{1}{2} \sum_{i = 1}^n w_i
     (f_{\tmmathbf{\theta}} (\tmmathbf{x}_i) - y_i)^2 \]
  加权最小二乘法,与没有权重时相同,
  \[ (\tmmathbf{\Phi}^T \tmmathbf{W}\tmmathbf{\Phi})^{\dag} \tmmathbf{\Phi}^T
     \tmmathbf{W}\tmmathbf{y} \]
  上式中的$\tmmathbf{W}$是以$w_1, \cdots,
  w_n$为对角元素的对角矩阵。
\end{frame}}{\begin{frame}
  \frametitle{核模型}
  
  核模型
  \[ f_{\tmmathbf{\theta}} (\tmmathbf{x}) = \sum_{j = 1}^n \theta_j K
     (\tmmathbf{x}, \tmmathbf{x}_j) \]
  也可以认为是线性模型的一种,把设计矩阵$\tmmathbf{\Phi}$置换为核矩阵$K$,就可以使用和线性模型相同的方法来求得核模型的最小二乘解
  \[ K = \left(\begin{array}{ccc}
       K (\tmmathbf{x}_1, \tmmathbf{x}_1) & \cdots & K (\tmmathbf{x}_1,
       \tmmathbf{x}_n)\\
       \vdots & \ddots & \vdots\\
       K (\tmmathbf{x}_n, \tmmathbf{x}_1) & \cdots & K (\tmmathbf{x}_n,
       \tmmathbf{x}_n)
     \end{array}\right) \]
  
\end{frame}}{\begin{frame}
  \frametitle{最小二乘解的性质}
  
  设计矩阵$\tmmathbf{\Phi}$的奇异值分解
  \[ \tmmathbf{\Phi}= \sum_{k = 1}^{\min (n, b)} \kappa_k \tmmathbf{\psi}_k
     \tmmathbf{\varphi}_k^T \]
  其中$\kappa_k, \tmmathbf{\psi}_k,
  \tmmathbf{\varphi}_k$分别称为奇异值,左奇异向量,右奇异向量。奇异值全部是非负的,奇异向量满足正交性
  \[ \tmmathbf{\psi}_i^T \tmmathbf{\psi}_{i'} = \text{}
     \left\{\begin{array}{l}
       1 \quad (i = i')\\
       0 \quad (i \neq i')
     \end{array}\right. \quad \tmmathbf{\varphi}_j^T
     \tmmathbf{\varphi}_{_{j'}} = \left\{\begin{array}{l}
       1 \quad (j = j')\\
       0 \quad (j \neq j')
     \end{array}\right. \]
  
\end{frame}}{\begin{frame}
  \frametitle{奇异值分解}
  
  进行奇异值分解后,$\tmmathbf{\Phi}$的广义逆矩阵$\tmmathbf{\Phi}^{\dag}$就可以表示为
  \[ \tmmathbf{\Phi}^{\dag} = \sum_{k = 1}^{\min (n, b)} \kappa_k^{\dag}
     \tmmathbf{\varphi}_k \tmmathbf{\psi}_k^T \]
  其中$\kappa^{\dag}$是标量$\kappa$的广义逆矩阵
  \[ \kappa^{\dag} = \left\{\begin{array}{l}
       \frac{1}{\kappa} \quad (\kappa \neq 0)\\
       0 \quad (\kappa = 0)
     \end{array}\right. \]
  最小二乘解$\hat{\tmmathbf{\theta}}_{\tmop{LS}}$可以表示为
  \[ \hat{\tmmathbf{\theta}}_{\tmop{LS}} = \sum_{k = 1}^{\min (n, b)}
     \kappa_k^{\dag} (\tmmathbf{\psi}_k^T \tmmathbf{y}) \tmmathbf{\varphi}_k
  \]
\end{frame}}{\begin{frame}
  \frametitle{无偏性}
  
  将最小二乘学习法中得到的函数以训练输入$\{ \tmmathbf{x}_i
  \}_{i = 1}^n$得到输出值$\{ f_{{\hat{\tmmathbf{\theta}}_{\tmop{LS}}} 
  (\tmmathbf{x}_i)} \}_{i = 1}^n$,变换为列向量表示,得
  \[ (f_{\hat{\tmmathbf{\theta}}_{\tmop{LS}}} (\tmmathbf{x}_1), \cdots
     f_{\hat{\tmmathbf{\theta}}_{\tmop{LS}}} (\tmmathbf{x}_n))^T
     =\tmmathbf{\Phi} \hat{\tmmathbf{\theta}}_{\tmop{LS}}
     =\tmmathbf{\Phi}\tmmathbf{\Phi}^{\dag} \tmmathbf{y} \]
  其中$\tmmathbf{\Phi}\tmmathbf{\Phi}^{\dag}$是$\tmmathbf{\Phi}$的值域$\mathcal{R}
  (\tmmathbf{\Phi})$的正交投影矩阵,最小二乘学习法的输出向量是由$\mathcal{R}
  (\tmmathbf{\Phi})$的正交投影得到的。
  
  如果噪声的期望值为0,则最小二乘解$\hat{\tmmathbf{\theta}}_{\tmop{LS}}$就是真实参数$\tmmathbf{\theta}^{\ast}$的无偏估计量。
  \[ \mathbb{E} [\hat{\tmmathbf{\theta}}_{\tmop{LS}}]
     =\tmmathbf{\theta}^{\ast} \]
  上式中$\mathbb{E}$为对噪声的期望值。另外,即使真实函数没有包含在模型中(即无论对于什么样的$\tmmathbf{\theta}$,都存在$f
  \neq f_{\tmmathbf{\theta}}$,如果增加训练样本数$n$,$\mathbb{E}
  [\hat{\tmmathbf{\theta}}_{\tmop{LS}}]$也会向着模型中的最优参数方向收敛。这种性质称为无偏性。
\end{frame}}{\begin{frame}
  \frametitle{随机梯度法}
  
  \qquad\resizebox{0.8\columnwidth}{!}{\includegraphics{img/gradient.png}}
\end{frame}}{\begin{frame}
  \frametitle{}
  
  \
  
  随机梯度法是指,沿着训练平方误差$J_{\tmop{LS}}$的梯度下降,对参数依次进行学习的算法。
  
  \qquad一般而言,与线性模型相对应的训练平方误差$J_{\tmop{LS}}$为凸函数。$J
  (\tmmathbf{\theta})$函数为凸函数是指对任意的两点$\tmmathbf{\theta}_1,
  \tmmathbf{\theta}_2$和任意的$t \in [0, 1]$,
  \[ J (t\tmmathbf{\theta}_1 + (1 - t) \tmmathbf{\theta}_2) \leqslant
     \tmop{tJ} (\tmmathbf{\theta}_1) + (1 - t) J (\tmmathbf{\theta}_2) \]
  \qquad凸函数只有一个峰值,可以通过梯度法得到全局最优解。
\end{frame}}{\begin{frame}
  \frametitle{}
  
  \resizebox{0.8\columnwidth}{!}{\includegraphics{img/convex.png}}
\end{frame}}{\begin{frame}
  \frametitle{随机梯度最小二乘}
  
  \qquad随机梯度算法对线性模型进行最小二乘学习的算法流程
  \begin{enumeratenumeric}
    \item 给$\tmmathbf{\theta}$以适当的初值
    
    \item 随机先择一个训练样本, 如:$(\tmmathbf{x}_i, y_i)$
    
    \item
    对于选定的训练样本,采用使其梯度下降的方式,对参数$\tmmathbf{\theta}$进行更新
    \[ \tmmathbf{\theta} \leftarrow \tmmathbf{\theta}- \varepsilon \nabla
       J_{\tmop{LS}}^{(i)} (\tmmathbf{\theta}) \]
    这里$\varepsilon$是名为学习系数的小标量,表示梯度下降的步幅,$\nabla
    J_{\tmop{LS}}^{(i)}$是顺度为$i$的训练样本相对应的训练平方误差的梯度
    \[ \nabla J_{\tmop{LS}}^{(i)} (\tmmathbf{\theta}) = \phi (\tmmathbf{x}_i)
       (f_{\tmmathbf{\theta}} (\tmmathbf{x}_i) - y_i) \]
    \item
    直到解$\tmmathbf{\theta}$达到收敛精度为止,否则重复2,3
  \end{enumeratenumeric}
\end{frame}}{\begin{frame}
  \frametitle{带有约束条件的最小二乘法}
  
  \
  
  \qquad单纯的最小二乘法对于包含噪声的学习过程经常有过拟合的弱点。原因是学习模型对于训练样本而言过度复杂。利用约束条件可以控制模型复杂程序。
\end{frame}}{\begin{frame}
  \frametitle{部分空间约束的最小二乘学习法}
  
  \quad对于有参数的线性模型,
  \[ f_{\tmmathbf{\theta}} (\tmmathbf{x}) = \sum_{j = 1}^b \theta_j \phi_j
     (\tmmathbf{x}) =\tmmathbf{\theta}^{\Tau} \tmmathbf{\phi}_j (\tmmathbf{x})
  \]
  参数$\{ \theta_j \}_{j =
  1}^b$可以自由设置,利用部分空间约束的最小二乘法,通过把参数空间限制在一定范围内,防止过拟合现象。
  \[ \min_{\tmmathbf{\theta}} J_{\tmop{LS}} (\tmmathbf{\theta}) \hspace{4em}
     s.t. \quad \tmmathbf{P}\tmmathbf{\theta}=\tmmathbf{\theta} \]
  其中$\tmmathbf{P}$是满足$\tmmathbf{P}^2
  =\tmmathbf{P}$和$\tmmathbf{P}^T =\tmmathbf{P}$的$b \times
  b$维矩阵,表示的是矩阵$\tmmathbf{P}$的值域$\mathcal{R}
  (\tmmathbf{P})$的正交投影矩阵。通过附加约束条件$\tmmathbf{P}\tmmathbf{\theta}=\tmmathbf{\theta}$,
  参数$\tmmathbf{\theta}$不会偏 移到值域$\mathcal{R}
  (\tmmathbf{P})$的范围外。部分空间约束的最小二乘学习法的解$\hat{\tmmathbf{\theta}}$,一般是通过将最小二乘学习的设计矩阵$\tmmathbf{\Phi}$置换为$\tmmathbf{\Phi}\tmmathbf{P}$的方式求得
  \[ \hat{\tmmathbf{\theta}} = (\tmmathbf{\Phi}\tmmathbf{P})^{\dag}
     \tmmathbf{y} \]
\end{frame}}{\begin{frame}
  \frametitle{$\mathcal{l}_2$约束的最小二乘学习法}
  
  拉格朗日对偶问题:
  
  可微分的凸函数$f : \mathbb{R}^d \rightarrow
  \mathbb{R}$和$\tmmathbf{g}: \mathbb{R}^d \rightarrow
  \mathbb{R}^p$的约束条件的最小化问题
  \[ \min_{\tmmathbf{t}} f (\tmmathbf{t}) \hspace{3em} s.t. \quad \tmmathbf{g}
     (\tmmathbf{t}) \leqslant 0 \]
  可定义拉格朗日对偶问题:
  
  \ 
\end{frame}}{\begin{frame}
  \frametitle{}
  
  使用拉格朗日待定因子:
  \[ \tmmathbf{\lambda}= (\lambda_1, \cdots, \lambda_p)^T  \]
  和拉格朗日函数:
  \[ L (\tmmathbf{t}, \tmmathbf{\lambda}) = f (\tmmathbf{t})
     +\tmmathbf{\lambda}^T \tmmathbf{g} (\tmmathbf{t}) \]
  采用以下方式进行定义:
  \[ \max_{\tmmathbf{\lambda}} \inf_{\tmmathbf{t}} L (\tmmathbf{t},
     \tmmathbf{\lambda}) \hspace{4em} s.t. \quad \tmmathbf{\lambda} \geqslant
     0 \]
  拉格朗日对偶问题的$\tmmathbf{t}$的解,与原来的问题的解是一致的。
\end{frame}}{\begin{frame}
  \frametitle{}
  
  部分空间约束的最小二乘法中,只使用了参数空间的一部分,但是由于正交投影矩阵$P$的设置有很大的自由度,因此在实际应用中操作起来有很大难度。$\ell_2$约束的最小二乘学习法相对容易操作。
  \[ \min_{\tmmathbf{\theta}} J_{\tmop{LS}} (\tmmathbf{\theta}) \hspace{4em}
     s.t. \quad \| \tmmathbf{\theta} \|^2 \leqslant R \]
  $\ell_2$约束的最小二乘学习法以参数空间的原点为圆心,在一定半径范围的圆(一般为超球)内进行求解。$R$表示的即是圆的半径。
  
  \ 
\end{frame}}{\begin{frame}
  \frametitle{}
  
  利用拉格朗日对偶问题,通过求解下式的最优解,得到原最优化问题的解。
  \[ \max_{\tmmathbf{\lambda}} \min_{\tmmathbf{\theta}}  \left[ J_{\tmop{LS}}
     (\tmmathbf{\theta}) + \frac{\lambda}{2} (\| \tmmathbf{\theta} \|^2 - R)
     \right] \hspace{4em} s.t. \lambda \geqslant 0 \]
  拉格朗日对偶问题的拉格朗日待定因子$\lambda$的解由圆的半径$R$决定,如果不根据$R$来决定$\lambda$,
  而是直接指定$\lambda$的值,$\ell_2$约束的最小二乘学习法的解$\hat{\tmmathbf{\theta}}$就可以通过下式求得:
  \[ \hat{\tmmathbf{\theta}} = \arg \min_{\tmmathbf{\theta}} \left[
     J_{\tmop{LS}} (\tmmathbf{\theta}) + \frac{\lambda}{2} \|
     \tmmathbf{\theta} \|^2 \right] \]
  
\end{frame}}{\begin{frame}
  \frametitle{}
  
  其中第一项$J_{\tmop{LS}}
  (\tmmathbf{\theta})$表示的是对训练样本的拟合程度,通过与第二项的$\frac{\lambda}{2}
  \| \tmmathbf{\theta}
  \|^2$相结合得到最小值,可防止对训练样本的过拟合。
  
  令关于参数$\tmmathbf{\theta}$的偏微分等于0,可得:
  \[ \hat{\tmmathbf{\theta}} = (\tmmathbf{\Phi}^T \tmmathbf{\Phi}+ \lambda
     \tmmathbf{I})^{- 1} \tmmathbf{\Phi}^T \tmmathbf{y} \]
  $\ell_2$约束的最小二乘法通过将矩阵$\tmmathbf{\Phi}^T
  \tmmathbf{\Phi}$与$\lambda
  I$相加提高了其正则性,进而可以更稳定地进行逆矩阵的求解。因此,$\ell_2$约束的最小二乘学习法也称为$\ell_2$正则化的最小二乘法。$\|
  \tmmathbf{\theta}
  \|^2$称为正则项,$\lambda$为正则化参数。$\ell_2$正则化的最小二乘法在有些著作中也称为岭回归。
\end{frame}}{\begin{frame}
  \frametitle{}
  
  考虑设计矩阵$\tmmathbf{\Phi}$的奇异值分解:
  \[ \tmmathbf{\Phi}= \sum_{k = 1}^{\min (n, b)} \kappa_k \tmmathbf{\psi}_k
     \tmmathbf{\varphi}_k^T \]
  $\ell_2$约束的最小二乘法的解可表示为:
  \[ \hat{\tmmathbf{\theta}} = \sum_{k = 1}^{\min (n, b)}
     \frac{\kappa_k}{\kappa_k^2 + \lambda} (\tmmathbf{\psi}_k^T \tmmathbf{y})
     \tmmathbf{\varphi}_k \]
\end{frame}}{\frametitle{}

当$\lambda =
0$时,$\ell_2$约束的最小二乘法就与一般的最小二乘法相同,当设计矩阵$\tmmathbf{\Phi}$的计算条件很恶劣,即包含非常小的奇异值$\kappa_k$时,$\frac{\kappa_k}{\kappa_k^2}
=
\frac{1}{\kappa_k}$会非常大,训练输出向量$\tmmathbf{y}$包含的噪声会增大,$\ell_2$约束最小二乘法,$\frac{\kappa_k}{\kappa_k^2
+
\lambda}$由于正常数$\lambda$的作用,不会变得过大,进而可以达到防止过拟合的目的。

通过使用$b \times
b$的正则化矩阵$G$,就可以得到更普遍的表示方法
\[ \min_{\tmmathbf{\theta}} J_{\tmop{LS}} (\tmmathbf{\theta}) \hspace{4em}
   s.t. \quad \tmmathbf{\theta}^T \tmmathbf{G}\tmmathbf{\theta} \leqslant R \]
上述表示方式称为一般$\ell_2$约束的最小二乘法。矩阵$\tmmathbf{G}$为对称正定矩阵时,$\tmmathbf{\theta}^T
\tmmathbf{G}\tmmathbf{\theta} \leqslant
R$可以把参数限制在椭圆形的区域内。一般$\ell_2$约束的最小二乘法的解:
\[ \hat{\tmmathbf{\theta}} = (\tmmathbf{\Phi}^T \tmmathbf{\Phi}+ \lambda
   \tmmathbf{G})^{- 1} \tmmathbf{\Phi}^T \tmmathbf{y} \]}}

\end{CJK*}
\end{document}
